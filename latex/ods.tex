\documentclass[11pt]{book}
\usepackage{amsopn}
\usepackage{amsmath}
\usepackage{amsthm}
\usepackage{url}
\usepackage{graphicx}

\usepackage[sf,bf,small,raggedright,compact]{titlesec} % formatting titles
\usepackage{fancyvrb}  % formatting pseudocode
\usepackage{ods} % Personalization and commands

% Colors for syntax highlighting
\usepackage[usenames,dvipsnames]{color}
\definecolor{keyword}{named}{RoyalPurple}
\definecolor{var}{named}{Blue}
\definecolor{comment}{named}{ForestGreen}
\definecolor{boxcolor}{named}{Gray}

% Title page content
\title{Basic Data Structures\\(in Java)}
\author{Pat Morin}
\date{Version 0.0 pre $\alpha$ : \today}

\pagenumbering{roman}

% Draft mode only - mark overfull hboxes
\overfullrule=5pt

\begin{document}
\begin{titlepage}
  \maketitle
\end{titlepage}

% blank page behind title page
\ \thispagestyle{empty}\newpage

\setcounter{page}{1}
\section*{Why this Book?}

There are plenty of books that teach introductory data structures.
Some of them are very good.  Most of them cost money, and the vast
majority of computer science undergraduate students will shell-out at
least some cash on a data structures book.

There are a few free data structures books available online.  Some are
very good, but most of them are getting old.  The majority of these
books became free when the author and/or publisher decided to stop
updating them.  Updating these books is usually not possible, for two
reasons:  (1)~The copyright belongs to the author or publisher, who
may not allow it.  (2)~The \emph{source code} for these books is often
not available.  That is, the Word, WordPerfect, FrameMaker, or \LaTeX\
source for the book is not available, and the version of the software
that handles this source may not even be available.

The goal of this project is to forever free undergraduate computer science
students from having to pay for an introductory data structures book.
I have decided to implement this goal by treating this book like an
Open Source software project.  The \LaTeX\ source, Java source, and
build scripts for the book are available for download on the book's website
(\url{opendatastructures.org}) and also --- more importantly --- on a reliable
source code management site (\url{https://github.com/patmorin/ods}).

This source code is released under a Creative Commons Attribution license,
meaning that anyone is free
\begin{itemize}
  \item to Share --- to copy, distribute and transmit the work; and
  \item to Remix --- to adapt the work.
\end{itemize}
This includes the right to make commercial use of the work.  The only
condition on these rights is
\begin{itemize}
  \item Attribution --- You must attribute the work by displaying a
  notice stating the derived work contains code and/or text from the
  Open Data Structures Project and/or linking to
  \url{opendatastructures.org}.
\end{itemize}

Anyone can contribute corrections/fixes using the \texttt{git} source-code
management system.  Anyone can fork from the current version of the
book and develop their own version (for example, in another programming
language).  They can then ask that their changes be merged back into
my version.  My hope is that, by doing things this way, this book will
continue to be a useful textbook long after my interest in the project
(or my pulse, whichever comes first) has waned.




\tableofcontents

\chapter{Introduction}
\pagenumbering{arabic}

\section{Common Interfaces}
\subsection{The Queue, Stack, and Deque Interfaces}
\subsection{The List Interface}
\subsection{The Set and Map Interfaces}

The \mbox{\texttt{Set}} and \mbox{\texttt{Map}} interfaces are easily implemented by the following
interface:

\mbox{\texttt{add({\color{var}x})}}

\mbox{\texttt{remove({\color{var}x})}}: Remove, from the data structure the element that is equal to
\mbox{\texttt{{\color{var}x}}}, if one exists.

\mbox{\texttt{find({\color{var}x})}}: Find the element \mbox{\texttt{{\color{var}x}}} in the data structure. If there is an
element \mbox{\texttt{{\color{var}e}}} in the data structure equal to \mbox{\texttt{{\color{var}x}}}, then 
return a reference \mbox{\texttt{{\color{var}e}}}, otherwise return \mbox{\texttt{{\color{var}null}}}.


\subsection{The SortedSet and SortedMap Interfaces}

The \mbox{\texttt{SortedSet}} and \mbox{\texttt{SortedMap}} interfaces are easily implemented by the following
interface:

\mbox{\texttt{add({\color{var}x})}}

\mbox{\texttt{remove({\color{var}x})}}: Remove, from the data structure the element that is equal to
\mbox{\texttt{{\color{var}x}}}, if one exists.

\mbox{\texttt{find({\color{var}x})}}: 
Return the minimum element \mbox{\texttt{{\color{var}y}}} in the data structure such that
$\mbox{\texttt{{\color{var}y}}}\ge \mbox{\texttt{{\color{var}x}}}$. If no such element exists, then return \mbox{\texttt{{\color{var}null}}}.



\subsection{The DynamicString Interface}
\section{The Model of Computation}
\seclabel{model}
\begin{itemize}
\item Word-RAM, $c\log n$ bit words, can store values in $0,\ldots,n^{c}-1$.
\item Constant-time arithmetic, boolean, and bitwise boolean operations.
\item Constant time array accessing.  
\item Allocating an array of size \mbox{\texttt{{\color{var}n}}} takes $O(\mbox{\texttt{{\color{var}n}}})$ time.
\item Space is measured in words; items of type \mbox{\texttt{{\color{keyword}T}}}, \mbox{\texttt{{\color{keyword}K}}}, and \mbox{\texttt{{\color{keyword}V}}} take 1 word of memory each.
\end{itemize}

\section{Asymptotic Notation}

The usual stuff plus multivariate stuff.

$O(f(n_1,\ldots,n_3))$ means the set of all functions $g(n_1,\ldots,n_k)$ such that
$g(n_1,\ldots,n_k) \le c\cdot f(n_1,\ldots,n_3)$
for all $n_1,\ldots,n_k$ such that $f(n_1,\ldots,n_3)\ge n_0$.


\section{Amortization and Randomization}

\section{Code Samples}

The code samples in this book are written in the Java programming
language.  However to make the book accessible even to reader not
familiar with all of Java's constructs and keywords, the code samples have
been simplified.  For example, a reader won't find any of the keywords
\mbox{\texttt{{\color{keyword}public}}}, \mbox{\texttt{{\color{keyword}protected}}}, \mbox{\texttt{{\color{keyword}private}}}, or \mbox{\texttt{{\color{keyword}static}}}.  A reader also won't
find the \mbox{\texttt{{\color{keyword}extends}}} or \mbox{\texttt{{\color{keyword}implements}}} keywords, so any notion of class
hierarchy is missing.  Which interface a particular class implements
or which class it extends, if relevant to the discussion, will be clear
from the accompanying text.

These conventions should make most of the code samples understandable by
anyone with a background in any of the languages from the ALGOL tradition,
including B, C, C++, C\# Java, JavaScript, and so on.  Readers who want
the full details of all implementations are encouraged to look at the
Java source code that accompanies this book.



\chapter{Array-Based Lists and Queues}
\chaplabel{arrays}

In this chapter, we study implementations of the \mbox{\texttt{List}} and \mbox{\texttt{Queue}}
interfaces where the underlying data is stored in an array, called the
\emph{backing array}.  The following table summarizes the running times
of operations for the data structures presented in this chapter:

\noindent
\begin{tabular}{lcccc}
 & \mbox{\texttt{get({\color{var}i})}} & \mbox{\texttt{set({\color{var}i},{\color{var}x})}} & \mbox{\texttt{add({\color{var}i},{\color{var}x})}} & \mbox{\texttt{remove({\color{var}i})}} \\
\mbox{\texttt{ArrayStack}} & $O(1)$ & $O(1)$ & $O(1+\mbox{\texttt{{\color{var}n}}}-\mbox{\texttt{{\color{var}i}}})$ & $O(\mbox{\texttt{{\color{var}n}}}-\mbox{\texttt{{\color{var}i}}})$ \\
\mbox{\texttt{ArrayDeque}} & $O(1)$ & $O(1)$ & $O(1+\min\{\mbox{\texttt{{\color{var}i}}},\mbox{\texttt{{\color{var}n}}}-\mbox{\texttt{{\color{var}i}}}\})$ 
             & $O(1+\min\{\mbox{\texttt{{\color{var}i}}},\mbox{\texttt{{\color{var}n}}}-\mbox{\texttt{{\color{var}i}}}\})$ \\
\mbox{\texttt{DualArrayDeque}} & $O(1)$ & $O(1)$ & $O(1+\min\{\mbox{\texttt{{\color{var}i}}},\mbox{\texttt{{\color{var}n}}}-\mbox{\texttt{{\color{var}i}}}\})$ 
             & $O(1+\min\{\mbox{\texttt{{\color{var}i}}},\mbox{\texttt{{\color{var}n}}}-\mbox{\texttt{{\color{var}i}}}\})$ \\
\mbox{\texttt{RootishArrayStack}} & $O(1)$ & $O(1)$ & $O(1+\mbox{\texttt{{\color{var}n}}}-\mbox{\texttt{{\color{var}i}}})$ & $O(\mbox{\texttt{{\color{var}n}}}-\mbox{\texttt{{\color{var}i}}})$ \\
\end{tabular}

Data structures that work by storing data in a single array have many
advantages and limitations in common:
\begin{itemize}
  \item Arrays offer constant time access to any value in the array.
  This is what allows \mbox{\texttt{get({\color{var}i})}} and \mbox{\texttt{set({\color{var}i},{\color{var}x})}} to run in constant time.
  \item Arrays are not very dynamic.  Adding or removing an element
  near the middle of a list means that a large number of elements in the
  array need to be shifted to make room for the newly added element or
  to fill in the gap created by the deleted element.  This is why the operations
  \mbox{\texttt{add({\color{var}i},{\color{var}x})}} and \mbox{\texttt{remove({\color{var}i})}} have running times that depend on \mbox{\texttt{{\color{var}n}}} and \mbox{\texttt{{\color{var}i}}}.
  \item Arrays can not expand or shrink.  When the number of elements in
  the data structure exceeds the size of the backing array, a new array needs
  to be allocated and the data from the old array needs to be copied
  into the new array.  This is an expensive operation.
\end{itemize}
The third point is important.  The running times cited in the table
above do not include the cost of growing and shrinking the backing array.
We will see that, if carefully managed, the cost of growing and shrinking
the backing array does not add much to the cost of an \emph{average}
operation.  More precisely, if we start with an empty data structure,
and perform any sequence of $m$ \mbox{\texttt{add({\color{var}i},{\color{var}x})}} or \mbox{\texttt{remove({\color{var}i})}} operations,
then the total cost of growing and shrinking the backing array, over the
entire sequence of $m$ operations is $O(m)$.  Although some individual
operations require growing or shrinking the backing array, and are
therefore expensive, the amortized cost, when amortized over all $m$
operations is only $O(1)$ per operation.

\section{\mbox{\texttt{ArrayStack}}: Fast Stack Operations Using an Array}
\seclabel{arraystack}
% TODO: One big figure showing add(x), remove() and resize()
An \mbox{\texttt{ArrayStack}} implements the list interface using an array \mbox{\texttt{{\color{var}a}}}, called
the \emph{backing array}.  The list element with index \mbox{\texttt{{\color{var}i}}} is stored
in \mbox{\texttt{{\color{var}a}[{\color{var}i}]}}.  At most times, \mbox{\texttt{{\color{var}a}}} is larger than strictly necessary,
so an integer \mbox{\texttt{{\color{var}n}}} is used to keep track of the number of elements
actually stored in \mbox{\texttt{{\color{var}a}}}.  In this way, the list elements are stored in
\mbox{\texttt{{\color{var}a}[0]}},\ldots,\mbox{\texttt{{\color{var}a}[{\color{var}n}-1]}} and, at all times, $\mbox{\texttt{{\color{var}a}.{\color{var}length}}} \ge \mbox{\texttt{{\color{var}n}}}$.

\begin{Verbatim}[tabsize=2,frame=single,commandchars=\\@\$,label=\texttt{ArrayStack},labelposition=topline]
	@\color@keyword$T$[] @\color@var$a$;
	@\color@keyword$int$ @\color@var$n$;
	@\color@keyword$int$ size() {
		@\color@var$return$ @\color@var$n$;
	}
\end{Verbatim}

\subsection{The Basics}

Accessing and modifying the elements of an \mbox{\texttt{ArrayStack}} using \mbox{\texttt{get({\color{var}i})}}
and \mbox{\texttt{set({\color{var}i},{\color{var}x})}} is trivial. After performing any necessary bounds-checking
we simply return or set, respectively, \mbox{\texttt{{\color{var}a}[{\color{var}i}]}}.

\begin{Verbatim}[tabsize=2,frame=single,commandchars=\\@\$,label=\texttt{ArrayStack},labelposition=topline]
	@\color@keyword$T$ get(@\color@keyword$int$ @\color@var$i$) {
		@\color@keyword$if$ (@\color@var$i$ < 0 || @\color@var$i$ > @\color@var$n$ - 1) @\color@keyword$throw$ @\color@keyword$new$ IndexOutOfBoundsException();
		@\color@var$return$ @\color@var$a$[@\color@var$i$];
	}
	@\color@keyword$T$ set(@\color@keyword$int$ @\color@var$i$, @\color@keyword$T$ @\color@var$x$) {
		@\color@keyword$if$ (@\color@var$i$ < 0 || @\color@var$i$ > @\color@var$n$ - 1) @\color@keyword$throw$ @\color@keyword$new$ IndexOutOfBoundsException();
		@\color@keyword$T$ @\color@var$y$ = @\color@var$a$[@\color@var$i$];
		@\color@var$a$[@\color@var$i$] = @\color@var$x$;
		@\color@var$return$ @\color@var$y$;
	}
\end{Verbatim}

Adding and removing elements from an \mbox{\texttt{ArrayStack}} is illustrated
in \figref{arraystack}.  To implement the \mbox{\texttt{add({\color{var}i},{\color{var}x})}} operation, we first
check if \mbox{\texttt{{\color{var}a}}} is already full.  If so, we call the method \mbox{\texttt{resize()}}
to increase the size of \mbox{\texttt{{\color{var}a}}}.  How \mbox{\texttt{resize()}} is implemented will be
discussed later.  For now, it is sufficient to know that, after a call
to \mbox{\texttt{resize()}}, we can be sure that $\mbox{\texttt{{\color{var}a}.{\color{var}length}}} > \mbox{\texttt{{\color{var}n}}}$.  With this out of
the way, we now shift the elements $\mbox{\texttt{{\color{var}a}[{\color{var}i}]}},\ldots,\mbox{\texttt{{\color{var}a}[{\color{var}n}-1]}}$ right by one
position to make room for \mbox{\texttt{{\color{var}x}}}, set \mbox{\texttt{{\color{var}a}[{\color{var}i}]}} equal to \mbox{\texttt{{\color{var}x}}} and increment \mbox{\texttt{{\color{var}n}}}.

\begin{figure}
  \begin{center}
    \includegraphics{figs/arraystack}
  \end{center}
  \caption{A sequence of \mbox{\texttt{add({\color{var}i},{\color{var}x})}} and \mbox{\texttt{remove({\color{var}i})}} operation in an
  \mbox{\texttt{ArrayStack}}.  Arrows denote elements being copied.  Operations that
  result in a call to \mbox{\texttt{resize()}} are marked with an asterisk.}
  \figlabel{arraystack}
\end{figure}

\begin{Verbatim}[tabsize=2,frame=single,commandchars=\\@\$,label=\texttt{ArrayStack},labelposition=topline]
	@\color@var$void$ add(@\color@keyword$int$ @\color@var$i$, @\color@keyword$T$ @\color@var$x$) {
		@\color@keyword$if$ (@\color@var$i$ < 0 || @\color@var$i$ > @\color@var$n$) @\color@keyword$throw$ @\color@keyword$new$ IndexOutOfBoundsException();
		@\color@keyword$if$ (@\color@var$n$ + 1 > @\color@var$a$.@\color@var$length$) resize();
		@\color@keyword$for$ (@\color@keyword$int$ @\color@var$j$ = @\color@var$n$; @\color@var$j$ > @\color@var$i$; @\color@var$j$--) 
			@\color@var$a$[@\color@var$j$] = @\color@var$a$[@\color@var$j$-1];
		@\color@var$a$[@\color@var$i$] = @\color@var$x$;
		@\color@var$n$++;
	}
\end{Verbatim}
% TODO: Add shifting figure
If we ignore the cost of the potential call to \mbox{\texttt{resize()}}, the cost of the
\mbox{\texttt{add({\color{var}i},{\color{var}x})}} operation is proportional to the number of elements we have
to shift to make room for \mbox{\texttt{{\color{var}x}}}.  Therefore the cost of this operation
(ignoring the cost of resizing \mbox{\texttt{{\color{var}a}}}) is $O(\mbox{\texttt{{\color{var}n}}}-\mbox{\texttt{{\color{var}i}}}+1)$.

Implementing the \mbox{\texttt{remove({\color{var}i})}} operation is similar.  We shift the elements
$\mbox{\texttt{{\color{var}a}[{\color{var}i}+1]}},\ldots,\mbox{\texttt{{\color{var}a}[{\color{var}n}-1]}}$ left by one position (overwriting \mbox{\texttt{{\color{var}a}[{\color{var}i}]}}) and
decrease the value of \mbox{\texttt{{\color{var}n}}}.  After doing this, we check if \mbox{\texttt{{\color{var}n}}} is getting
much smaller than \mbox{\texttt{{\color{var}a}.{\color{var}length}}} by checking if $\mbox{\texttt{{\color{var}a}.{\color{var}length}}} \ge 3\mbox{\texttt{{\color{var}n}}}$. If so,
we call \mbox{\texttt{resize()}} to reduce the size of \mbox{\texttt{{\color{var}a}}}.

\begin{Verbatim}[tabsize=2,frame=single,commandchars=\\@\$,label=\texttt{ArrayStack},labelposition=topline]
	@\color@keyword$T$ remove(@\color@keyword$int$ @\color@var$i$) {
		@\color@keyword$if$ (@\color@var$i$ < 0 || @\color@var$i$ > @\color@var$n$ - 1) @\color@keyword$throw$ @\color@keyword$new$ IndexOutOfBoundsException();
		@\color@keyword$T$ @\color@var$x$ = @\color@var$a$[@\color@var$i$];
		@\color@keyword$for$ (@\color@keyword$int$ @\color@var$j$ = @\color@var$i$; @\color@var$j$ < @\color@var$n$-1; @\color@var$j$++) 
			@\color@var$a$[@\color@var$j$] = @\color@var$a$[@\color@var$j$+1];
		@\color@var$n$--;
		@\color@keyword$if$ (@\color@var$a$.@\color@var$length$ >= 3*@\color@var$n$) resize();
		@\color@var$return$ @\color@var$x$;
	}
\end{Verbatim}
% TODO: Add shifting figure
If we ignore the cost of the \mbox{\texttt{resize()}} method, the cost of a \mbox{\texttt{remove({\color{var}i})}}
operation is proportional to the number of the number elements we shift,
which is $O(\mbox{\texttt{{\color{var}n}}}-\mbox{\texttt{{\color{var}i}}})$. 

\subsection{Growing and Shrinking}

The \mbox{\texttt{resize()}} method is fairly straightforward; it allocates a new
array \mbox{\texttt{{\color{var}b}}} whose size is $2\mbox{\texttt{{\color{var}n}}}$ and copies the \mbox{\texttt{{\color{var}n}}} elements of \mbox{\texttt{{\color{var}a}}} into
the first \mbox{\texttt{{\color{var}n}}} positions in \mbox{\texttt{{\color{var}b}}}, and then sets \mbox{\texttt{{\color{var}a}}} to \mbox{\texttt{{\color{var}b}}}. Thus, after a call to \mbox{\texttt{resize()}}, $\mbox{\texttt{{\color{var}a}.{\color{var}length}}} = 2\mbox{\texttt{{\color{var}n}}}$.

\begin{Verbatim}[tabsize=2,frame=single,commandchars=\\@\$,label=\texttt{ArrayStack},labelposition=topline]
	@\color@var$void$ resize() {
		@\color@keyword$T$[] @\color@var$b$ = newArray(max(@\color@var$n$*2,1));
		@\color@keyword$for$ (@\color@keyword$int$ @\color@var$i$ = 0; @\color@var$i$ < @\color@var$n$; @\color@var$i$++) {
			@\color@var$b$[@\color@var$i$] = @\color@var$a$[@\color@var$i$];
		}
		@\color@var$a$ = @\color@var$b$;
	}
\end{Verbatim}

Analyzing the actual cost of the \mbox{\texttt{resize()}} operation is easy. It
allocates an array \mbox{\texttt{{\color{var}b}}} of size $2\mbox{\texttt{{\color{var}n}}}$ and copies the \mbox{\texttt{{\color{var}n}}} elements of \mbox{\texttt{{\color{var}a}}}
into \mbox{\texttt{{\color{var}b}}}.  This takes $O(\mbox{\texttt{{\color{var}n}}})$ time.

The running time analysis from the previous section ignored the cost
of calls to \mbox{\texttt{resize()}}.  In this section we analyze this cost using a
technique known as \emph{amortized analysis}.  This technique does not
try and determine the cost of resizing during each individual \mbox{\texttt{add({\color{var}i},{\color{var}x})}}
and \mbox{\texttt{remove({\color{var}i})}} operation.  Instead, it considers the cost of all calls to
\mbox{\texttt{resize()}} during a sequence of $m$ calls to \mbox{\texttt{add({\color{var}i},{\color{var}x})}} or \mbox{\texttt{remove({\color{var}x})}}.
In particular, we will show:

\begin{lem}\lemlabel{arraystack-amortized}
  If an empty \mbox{\texttt{ArrayList}} is created and any sequence of $m\ge 1$ calls
  to \mbox{\texttt{add({\color{var}i},{\color{var}x})}} and \mbox{\texttt{remove({\color{var}i})}} are performed, then the total cost of all work
  done during all calls to \mbox{\texttt{resize()}} is $O(m)$.
\end{lem}

\begin{proof}
  We will show that anytime \mbox{\texttt{resize()}} is called, the number of calls
  to \mbox{\texttt{{\color{var}add}}} or \mbox{\texttt{{\color{var}remove}}} since the last call to \mbox{\texttt{resize()}} is at least
  $\mbox{\texttt{{\color{var}n}}}/2-1$.  Therefore, if $n_i$ denotes the value of \mbox{\texttt{{\color{var}n}}} during the $i$th
call to \mbox{\texttt{resize()}}, then the total number of calls to \mbox{\texttt{add({\color{var}i},{\color{var}x})}} or
\mbox{\texttt{remove({\color{var}i})}} is at least
  \[
     \sum_{i=1}^{r} (n_i/2-1) \le m  \enspace .
  \]
  On the other hand, the total work done during all calls to \mbox{\texttt{resize()}} is 
  \[
     \sum_{i=1}^{r} O(n_i) \le O(m+r) = O(m)  \enspace ,
  \]
  which will prove the theorem since $r$, the number of calls to
\mbox{\texttt{resize()}},
  is not more than $m$.  All that remains is to show that the number
  of calls to \mbox{\texttt{add({\color{var}i},{\color{var}x})}} or \mbox{\texttt{remove({\color{var}i})}} between the $(i-1)$th and the
  $i$th call to \mbox{\texttt{resize()}} is at least $n_i/2$.

  There are two cases to consider. In the first case, \mbox{\texttt{resize()}} is
  being called by \mbox{\texttt{add({\color{var}i},{\color{var}x})}} because the backing array \mbox{\texttt{{\color{var}a}}} is full, i.e.,
  $\mbox{\texttt{{\color{var}a}.length()}} = \mbox{\texttt{{\color{var}n}}}=n_i$.  Consider the previous call to \mbox{\texttt{resize()}}:
  After this previous call, the size of \mbox{\texttt{{\color{var}a}}} was \mbox{\texttt{{\color{var}a}.{\color{var}length}}}, but the
  number of elements stored in \mbox{\texttt{{\color{var}a}}} was at most $\mbox{\texttt{{\color{var}a}.{\color{var}length}}}/2=n_i/2$.
  But now the number of elements stored in \mbox{\texttt{{\color{var}a}}} is $n_i=\mbox{\texttt{{\color{var}a}.{\color{var}length}}}$,
  so there must have been at least $n_i/2$ calls to \mbox{\texttt{add({\color{var}i},{\color{var}x})}} since
  the previous call to \mbox{\texttt{resize()}}.
  % TODO: Add figure
  
  The second case to consider is when \mbox{\texttt{resize()}} is being called by
  \mbox{\texttt{remove({\color{var}i})}} because $\mbox{\texttt{{\color{var}a}.{\color{var}length}}} \ge 3\mbox{\texttt{{\color{var}n}}}=3n_i$.  Again, after the
  previous call to \mbox{\texttt{resize()}} the number of elements stored in \mbox{\texttt{{\color{var}a}}} was
  at least $\mbox{\texttt{{\color{var}a}.{\color{var}length}/2}}-1$. (The ${}-1$ in this formula accounts for
  the special case that occurs when $\mbox{\texttt{{\color{var}n}}}=0$ and \mbox{\texttt{{\color{var}a}.{\color{var}length}=1}}.) Now there
  are $n_i\le\mbox{\texttt{{\color{var}a}.{\color{var}length}}}/3$ elements stored in \mbox{\texttt{{\color{var}a}}}.  Therefore, the number
  of \mbox{\texttt{remove({\color{var}i})}} operations since the last call to \mbox{\texttt{resize()}} is at least
  \[
      \mbox{\texttt{{\color{var}a}.{\color{var}length}}}/2 - 1 - \mbox{\texttt{{\color{var}a}.{\color{var}length}}}/3 = \mbox{\texttt{{\color{var}a}.{\color{var}length}}}/6 - 1
         = (\mbox{\texttt{{\color{var}a}.{\color{var}length}}}/3)/2 - 1\ge n_i/2 -1\enspace .
  \]
  In either case, the number of calls to \mbox{\texttt{add({\color{var}i},{\color{var}x})}} or \mbox{\texttt{remove({\color{var}i})}} that
  occur between the $(i-1)$th call to \mbox{\texttt{resize()}} and the $i$th call to
  resize is at least $n_i/2-1$, as required to complete the proof.
\end{proof}

\subsection{Summary}

The following theorem summarizes the performance of an \mbox{\texttt{ArrayStack}}:

\begin{thm}\thmlabel{arraystack}
  An \mbox{\texttt{ArrayStack}} implements the \mbox{\texttt{List}} interface.  Ignoring the cost of
  calls to \mbox{\texttt{resize()}}, an \mbox{\texttt{ArrayStack}} supports the operations
  \begin{itemize}
    \item \mbox{\texttt{get({\color{var}i})}} and \mbox{\texttt{set({\color{var}i},{\color{var}x})}} in $O(1)$ time per operation; and
    \item \mbox{\texttt{add({\color{var}i},{\color{var}x})}} and \mbox{\texttt{remove({\color{var}i})}} in $O(1+\mbox{\texttt{{\color{var}n}}}-\mbox{\texttt{{\color{var}i}}})$ time per operation.
  \end{itemize}
  Furthermore, beginning with an empty \mbox{\texttt{ArrayStack}}, any sequence of $m$
  \mbox{\texttt{add({\color{var}i},{\color{var}x})}} and \mbox{\texttt{remove({\color{var}i})}} operations results in a total of $O(m)$
  work during all calls to \mbox{\texttt{resize()}}.
\end{thm}

The \mbox{\texttt{ArrayStack}} is an efficient way to implement a \mbox{\texttt{Stack}}.
In particular, we can implement \mbox{\texttt{push({\color{var}x})}} as \mbox{\texttt{add({\color{var}n},{\color{var}x})}} and \mbox{\texttt{pop()}}
as \mbox{\texttt{remove({\color{var}n}-1)}}, in which case these operations will run in $O(1)$
amortized time.

\section{\mbox{\texttt{FastArrayStack}}: An Optimized ArrayStack}
\seclabel{fastarraystack}
Much of the work done by an \mbox{\texttt{ArrayStack}} involves shifting (by
\mbox{\texttt{add({\color{var}i},{\color{var}x})}} and \mbox{\texttt{remove({\color{var}i})}}) and copying (by \mbox{\texttt{resize()}}) of data.  In the
implementations shown above, this was done using \mbox{\texttt{{\color{keyword}for}}} loops. It turns
out that many programming environments have specific functions that are
very efficient at copying and moving blocks of data.  In the C and C++
programming languages there is the \mbox{\texttt{memcpy({\color{var}d},{\color{var}s},{\color{var}n})}} function.  In Java
there is the \mbox{\texttt{System.arraycopy({\color{var}s},{\color{var}i},{\color{var}d},{\color{var}j},{\color{var}n})}} method.

\begin{Verbatim}[tabsize=2,frame=single,commandchars=\\@\$,label=\texttt{FastArrayStack},labelposition=topline]
	@\color@var$void$ resize() {
		@\color@keyword$T$[] @\color@var$b$ = newArray(max(2*@\color@var$n$,1));
		System.arraycopy(@\color@var$a$, 0, @\color@var$b$, 0, @\color@var$n$);
		@\color@var$a$ = @\color@var$b$;
	}
	@\color@var$void$ add(@\color@keyword$int$ @\color@var$i$, @\color@keyword$T$ @\color@var$x$) {
		@\color@keyword$if$ (@\color@var$i$ < 0 || @\color@var$i$ > @\color@var$n$) @\color@keyword$throw$ @\color@keyword$new$ IndexOutOfBoundsException();
		@\color@keyword$if$ (@\color@var$n$ + 1 > @\color@var$a$.@\color@var$length$) resize();
		System.arraycopy(@\color@var$a$, @\color@var$i$, @\color@var$a$, @\color@var$i$+1, @\color@var$n$-@\color@var$i$); 
		@\color@var$a$[@\color@var$i$] = @\color@var$x$;
		@\color@var$n$++;
	}
	@\color@keyword$T$ remove(@\color@keyword$int$ @\color@var$i$) {
		@\color@keyword$if$ (@\color@var$i$ < 0 || @\color@var$i$ > @\color@var$n$ - 1) @\color@keyword$throw$ @\color@keyword$new$ IndexOutOfBoundsException();
		@\color@keyword$T$ @\color@var$x$ = @\color@var$a$[@\color@var$i$];
		System.arraycopy(@\color@var$a$, @\color@var$i$+1, @\color@var$a$, @\color@var$i$, @\color@var$n$-@\color@var$i$-1);
		@\color@var$n$--; 
		@\color@keyword$if$ (@\color@var$a$.@\color@var$length$ >= 3*@\color@var$n$) resize();
		@\color@var$return$ @\color@var$x$;
	}
\end{Verbatim}

These functions are usually highly optimized and may even use special
machine instructions that can do this copying much faster than we could do
with \mbox{\texttt{{\color{keyword}for}}} loop.  Although using these functions does not asymptotically
decrease the running times, it can still be a worthwhile optimization.
In the Java implementations here, the use of \mbox{\texttt{System.arraycopy({\color{var}s},{\color{var}i},{\color{var}d},{\color{var}j},{\color{var}n})}}
resulted in speedups of a factor of 2-3 depending on the types of
operations performed.

\section{\mbox{\texttt{ArrayQueue}}: An Array-Based Queue}
\seclabel{arrayqueue}

In this section, we present the \mbox{\texttt{ArrayQueue}} data structure, which
implements a FIFO (first-in-first-out) queue; elements are removed (using
the \mbox{\texttt{remove()}} operation) from the queue in the same order they are added
(using the \mbox{\texttt{add({\color{var}x})}} operation).

Notice that an \mbox{\texttt{ArrayStack}} is a poor choice for an implementation of a
FIFO queue.  The reason is that we must choose one end of the list to
add to and then remove from the other end.  One of the two operations
must work on the head of the list, which involves calling \mbox{\texttt{add({\color{var}i},{\color{var}x})}}
or \mbox{\texttt{remove({\color{var}i})}} with a value of $\mbox{\texttt{{\color{var}i}}}=0$.  This gives a running time
of $\Theta(n)$.

To obtain an efficient array-based implementation of a queue, we
first notice that the problem would be easy if we had an infinite
array \mbox{\texttt{{\color{var}a}}}.  We could maintain one index \mbox{\texttt{{\color{var}j}}} that keeps track of the
next element to remove and an integer \mbox{\texttt{{\color{var}n}}} that counts the number of
elements in the queue.  The queue elements would always be stored in
\[ \mbox{\texttt{{\color{var}a}[{\color{var}j}]}},\mbox{\texttt{{\color{var}a}[{\color{var}j}+1]}},\ldots,\mbox{\texttt{{\color{var}a}[{\color{var}j}+{\color{var}n}-1]}} \enspace . \]
Initially, both \mbox{\texttt{{\color{var}i}}} and \mbox{\texttt{{\color{var}n}}} would be 
set to 0.  To add an element, we would place it in \mbox{\texttt{{\color{var}a}[{\color{var}j}+{\color{var}n}]}} and increment \mbox{\texttt{{\color{var}n}}}.
To remove an element, we would remove it from \mbox{\texttt{{\color{var}a}[{\color{var}j}]}}, increment \mbox{\texttt{{\color{var}j}}}, and
decrement \mbox{\texttt{{\color{var}n}}}.

Of course, the problem with this solution is that it requires an infinite
array \mbox{\texttt{{\color{var}a}}}.  An \mbox{\texttt{ArrayQueue}} simulates this by using a finite array \mbox{\texttt{{\color{var}a}}}
and \emph{modular arithmetic}.  This is the kind of arithmetic used when
we are talking about time of day.  For example 10 o'clock plus 5
hours gives 3 o'clock.  Formally, we say that
\[
    10 + 5 = 15 \equiv 3 \pmod{12} \enspace .
\]
We read the latter part of this equation as ``15 is congruent to 3 modulo
12.'' We can also treat $\bmod$ as binary operator, so that
\[
   15 \bmod 12 = 3 \enspace .
\]

More generally, for an integer $a$ and positive integer $m$, $a \bmod m$
is the unique integer $r\in\{0,\ldots,m-1\}$ such that $a = r + km$ for
some integer $k$.  Less formally, the value $r$ is the remainder we get
when we divide $a$ by $m$.  In many programming languages, including Java,
the $\bmod$ operator is represented using the \mbox{\texttt{\%}} symbol.\footnote{This
is sometimes referred to as the \emph{brain-dead} mod operator since
it does not correctly implement the mathematical mod operator when the
first argument is negative.}

Modular arithmetic is useful for simulating an infinite array,
since $\mbox{\texttt{{\color{var}i}}}\bmod \mbox{\texttt{{\color{var}a}.{\color{var}length}}}$ always gives a value in the range
$0,\ldots,\mbox{\texttt{{\color{var}a}.{\color{var}length}-1}}$.  Using modular arithmetic we can store the
queue elements at array locations
\[ \mbox{\texttt{{\color{var}a}[{\color{var}j}\%{\color{var}a}.{\color{var}length}]}},\mbox{\texttt{{\color{var}a}[({\color{var}j}+1)\%{\color{var}a}.{\color{var}length}]}},\ldots,\mbox{\texttt{{\color{var}a}[({\color{var}j}+{\color{var}n}-1)\%{\color{var}a}.{\color{var}length}]}}
\enspace. \]
This treats \mbox{\texttt{{\color{var}a}}} like a \emph{circular array} in which array indices
exceeding $\mbox{\texttt{{\color{var}a}.{\color{var}length}}}-1$ ``wrap around'' to the beginning of
the array.
% TODO: figure

The only remaining thing to worry about is taking care that the number
of elements in the \mbox{\texttt{ArrayQueue}} does not exceed the size of \mbox{\texttt{{\color{var}a}}}.

\begin{Verbatim}[tabsize=2,frame=single,commandchars=\\@\$,label=\texttt{ArrayQueue},labelposition=topline]
	@\color@keyword$T$[] @\color@var$a$;
	@\color@keyword$int$ @\color@var$j$;
	@\color@keyword$int$ @\color@var$n$;
\end{Verbatim}

A sequence of \mbox{\texttt{add({\color{var}x})}} and \mbox{\texttt{remove()}} operation on an \mbox{\texttt{ArrayQueue}} are illusted in \figref{arrayqueue}.
To implement \mbox{\texttt{add({\color{var}x})}}, we first check if \mbox{\texttt{{\color{var}a}}} is full and, if necessary,
call \mbox{\texttt{resize()}} to increase the size of \mbox{\texttt{{\color{var}a}}}.  Next, we store \mbox{\texttt{{\color{var}x}}} in
\mbox{\texttt{{\color{var}a}[({\color{var}j}+{\color{var}n})\%{\color{var}a}.{\color{var}length}]}} and increment \mbox{\texttt{{\color{var}n}}}.

\begin{figure}
  \begin{center}
    \includegraphics{figs/arrayqueue}
  \end{center}
  \caption{A sequence of \mbox{\texttt{add({\color{var}x})}} and \mbox{\texttt{remove({\color{var}i})}} operation on an
  \mbox{\texttt{ArrayQueue}}.  Arrows denote elements being copied.  Operations that
  result in a call to \mbox{\texttt{resize()}} are marked with an asterisk.}
  \figlabel{arrayqueue}
\end{figure}



\begin{Verbatim}[tabsize=2,frame=single,commandchars=\\@\$,label=\texttt{ArrayQueue},labelposition=topline]
	@\color@var$boolean$ add(@\color@keyword$T$ @\color@var$x$) {
		@\color@keyword$if$ (@\color@var$n$ + 1 > @\color@var$a$.@\color@var$length$) resize();
		@\color@var$a$[(@\color@var$j$+@\color@var$n$) % @\color@var$a$.@\color@var$length$] = @\color@var$x$;
		@\color@var$n$++;
		@\color@var$return$ @\color@var$true$;
	}
\end{Verbatim}

To implement \mbox{\texttt{remove()}} we first store \mbox{\texttt{{\color{var}a}[{\color{var}j}]}} so that we can return
it later.  Next, we decrement \mbox{\texttt{{\color{var}n}}} and ``increment'' \mbox{\texttt{{\color{var}j}}} by setting
$\mbox{\texttt{{\color{var}j}}}=(\mbox{\texttt{{\color{var}j}}}+1)\bmod \mbox{\texttt{{\color{var}a}.{\color{var}length}}}$ and return the stored value of \mbox{\texttt{{\color{var}a}[{\color{var}j}]}}. If
necessary, we may call \mbox{\texttt{resize()}} to decrease the size of \mbox{\texttt{{\color{var}a}}}.

\begin{Verbatim}[tabsize=2,frame=single,commandchars=\\@\$,label=\texttt{ArrayQueue},labelposition=topline]
	@\color@keyword$T$ remove() { 
		@\color@keyword$if$ (@\color@var$n$ == 0) @\color@keyword$throw$ @\color@keyword$new$ NoSuchElementException();
		@\color@keyword$T$ @\color@var$x$ = @\color@var$a$[@\color@var$j$];
		@\color@var$j$ = (@\color@var$j$ + 1) % @\color@var$a$.@\color@var$length$;
		@\color@var$n$--;
		@\color@keyword$if$ (@\color@var$a$.@\color@var$length$ >= 3*@\color@var$n$) resize();
		@\color@var$return$ @\color@var$x$;
	}
\end{Verbatim}

Finally, the \mbox{\texttt{resize()}} operation is very similar to the \mbox{\texttt{resize()}}
operation of \mbox{\texttt{ArrayQueue}}.  It allocates a new array \mbox{\texttt{{\color{var}b}}} of size $2\mbox{\texttt{{\color{var}n}}}$
and copies
\[
   \mbox{\texttt{{\color{var}a}[{\color{var}j}]}},\mbox{\texttt{{\color{var}a}[({\color{var}j}+1)\%{\color{var}a}.{\color{var}length}]}},\ldots,\mbox{\texttt{{\color{var}a}[({\color{var}j}+{\color{var}n}-1)\%{\color{var}a}.{\color{var}length}]}}
\]
onto
\[
   \mbox{\texttt{{\color{var}b}[0]}},\mbox{\texttt{{\color{var}b}[1]}},\ldots,\mbox{\texttt{{\color{var}b}[{\color{var}n}-1]}}
\]
and sets $\mbox{\texttt{{\color{var}j}}}=0$.

\begin{Verbatim}[tabsize=2,frame=single,commandchars=\\@\$,label=\texttt{ArrayQueue},labelposition=topline]
	@\color@var$void$ resize() {
		@\color@keyword$T$[] @\color@var$b$ = newArray(max(1,@\color@var$n$*2));
		@\color@keyword$for$ (@\color@keyword$int$ @\color@var$k$ = 0; @\color@var$k$ < @\color@var$n$; @\color@var$k$++) 
			@\color@var$b$[@\color@var$k$] = @\color@var$a$[(@\color@var$j$+@\color@var$k$) % @\color@var$a$.@\color@var$length$];
		@\color@var$a$ = @\color@var$b$;
		@\color@var$j$ = 0;
	}
\end{Verbatim}

\subsection{Summary}

The following theorem summarizes the performance of the \mbox{\texttt{ArrayQueue}}
data structure:

\begin{thm}
An \mbox{\texttt{ArrayQueue}} implements the (FIFO) \mbox{\texttt{Queue}} interface.  Ignoring the cost of
calls to \mbox{\texttt{resize()}}, an \mbox{\texttt{ArrayQueue}} supports the operations
\mbox{\texttt{add({\color{var}x})}} and \mbox{\texttt{remove()}} in $O(1)$ time per operation.
Furthermore, beginning with an empty \mbox{\texttt{ArrayQueue}}, any sequence of $m$
\mbox{\texttt{add({\color{var}i},{\color{var}x})}} and \mbox{\texttt{remove({\color{var}i})}} operations results in a total of $O(m)$
work during all calls to \mbox{\texttt{resize()}}.
\end{thm}

%TODO: Discuss the use of bitwise-and as a replacement for the mod operator

\section{\mbox{\texttt{ArrayDeque}}: Fast Deque Operations Using an Array}
\seclabel{arraydeque}

The \mbox{\texttt{ArrayQueue}} from the previous section gives data structure for
representing a sequence that allows us to efficiently add to one end
of the sequence and remove from the other end.  The \mbox{\texttt{ArrayDeque}} data
structure uses the same technique, but allows for efficient addition or
removal at both ends.  This structure implements the \mbox{\texttt{List}} interface
using the same circular array technique used to represent an \mbox{\texttt{ArrayQueue}}.

\begin{Verbatim}[tabsize=2,frame=single,commandchars=\\@\$,label=\texttt{ArrayDeque},labelposition=topline]
	@\color@keyword$T$[] @\color@var$a$;
	@\color@keyword$int$ @\color@var$j$;
	@\color@keyword$int$ @\color@var$n$;
\end{Verbatim}

The \mbox{\texttt{get({\color{var}i})}} and \mbox{\texttt{set({\color{var}i},{\color{var}x})}} operations in an \mbox{\texttt{ArrayDeque}} are
straightforward.  They get or set the array element $\mbox{\texttt{{\color{var}a}[}}{\mbox{\texttt{({\color{var}j}+{\color{var}i})}}\bmod
\mbox{\texttt{{\color{var}a}.{\color{var}length}}}}\mbox{\texttt{]}}$.

\begin{Verbatim}[tabsize=2,frame=single,commandchars=\\@\$,label=\texttt{ArrayDeque},labelposition=topline]
	@\color@keyword$T$ get(@\color@keyword$int$ @\color@var$i$) {
		@\color@keyword$if$ (@\color@var$i$ < 0 || @\color@var$i$ > @\color@var$n$-1) @\color@keyword$throw$ @\color@keyword$new$ IndexOutOfBoundsException();
		@\color@var$return$ @\color@var$a$[(@\color@var$j$+@\color@var$i$)%@\color@var$a$.@\color@var$length$];
	}
	@\color@keyword$T$ set(@\color@keyword$int$ @\color@var$i$, @\color@keyword$T$ @\color@var$x$) {
		@\color@keyword$if$ (@\color@var$i$ < 0 || @\color@var$i$ > @\color@var$n$-1) @\color@keyword$throw$ @\color@keyword$new$ IndexOutOfBoundsException();
		@\color@keyword$T$ @\color@var$y$ = @\color@var$a$[(@\color@var$j$+@\color@var$i$)%@\color@var$a$.@\color@var$length$];
		@\color@var$a$[(@\color@var$j$+@\color@var$i$)%@\color@var$a$.@\color@var$length$] = @\color@var$x$;
		@\color@var$return$ @\color@var$y$;
	}
\end{Verbatim}

Adding and removing elements from an \mbox{\texttt{ArrayDeque}} is illustrated in
\figref{arraydeque}.  The implementation of \mbox{\texttt{add({\color{var}i},{\color{var}x})}} is a little
more interesting.   As usual, we first check if \mbox{\texttt{{\color{var}a}}} is full and,
if necessary, call \mbox{\texttt{resize()}} to resize \mbox{\texttt{{\color{var}a}}}.   Remember that we want
this operation to be fast when \mbox{\texttt{{\color{var}i}}} is small (close to 0) or when \mbox{\texttt{{\color{var}i}}}
is large (close to \mbox{\texttt{{\color{var}n}}}).  Therefore, we check if $\mbox{\texttt{{\color{var}i}}}<\mbox{\texttt{{\color{var}n}}}/2$.  If so,
we shift the elements $\mbox{\texttt{{\color{var}a}[0]}},\ldots,\mbox{\texttt{{\color{var}a}[{\color{var}i}-1]}}$ left by one position.
Otherwise ($\mbox{\texttt{{\color{var}i}}}\ge\mbox{\texttt{{\color{var}n}}}/2$), we shift the elements $\mbox{\texttt{{\color{var}a}[{\color{var}i}]}},\ldots,\mbox{\texttt{{\color{var}a}[{\color{var}n}-1]}}$
right by one position.

\begin{figure}
  \begin{center}
    \includegraphics{figs/arraydeque}
  \end{center}
  \caption{A sequence of \mbox{\texttt{add({\color{var}i},{\color{var}x})}} and \mbox{\texttt{remove({\color{var}i})}} operation in an
  \mbox{\texttt{ArrayDeque}}.  Arrows denote elements being copied.}
  \figlabel{arraydeque}
\end{figure}


\begin{Verbatim}[tabsize=2,frame=single,commandchars=\\@\$,label=\texttt{ArrayDeque},labelposition=topline]
	@\color@var$void$ add(@\color@keyword$int$ @\color@var$i$, @\color@keyword$T$ @\color@var$x$) {
		@\color@keyword$if$ (@\color@var$i$ < 0 || @\color@var$i$ > @\color@var$n$) @\color@keyword$throw$ @\color@keyword$new$ IndexOutOfBoundsException();
		@\color@keyword$if$ (@\color@var$n$+1 > @\color@var$a$.@\color@var$length$) resize();
		@\color@keyword$if$ (@\color@var$i$ < @\color@var$n$/2) {	@\color@comment$// shift a[0],..,a[i-1] left one position$
			@\color@var$j$ = (@\color@var$j$ == 0) ? @\color@var$a$.@\color@var$length$ - 1 : @\color@var$j$ - 1;
			@\color@keyword$for$ (@\color@keyword$int$ @\color@var$k$ = 0; @\color@var$k$ <= @\color@var$i$-1; @\color@var$k$++)
				@\color@var$a$[(@\color@var$j$+@\color@var$k$)%@\color@var$a$.@\color@var$length$] = @\color@var$a$[(@\color@var$j$+@\color@var$k$+1)%@\color@var$a$.@\color@var$length$];
		} @\color@keyword$else$ {	    @\color@comment$// shift a[i],..,a[n-1] right one position$
			@\color@keyword$for$ (@\color@keyword$int$ @\color@var$k$ = @\color@var$n$; @\color@var$k$ > @\color@var$i$; @\color@var$k$--)
				@\color@var$a$[(@\color@var$j$+@\color@var$k$)%@\color@var$a$.@\color@var$length$] = @\color@var$a$[(@\color@var$j$+@\color@var$k$-1)%@\color@var$a$.@\color@var$length$];
		}
		@\color@var$a$[(@\color@var$j$+@\color@var$i$)%@\color@var$a$.@\color@var$length$] = @\color@var$x$;
		@\color@var$n$++;
	}
\end{Verbatim}

By doing the shifting in this way, we guarantee that \mbox{\texttt{add({\color{var}i},{\color{var}x})}} never
has to shift more than $\min\{ \mbox{\texttt{{\color{var}i}}}, \mbox{\texttt{{\color{var}n}}}-\mbox{\texttt{{\color{var}i}}} \}$ elements.  Thus, the running
time of the \mbox{\texttt{add({\color{var}i},{\color{var}x})}} operation (ignoring the cost of the \mbox{\texttt{resize()}}
operation) is $O(1+\min\{\mbox{\texttt{{\color{var}i}}},\mbox{\texttt{{\color{var}n}}}-\mbox{\texttt{{\color{var}i}}}\})$.

The \mbox{\texttt{remove({\color{var}i})}} operation is similar.  It either shifts elements
$\mbox{\texttt{{\color{var}a}[0]}},\ldots,\mbox{\texttt{{\color{var}a}[{\color{var}i}-1]}}$ right by one position or shifts the elements
$\mbox{\texttt{{\color{var}a}[{\color{var}i}+1]}},\ldots,\mbox{\texttt{{\color{var}a}[{\color{var}n}-1]}}$ left by one position depending on whether
$\mbox{\texttt{{\color{var}i}}}<\mbox{\texttt{{\color{var}n}}}/2$.  Again, this means that \mbox{\texttt{remove({\color{var}i})}} never does more than 
$O(1+\min\{\mbox{\texttt{{\color{var}i}}},\mbox{\texttt{{\color{var}n}}}-\mbox{\texttt{{\color{var}i}}}\})$ work to shift elements.

\begin{Verbatim}[tabsize=2,frame=single,commandchars=\\@\$,label=\texttt{ArrayDeque},labelposition=topline]
	@\color@keyword$T$ remove(@\color@keyword$int$ @\color@var$i$) {
		@\color@keyword$if$ (@\color@var$i$ < 0 || @\color@var$i$ > @\color@var$n$ - 1)	@\color@keyword$throw$ @\color@keyword$new$ IndexOutOfBoundsException();
		@\color@keyword$T$ @\color@var$x$ = @\color@var$a$[(@\color@var$j$+@\color@var$i$)%@\color@var$a$.@\color@var$length$];
		@\color@keyword$if$ (@\color@var$i$ < @\color@var$n$/2) {  @\color@comment$// shift a[0],..,[i-1] right one position$
			@\color@keyword$for$ (@\color@keyword$int$ @\color@var$k$ = @\color@var$i$; @\color@var$k$ > 0; @\color@var$k$--)
				@\color@var$a$[(@\color@var$j$+@\color@var$k$)%@\color@var$a$.@\color@var$length$] = @\color@var$a$[(@\color@var$j$+@\color@var$k$-1)%@\color@var$a$.@\color@var$length$];
			@\color@var$j$ = (@\color@var$j$ + 1) % @\color@var$a$.@\color@var$length$;
		} @\color@keyword$else$ {        @\color@comment$// shift a[i+1],..,a[n-1] left one position$
			@\color@keyword$for$ (@\color@keyword$int$ @\color@var$k$ = @\color@var$i$; @\color@var$k$ < @\color@var$n$-1; @\color@var$k$++)
				@\color@var$a$[(@\color@var$j$+@\color@var$k$)%@\color@var$a$.@\color@var$length$] = @\color@var$a$[(@\color@var$j$+@\color@var$k$+1)%@\color@var$a$.@\color@var$length$];
		}
		@\color@var$n$--;
		@\color@keyword$if$ (3*@\color@var$n$ < @\color@var$a$.@\color@var$length$) resize();
		@\color@var$return$ @\color@var$x$;
	}
\end{Verbatim}

\subsection{Summary}

The following theorem summarizes the performance of the \mbox{\texttt{ArrayDeque}}
data structure:
\begin{thm}\thmlabel{arraydeque}
  An \mbox{\texttt{ArrayDeque}} implements the \mbox{\texttt{List}} interface.  Ignoring the cost of
  calls to \mbox{\texttt{resize()}}, an \mbox{\texttt{ArrayDeque}} supports the operations
  \begin{itemize}
    \item \mbox{\texttt{get({\color{var}i})}} and \mbox{\texttt{set({\color{var}i},{\color{var}x})}} in $O(1)$ time per operation; and
    \item \mbox{\texttt{add({\color{var}i},{\color{var}x})}} and \mbox{\texttt{remove({\color{var}i})}} in $O(1+\min\{\mbox{\texttt{{\color{var}i}}},\mbox{\texttt{{\color{var}n}}}-\mbox{\texttt{{\color{var}i}}}\})$ time
          per operation.
  \end{itemize}
  Furthermore, beginning with an empty \mbox{\texttt{ArrayDeque}}, any sequence of $m$
  \mbox{\texttt{add({\color{var}i},{\color{var}x})}} and \mbox{\texttt{remove({\color{var}i})}} operations results in a total of $O(m)$
  work during all calls to \mbox{\texttt{resize()}}.
\end{thm}

\section{\mbox{\texttt{DualArrayDeque}}: Buiding a Deque from Two Stacks}
\seclabel{dualarraydeque}

Next, we present another data structure, the \mbox{\texttt{DualArrayDeque}} that
achieves the same performance bounds as an \mbox{\texttt{ArrayDeque}} by using
two \mbox{\texttt{ArrayStack}}s.  Although the asymptotic performance of the
\mbox{\texttt{DualArrayDeque}} is no better than that of the \mbox{\texttt{ArrayDeque}}, it is
still worth studying since it offer a good example of how to make a
sophisticated data structure by combining two simpler data structures.

A \mbox{\texttt{DualArrayDeque}} represents a list using two \mbox{\texttt{ArrayStack}}s.  Recall that
an \mbox{\texttt{ArrayStack}} is fast when the operations on it modify elements near
the end.  A \mbox{\texttt{DualArrayDeque}} places two \mbox{\texttt{ArrayStack}}s, called \mbox{\texttt{{\color{var}front}}}
and \mbox{\texttt{{\color{var}back}}}, back-to-back so that operations are fast at either end.

\begin{Verbatim}[tabsize=2,frame=single,commandchars=\\@\$,label=\texttt{DualArrayDeque},labelposition=topline]
	List<@\color@keyword$T$> @\color@var$front$;
	List<@\color@keyword$T$> @\color@var$back$;
\end{Verbatim}

A \mbox{\texttt{DualArrayDeque}} does not explicitly store the number, \mbox{\texttt{{\color{var}n}}},
of elements it contains.  It doesn't need to, since it contains
$\mbox{\texttt{{\color{var}n}}}=\mbox{\texttt{{\color{var}front}.size()}} + \mbox{\texttt{{\color{var}back}.size()}}$ elements.  Nevertheless, when
analyzing the \mbox{\texttt{DualArrayDeque}} we will still use \mbox{\texttt{{\color{var}n}}} to denote the number
of elements it contains.

\begin{Verbatim}[tabsize=2,frame=single,commandchars=\\@\$,label=\texttt{DualArrayDeque},labelposition=topline]
	@\color@keyword$int$ size() {
		@\color@var$return$ @\color@var$front$.size() + @\color@var$back$.size();		
	}
\end{Verbatim}

The \mbox{\texttt{{\color{var}front}}} \mbox{\texttt{ArrayStack}} contains list elements with indices
$0,\ldots,\mbox{\texttt{{\color{var}front}.size()}}-1$, but stores them in reverse order.
The \mbox{\texttt{{\color{var}back}}} \mbox{\texttt{ArrayStack}} contains list elements with indices
$\mbox{\texttt{{\color{var}front}.size()}},\ldots,\mbox{\texttt{size()}}-1$ in the normal order.  In this way,
\mbox{\texttt{get({\color{var}i})}} and \mbox{\texttt{set({\color{var}i},{\color{var}x})}} translate into appropriate calls to \mbox{\texttt{get({\color{var}i})}}
or \mbox{\texttt{set({\color{var}i},{\color{var}x})}} on either \mbox{\texttt{{\color{var}front}}} or \mbox{\texttt{{\color{var}back}}}, which take $O(1)$ time per operation.

\begin{Verbatim}[tabsize=2,frame=single,commandchars=\\@\$,label=\texttt{DualArrayDeque},labelposition=topline]
	@\color@keyword$T$ get(@\color@keyword$int$ @\color@var$i$) {
		@\color@keyword$if$ (@\color@var$i$ < @\color@var$front$.size()) {
			@\color@var$return$ @\color@var$front$.get(@\color@var$front$.size()-@\color@var$i$-1);
		} @\color@keyword$else$ {
			@\color@var$return$ @\color@var$back$.get(@\color@var$i$-@\color@var$front$.size());
		}
	}
	@\color@keyword$T$ set(@\color@keyword$int$ @\color@var$i$, @\color@keyword$T$ @\color@var$x$) {
		@\color@keyword$if$ (@\color@var$i$ < @\color@var$front$.size()) {
			@\color@var$return$ @\color@var$front$.set(@\color@var$front$.size()-@\color@var$i$-1, @\color@var$x$);
			
		} @\color@keyword$else$ {
			@\color@var$return$ @\color@var$back$.set(@\color@var$i$-@\color@var$front$.size(), @\color@var$x$);
		}
	}
\end{Verbatim}

Note that, if a $\mbox{\texttt{{\color{var}i}}}<\mbox{\texttt{{\color{var}front}.size()}}$, then this translates to accessing
the element of \mbox{\texttt{{\color{var}front}}} at position $\mbox{\texttt{{\color{var}front}.size()}}-\mbox{\texttt{{\color{var}i}}}-1$, since the
elements of \mbox{\texttt{{\color{var}front}}} are stored in reverse order.

Adding and removing elements from an \mbox{\texttt{ArrayDeque}} is illustrated in
\figref{arraydeque}.  The \mbox{\texttt{add({\color{var}i},{\color{var}x})}} operation manipulates either \mbox{\texttt{{\color{var}front}}}
or \mbox{\texttt{{\color{var}back}}}, as appropriate:

\begin{figure}
  \begin{center}
    \includegraphics{figs/dualarraydeque}
  \end{center}
  \caption{A sequence of \mbox{\texttt{add({\color{var}i},{\color{var}x})}} and \mbox{\texttt{remove({\color{var}i})}} operation in a
  \mbox{\texttt{DualArrayDeque}}.  Arrows denote elements being copied.
  Operations that result in a rebalancing by \mbox{\texttt{rebalance()}} are marked with an asterisk.}
  \figlabel{dualarraydeque}
\end{figure}



\begin{Verbatim}[tabsize=2,frame=single,commandchars=\\@\$,label=\texttt{DualArrayDeque},labelposition=topline]
	@\color@var$void$ add(@\color@keyword$int$ @\color@var$i$, @\color@keyword$T$ @\color@var$x$) {
		@\color@keyword$if$ (@\color@var$i$ < @\color@var$front$.size()) { 
			@\color@var$front$.add(@\color@var$front$.size()-@\color@var$i$, @\color@var$x$);
		} @\color@keyword$else$ {
			@\color@var$back$.add(@\color@var$i$-@\color@var$front$.size(), @\color@var$x$);
		}
		balance();
	}
\end{Verbatim}

The \mbox{\texttt{add({\color{var}i},{\color{var}x})}} method performs rebalancing of the two \mbox{\texttt{ArrayStacks}}
\mbox{\texttt{{\color{var}front}}} and \mbox{\texttt{{\color{var}back}}}, by calling the \mbox{\texttt{balance()}} method.  The implementation
of \mbox{\texttt{balance()}} is described below, but for now it is sufficient to know
that \mbox{\texttt{balance()}} ensures that, unless $\mbox{\texttt{size()}}<2$, \mbox{\texttt{{\color{var}front}.size()}}
and \mbox{\texttt{{\color{var}back}.size()}} do not differ by more than a factor of 3.  In
particular, $3\mbox{\texttt{{\color{var}front}.size()}} \le \mbox{\texttt{{\color{var}back}.size()}}$ and $3\mbox{\texttt{{\color{var}back}.size()}}\le
\mbox{\texttt{{\color{var}front}.size()}}$.

Next we analyze the cost of \mbox{\texttt{add({\color{var}i},{\color{var}x})}}, ignoring the cost of the
\mbox{\texttt{balance()}} operation.  If $\mbox{\texttt{{\color{var}i}}}<\mbox{\texttt{{\color{var}front}.size()}}$, then this translates into $\mbox{\texttt{{\color{var}front}.add({\color{var}front}.size()-{\color{var}i}-1,{\color{var}x})}}$.  Since \mbox{\texttt{{\color{var}front}}} is an \mbox{\texttt{ArrayStack}}, the cost of this is 
\begin{equation}
  O(\mbox{\texttt{{\color{var}front}.size()}}-(\mbox{\texttt{{\color{var}front}.size()}}-\mbox{\texttt{{\color{var}i}}}-1)+1) = O(\mbox{\texttt{{\color{var}i}}}+1) \enspace .
  \eqlabel{das-front}
\end{equation}

On the other hand, if $\mbox{\texttt{{\color{var}i}}}\ge\mbox{\texttt{{\color{var}front}.size()}}$, then this translates into
$\mbox{\texttt{{\color{var}back}.add({\color{var}i}-{\color{var}front}.size(),{\color{var}x})}}$.  The cost of this is 
\begin{equation}
  O(\mbox{\texttt{{\color{var}back}.size()}}-(\mbox{\texttt{{\color{var}i}}}-\mbox{\texttt{{\color{var}front}.size()}})+1) = O(\mbox{\texttt{size()}}-\mbox{\texttt{{\color{var}i}}}+1)= O(\mbox{\texttt{{\color{var}n}}}-\mbox{\texttt{{\color{var}i}}}+1)\enspace .
  \eqlabel{das-back}
\end{equation}

Notice that the first case \eqref{das-front} occurs when $\mbox{\texttt{{\color{var}i}}}<\mbox{\texttt{{\color{var}n}}}/3$.
The second case \eqref{das-back} occurs when $\mbox{\texttt{{\color{var}i}}}>\mbox{\texttt{{\color{var}n}}}/3$.  When
$\mbox{\texttt{{\color{var}n}}}/3\le\mbox{\texttt{{\color{var}i}}}\le2\mbox{\texttt{{\color{var}n}}}/3$, we can't be sure whether the operation affects
\mbox{\texttt{{\color{var}front}}} or \mbox{\texttt{{\color{var}back}}}, but in either case, the operation takes $O(\mbox{\texttt{{\color{var}n}}})=O(\mbox{\texttt{{\color{var}i}}})$
time, since $\mbox{\texttt{{\color{var}i}}}\ge \mbox{\texttt{{\color{var}n}}}/3$.  Summarizing the situation, we have
\[
     \mbox{Running time of } \mbox{\texttt{add({\color{var}i},{\color{var}x})}} \le 
          \left\{\begin{array}{ll}
            O(1+ \mbox{\texttt{{\color{var}i}}}) & \mbox{if $\mbox{\texttt{{\color{var}i}}}< \mbox{\texttt{{\color{var}n}}}/3$} \\
            O(\mbox{\texttt{{\color{var}n}}}) & \mbox{if $\mbox{\texttt{{\color{var}n}}}/3 \le \mbox{\texttt{{\color{var}i}}} \le 2\mbox{\texttt{{\color{var}n}}}/3$} \\
            O(1+\mbox{\texttt{{\color{var}n}}}-\mbox{\texttt{{\color{var}i}}}) & \mbox{if $\mbox{\texttt{{\color{var}i}}} > 2\mbox{\texttt{{\color{var}n}}}/3$}
          \end{array}\right.
\]
Thus, the running time of \mbox{\texttt{add({\color{var}i},{\color{var}x})}} (ignoring the cost of the call to
\mbox{\texttt{balance()}}) is $O(1+\min\{\mbox{\texttt{{\color{var}i}}}, \mbox{\texttt{{\color{var}n}}}-\mbox{\texttt{{\color{var}i}}}\})$.

The \mbox{\texttt{remove({\color{var}i})}} operation, and its analysis, is similar to the \mbox{\texttt{add({\color{var}i},{\color{var}x})}}
operation.

\begin{Verbatim}[tabsize=2,frame=single,commandchars=\\@\$,label=\texttt{DualArrayDeque},labelposition=topline]
	@\color@keyword$T$ remove(@\color@keyword$int$ @\color@var$i$) {
		@\color@keyword$T$ @\color@var$x$;
		@\color@keyword$if$ (@\color@var$i$ < @\color@var$front$.size()) {
			@\color@var$x$ = @\color@var$front$.remove(@\color@var$front$.size()-@\color@var$i$-1);
		} @\color@keyword$else$ {
			@\color@var$x$ = @\color@var$back$.remove(@\color@var$i$-@\color@var$front$.size());
		}
		balance();
		@\color@var$return$ @\color@var$x$;
	}
\end{Verbatim}

\subsection{Balancing}

Finally, we study the \mbox{\texttt{balance()}} operation performed by \mbox{\texttt{add({\color{var}i},{\color{var}x})}}
and \mbox{\texttt{remove({\color{var}i})}}.  This operation is used to ensure that neither \mbox{\texttt{{\color{var}front}}}
nor \mbox{\texttt{{\color{var}back}}} gets too big (or too small).  It ensures that, unless there
are fewer than 2 elements, each of \mbox{\texttt{{\color{var}front}}} and \mbox{\texttt{{\color{var}back}}} contain at least
$\mbox{\texttt{{\color{var}n}}}/3$ elements. If this is not the case, then it moves elements between
them so that \mbox{\texttt{{\color{var}front}}} and \mbox{\texttt{{\color{var}back}}} contain exactly $\lfloor\mbox{\texttt{{\color{var}n}}}/2\rfloor$
elements and $\lceil\mbox{\texttt{{\color{var}n}}}/2\rceil$ elements, respectively.

\begin{Verbatim}[tabsize=2,frame=single,commandchars=\\@\$,label=\texttt{DualArrayDeque},labelposition=topline]
	@\color@var$void$ balance() {
		@\color@keyword$int$ @\color@var$n$ = size();
		@\color@keyword$if$ (3*@\color@var$front$.size() < @\color@var$back$.size()) {
			@\color@keyword$int$ @\color@var$s$ = @\color@var$n$/2 - @\color@var$front$.size();
			List<@\color@keyword$T$> @\color@var$l1$ = newStack();
			List<@\color@keyword$T$> @\color@var$l2$ = newStack();
			@\color@var$l1$.addAll(@\color@var$back$.subList(0,@\color@var$s$));
			Collections.reverse(@\color@var$l1$);
			@\color@var$l1$.addAll(@\color@var$front$);
			@\color@var$l2$.addAll(@\color@var$back$.subList(@\color@var$s$, @\color@var$back$.size()));
			@\color@var$front$ = @\color@var$l1$;
			@\color@var$back$ = @\color@var$l2$;
		} @\color@keyword$else$ @\color@keyword$if$ (3*@\color@var$back$.size() < @\color@var$front$.size()) {
			@\color@keyword$int$ @\color@var$s$ = @\color@var$front$.size() - @\color@var$n$/2;
			List<@\color@keyword$T$> @\color@var$l1$ = newStack();
			List<@\color@keyword$T$> @\color@var$l2$ = newStack();
			@\color@var$l1$.addAll(@\color@var$front$.subList(@\color@var$s$, @\color@var$front$.size()));
			@\color@var$l2$.addAll(@\color@var$front$.subList(0, @\color@var$s$));
			Collections.reverse(@\color@var$l2$);
			@\color@var$l2$.addAll(@\color@var$back$);
			@\color@var$front$ = @\color@var$l1$;
			@\color@var$back$ = @\color@var$l2$;
		}
	}
\end{Verbatim}

There is not much to analyze.  If the \mbox{\texttt{balance()}} operation does
do rebalancing, then it moves $\Theta(\mbox{\texttt{{\color{var}n}}})$ elements and this takes
$O(\mbox{\texttt{{\color{var}n}}})$ time. This is bad, since \mbox{\texttt{balance()}} is called with each call to
\mbox{\texttt{add({\color{var}i},{\color{var}x})}} and \mbox{\texttt{remove({\color{var}i})}}.  However, the following theorem shows that,
on average, \mbox{\texttt{balance()}} only does a constant amount of work per operation

\begin{lem}\lemlabel{dualarraydeque-amortized}
  If an empty \mbox{\texttt{DualArrayDeque}} is created and any sequence of $m\ge 1$ calls
  to \mbox{\texttt{add({\color{var}i},{\color{var}x})}} and \mbox{\texttt{remove({\color{var}i})}} are performed, then the total cost of all work
  done during all calls to \mbox{\texttt{balance()}} is $O(m)$.
\end{lem}

\begin{proof}
  We will show that, if \mbox{\texttt{balance()}} is forced to shift elements,
  then the number of \mbox{\texttt{add({\color{var}i},{\color{var}x})}} and \mbox{\texttt{remove({\color{var}i})}} operations since the
  last time \mbox{\texttt{balance()}} shifted any elements was at least $\mbox{\texttt{{\color{var}n}}}/2-1$.
  As in the proof of \lemref{arraystack-amortized}, this is sufficient
  to prove that the total work done by \mbox{\texttt{balance()}} is $O(m)$.

  We will perform our analysis using the \emph{potential method}.
  Define the \emph{potential} of the \mbox{\texttt{DualArrayDeque}} as
  \[  \Phi = |\mbox{\texttt{{\color{var}front}.size()}} - \mbox{\texttt{{\color{var}back}.size()}}| \enspace . \]
  The interesting thing about this potential is that a call to \mbox{\texttt{add({\color{var}i},{\color{var}x})}}
  or \mbox{\texttt{remove({\color{var}i})}} that does not do any balancing can increase the potential
  by at most 1.

  Observe that, immediately after a call to \mbox{\texttt{balance()}} that shifts
  elements, the potential, $\Phi_0$, is at most 1, since
  \[ \Phi_0 = \left|\lfloor\mbox{\texttt{{\color{var}n}}}/2\rfloor-\lceil\mbox{\texttt{{\color{var}n}}}/2\rceil\right|\le 1  \enspace .\]

  Now, consider the situation immediately before a call to \mbox{\texttt{balance()}} that
  shifts elements, and suppose, without loss of generality that \mbox{\texttt{balance()}}
  is shifting elements because $3\mbox{\texttt{{\color{var}front}.size()}} < \mbox{\texttt{{\color{var}back}.size()}}$.
  Notice that, in this case,
  \begin{eqnarray*}
   \mbox{\texttt{{\color{var}n}}} & = & \mbox{\texttt{{\color{var}front}.size()}}+\mbox{\texttt{{\color{var}back}.size()}} \\
       & \le & \mbox{\texttt{{\color{var}back}.size()/3}}+\mbox{\texttt{{\color{var}back}.size()}} \\
       & = & \frac{4}{3}\mbox{\texttt{{\color{var}back}.size()}}
  \end{eqnarray*}
  Furthermore, the potential at this point in time is
  \begin{eqnarray*}
  \Phi_1 & = & \mbox{\texttt{{\color{var}back}.size()}} - \mbox{\texttt{{\color{var}front}.size()}} \\
      &>& \mbox{\texttt{{\color{var}back}.size()}} - \mbox{\texttt{{\color{var}back}.size()/3}} \\
      &=& \frac{2}{3}\mbox{\texttt{{\color{var}back}.size()}} \\
      &\ge& \frac{3}{4}\times\frac{2}{3}\mbox{\texttt{{\color{var}n}}} \\
      &=& \mbox{\texttt{{\color{var}n}}}/2
  \end{eqnarray*}
  Therefore, the number of calls to \mbox{\texttt{add({\color{var}i},{\color{var}x})}} or \mbox{\texttt{remove({\color{var}i})}} since
  the last time \mbox{\texttt{balance()}} shifted elements is at least $\Phi_1-\Phi_0
  \ge \mbox{\texttt{{\color{var}n}}}/2-1$. This completes the proof.
\end{proof}

\subsection{Summary}

The following theorem summarizes the performance of a \mbox{\texttt{DualArrayStack}}

\begin{thm}\thmlabel{dualarraydeque}
  A \mbox{\texttt{DualArrayDeque}} implements the \mbox{\texttt{List}} interface.  Ignoring the
  cost of calls to \mbox{\texttt{resize()}} and \mbox{\texttt{balance()}}, an \mbox{\texttt{DualArrayDeque}}
  supports the operations
  \begin{itemize}
    \item \mbox{\texttt{get({\color{var}i})}} and \mbox{\texttt{set({\color{var}i},{\color{var}x})}} in $O(1)$ time per operation; and
    \item \mbox{\texttt{add({\color{var}i},{\color{var}x})}} and \mbox{\texttt{remove({\color{var}i})}} in $O(1+\min\{\mbox{\texttt{{\color{var}i}}},\mbox{\texttt{{\color{var}n}}}-\mbox{\texttt{{\color{var}i}}}\})$ time
          per operation.
  \end{itemize}
  Furthermore, beginning with an empty \mbox{\texttt{DualArrayDeque}}, any sequence of $m$
  \mbox{\texttt{add({\color{var}i},{\color{var}x})}} and \mbox{\texttt{remove({\color{var}i})}} operations results in a total of $O(m)$
  work during all calls to \mbox{\texttt{resize()}} and \mbox{\texttt{balance()}}.
\end{thm}


\section{\mbox{\texttt{RootishArrayStack}}: A Space-Efficient Array Stack}

One of the drawbacks of all previous data structures in this chapter
is that, because they store their data in one or two arrays, and they
don't want to resize these arrays too often, the arrays are frequently
not very full.  For example, immediately after a \mbox{\texttt{resize()}} operation on
an \mbox{\texttt{ArrayStack}}, the backing array \mbox{\texttt{{\color{var}a}}} is only half full.  Even worse,
there are times when only $1/3$ of \mbox{\texttt{{\color{var}a}}} contains data.

In this section, we discuss a data structure, the \mbox{\texttt{RootishArrayStack}},
that addresses this problem of wasted space.  The \mbox{\texttt{RootishArrayStack}}
stores \mbox{\texttt{{\color{var}n}}} elements using $O(\sqrt{\mbox{\texttt{{\color{var}n}}}})$ arrays.  In these arrays, at
most $O(\sqrt{\mbox{\texttt{{\color{var}n}}}})$ array locations are unused.  All remaining array
locations are used to store data.  Therefore, these data structures waste
at most $O(\sqrt{\mbox{\texttt{{\color{var}n}}}})$ space when storing \mbox{\texttt{{\color{var}n}}} elements.

A \mbox{\texttt{RootishArrayStack}} stores its elements in a list of \mbox{\texttt{{\color{var}r}}}
arrays called \emph{blocks} that are numbered $0,1,\ldots,\mbox{\texttt{{\color{var}r}}}-1$.
See \figref{rootisharraystack}.  Block $b$ contains $b+1$ elements.
Therefore, all \mbox{\texttt{{\color{var}r}}} blocks contain a total of
\[
  1+ 2+ 3+\cdots +\mbox{\texttt{{\color{var}r}}} = \mbox{\texttt{{\color{var}r}}}(\mbox{\texttt{{\color{var}r}}}+1)/2
\]
elements.  The above formula (allegedly discovered by the Gau\ss\ at
the age of 9) can be obtained as shown in \figref{gauss}.

\begin{figure}
  \begin{center}
    \includegraphics{figs/rootisharraystack}
  \end{center}
  \caption{A sequence of \mbox{\texttt{add({\color{var}i},{\color{var}x})}} and \mbox{\texttt{remove({\color{var}i})}} operation in a
  \mbox{\texttt{RootishArrayStack}}.  Arrows denote elements being copied. }
  \figlabel{rootisharraystack}
\end{figure}

\begin{Verbatim}[tabsize=2,frame=single,commandchars=\\@\$,label=\texttt{RootishArrayStack},labelposition=topline]
	List<@\color@keyword$T$[]> @\color@var$blocks$;
	@\color@keyword$int$ @\color@var$n$;
\end{Verbatim}

\begin{figure}
  \begin{center}
    \includegraphics{figs/gauss}
  \end{center}
  \caption{The number of white squares is $1+2+3+\cdots+\mbox{\texttt{{\color{var}r}}}$.  The number of
  shaded squares is the same.  Together the white and shaded squares make a
  rectangle consisting of $\mbox{\texttt{{\color{var}r}}}(\mbox{\texttt{{\color{var}r}}}+1)$ squares.}
  \figlabel{gauss}
\end{figure}

The elements of the list are layed out in the blocks as we might expect.
The list element with index 0 is stored in block 0,  the elements with
list indices 1 and 2 are stored in block 1, the elements with list
indices 3, 4, and 5 are stored in block 2, and so on.  The main problem
we have to address is that of determining, given an index $\mbox{\texttt{{\color{var}i}}}$, which
block contains \mbox{\texttt{{\color{var}i}}} as well as the index of \mbox{\texttt{{\color{var}i}}} within its block.

Determining the index of \mbox{\texttt{{\color{var}i}}} within its block turns out to be easy. If
index \mbox{\texttt{{\color{var}i}}} is in block \mbox{\texttt{{\color{var}b}}}, then the number of elements in blocks
$0,\ldots,\mbox{\texttt{{\color{var}b}}}-1$ is $\mbox{\texttt{{\color{var}b}}}(\mbox{\texttt{{\color{var}b}}}+1)/2$.  Therefore, \mbox{\texttt{{\color{var}i}}} is stored at location
\[
     \mbox{\texttt{{\color{var}j}}} = \mbox{\texttt{{\color{var}i}}} - \mbox{\texttt{{\color{var}b}}}(\mbox{\texttt{{\color{var}b}}}+1)/2
\]
within block \mbox{\texttt{{\color{var}b}}}.  Somewhat more challenging is the problem of determining
the value of \mbox{\texttt{{\color{var}b}}}.  The number of elements that have indices less than
or equal to \mbox{\texttt{{\color{var}i}}} is $\mbox{\texttt{{\color{var}i}}}+1$.  On the other hand, the number of elements
in blocks 0,\ldots,b is $(\mbox{\texttt{{\color{var}b}}}+1)(\mbox{\texttt{{\color{var}b}}}+2)/2$.  Therefore, \mbox{\texttt{{\color{var}b}}} is smallest
integer such that
\[
    (\mbox{\texttt{{\color{var}b}}}+1)(\mbox{\texttt{{\color{var}b}}}+2)/2 \ge \mbox{\texttt{{\color{var}i}}}+1 \enspace .
\]
We can rewrite this equation as
\[
    \mbox{\texttt{{\color{var}b}}}^2 + 3\mbox{\texttt{{\color{var}b}}} - 2\mbox{\texttt{{\color{var}i}}} \ge  0 \enspace .
\]
The corresponding quadratic equation $\mbox{\texttt{{\color{var}b}}}^2 + 3\mbox{\texttt{{\color{var}b}}} - 2\mbox{\texttt{{\color{var}i}}} =  0$ has two
solutions: $\mbox{\texttt{{\color{var}b}}}=(-3 + \sqrt{9+8\mbox{\texttt{{\color{var}i}}}}) / 2$ and $\mbox{\texttt{{\color{var}b}}}=(-3 - \sqrt{9+8\mbox{\texttt{{\color{var}i}}}}) / 2$.
The second solution makes no sense in our application since it always
gives a negative value. Therefore, we obtain the solution $b = (-3 +
\sqrt{9+8i}) / 2$.  In general, this solution is not an integer, but
going back to our inequality, we want the smallest integer $b$ such that 
$b \ge (-3 + \sqrt{9+8i}) / 2$.  This is simply
\[
   b = \left\lceil(-3 + \sqrt{9+8i}) / 2\right\rceil \enspace .
\]

\begin{Verbatim}[tabsize=2,frame=single,commandchars=\\@\$,label=\texttt{RootishArrayStack},labelposition=topline]
	 @\color@keyword$int$ i2b(@\color@keyword$int$ @\color@var$i$) {
		@\color@keyword$double$ @\color@var$db$ = (-3.0 + Math.sqrt(9 + 8*@\color@var$i$)) / 2.0;
		@\color@keyword$int$ @\color@var$b$ = (@\color@keyword$int$)Math.ceil(@\color@var$db$);
		@\color@var$return$ @\color@var$b$; 
	}
\end{Verbatim}

With this out of the way, the \mbox{\texttt{get({\color{var}i})}} and \mbox{\texttt{set({\color{var}i},{\color{var}x})}} methods are straightforward.  We first compute the appropriate block \mbox{\texttt{{\color{var}b}}} and the appropriate index \mbox{\texttt{{\color{var}j}}} within the block and then perform the appropriate operation:

\begin{Verbatim}[tabsize=2,frame=single,commandchars=\\@\$,label=\texttt{RootishArrayStack},labelposition=topline]
	@\color@keyword$T$ get(@\color@keyword$int$ @\color@var$i$) {
		@\color@keyword$if$ (@\color@var$i$ < 0 || @\color@var$i$ > @\color@var$n$ - 1) @\color@keyword$throw$ @\color@keyword$new$ IndexOutOfBoundsException();
		@\color@keyword$int$ @\color@var$b$ = i2b(@\color@var$i$);
		@\color@keyword$int$ @\color@var$j$ = @\color@var$i$ - @\color@var$b$*(@\color@var$b$+1)/2;
		@\color@var$return$ @\color@var$blocks$.get(@\color@var$b$)[@\color@var$j$];
	}
	@\color@keyword$T$ set(@\color@keyword$int$ @\color@var$i$, @\color@keyword$T$ @\color@var$x$) {
		@\color@keyword$if$ (@\color@var$i$ < 0 || @\color@var$i$ > @\color@var$n$ - 1) @\color@keyword$throw$ @\color@keyword$new$ IndexOutOfBoundsException();
		@\color@keyword$int$ @\color@var$b$ = i2b(@\color@var$i$);
		@\color@keyword$int$ @\color@var$j$ = @\color@var$i$ - @\color@var$b$*(@\color@var$b$+1)/2;
		@\color@keyword$T$ @\color@var$y$ = @\color@var$blocks$.get(@\color@var$b$)[@\color@var$j$];
		@\color@var$blocks$.get(@\color@var$b$)[@\color@var$j$] = @\color@var$x$;
		@\color@var$return$ @\color@var$y$;
	}
\end{Verbatim}

If we use any of the data structures in this chapter for representing the \mbox{\texttt{{\color{var}blocks}}} list, then \mbox{\texttt{get({\color{var}i})}} and \mbox{\texttt{set({\color{var}i},{\color{var}x})}} will each run in constant time.

The \mbox{\texttt{add({\color{var}i},{\color{var}x})}} method will, by now, look familiar.  We first check if
our data structure is full, by checking if the number of blocks \mbox{\texttt{{\color{var}r}}}
is such that that $\mbox{\texttt{{\color{var}r}}}(\mbox{\texttt{{\color{var}r}}}+1) = \mbox{\texttt{{\color{var}n}}}$ and, if so, we call \mbox{\texttt{grow()}}
to add another block.  With this done, we shift elements with indices
$\mbox{\texttt{{\color{var}i}}},\ldots,\mbox{\texttt{{\color{var}n}}}-1$ to the right by one position to make room for the
new element with index \mbox{\texttt{{\color{var}i}}}:

\begin{Verbatim}[tabsize=2,frame=single,commandchars=\\@\$,label=\texttt{RootishArrayStack},labelposition=topline]
	@\color@var$void$ add(@\color@keyword$int$ @\color@var$i$, @\color@keyword$T$ @\color@var$x$) {
		@\color@keyword$if$ (@\color@var$i$ < 0 || @\color@var$i$ > @\color@var$n$) @\color@keyword$throw$ @\color@keyword$new$ IndexOutOfBoundsException();
		@\color@keyword$int$ @\color@var$r$ = @\color@var$blocks$.size();
		@\color@keyword$if$ (@\color@var$r$*(@\color@var$r$+1)/2 < @\color@var$n$ + 1) grow();
		@\color@var$n$++;
		@\color@keyword$for$ (@\color@keyword$int$ @\color@var$j$ = @\color@var$n$-1; @\color@var$j$ > @\color@var$i$; @\color@var$j$--)
			set(@\color@var$j$, get(@\color@var$j$-1));
		set(@\color@var$i$, @\color@var$x$);
	}
\end{Verbatim}

The \mbox{\texttt{grow()}} method does what we expect. It adds a new block:

\begin{Verbatim}[tabsize=2,frame=single,commandchars=\\@\$,label=\texttt{RootishArrayStack},labelposition=topline]
	@\color@var$void$ grow() {
		@\color@var$blocks$.add(newArray(@\color@var$blocks$.size()+1));
	}
\end{Verbatim}

Ignoring the cost of the \mbox{\texttt{grow()}} operation, the cost of an \mbox{\texttt{add({\color{var}i},{\color{var}x})}}
operation is dominated by the cost of shifting and is therefore
$O(1+\mbox{\texttt{{\color{var}n}}}-\mbox{\texttt{{\color{var}i}}})$, just like an \mbox{\texttt{ArrayStack}}.

The \mbox{\texttt{remove({\color{var}i})}} operation is similar to \mbox{\texttt{add({\color{var}i},{\color{var}x})}}.  It shifts the
elements with indices $\mbox{\texttt{{\color{var}i}}}+1,\ldots,\mbox{\texttt{{\color{var}n}}}$ left by one position and then,
if there is more than one empty block, it calls the \mbox{\texttt{shrink()}} method
to remove all but one of the unused blocks:

\begin{Verbatim}[tabsize=2,frame=single,commandchars=\\@\$,label=\texttt{RootishArrayStack},labelposition=topline]
	@\color@keyword$T$ remove(@\color@keyword$int$ @\color@var$i$) {
		@\color@keyword$if$ (@\color@var$i$ < 0 || @\color@var$i$ > @\color@var$n$ - 1) @\color@keyword$throw$ @\color@keyword$new$ IndexOutOfBoundsException();
		@\color@keyword$T$ @\color@var$x$ = get(@\color@var$i$);
		@\color@keyword$for$ (@\color@keyword$int$ @\color@var$j$ = @\color@var$i$; @\color@var$j$ < @\color@var$n$-1; @\color@var$j$++)
			set(@\color@var$j$, get(@\color@var$j$+1));
		@\color@var$n$--;
		@\color@keyword$int$ @\color@var$r$ = @\color@var$blocks$.size();
		@\color@keyword$if$ ((@\color@var$r$-2)*(@\color@var$r$-1)/2 >= @\color@var$n$)	shrink();
		@\color@var$return$ @\color@var$x$;
	}
\end{Verbatim}
\begin{Verbatim}[tabsize=2,frame=single,commandchars=\\@\$,label=\texttt{RootishArrayStack},labelposition=topline]
	@\color@var$void$ shrink() {
		@\color@keyword$int$ @\color@var$r$ = @\color@var$blocks$.size();
		@\color@keyword$while$ (@\color@var$r$ > 0 && (@\color@var$r$-2)*(@\color@var$r$-1)/2 >= @\color@var$n$) {
			@\color@var$blocks$.remove(@\color@var$blocks$.size()-1);
			@\color@var$r$--;
		}
	}
\end{Verbatim}

Once again, ignoring the cost of the \mbox{\texttt{shrink()}} operation, the cost of
a \mbox{\texttt{remove({\color{var}i})}} operation is dominated by the cost of shifting  and is
therefore $O(\mbox{\texttt{{\color{var}n}}}-\mbox{\texttt{{\color{var}i}}})$

\subsection{Analysis of Growing and Shrinking}

The analysis of \mbox{\texttt{add({\color{var}i},{\color{var}x})}} and \mbox{\texttt{remove({\color{var}i})}} does not account for the cost
of \mbox{\texttt{grow()}} and \mbox{\texttt{shrink()}}.  Note that, unlike the \mbox{\texttt{ArrayStack.resize()}}
operation, \mbox{\texttt{grow()}} and \mbox{\texttt{shrink()}} do not do any copying of data.
They only allocate or free an array of size \mbox{\texttt{{\color{var}r}}}.  In some
environments, this takes only constant time, while in others, it may
require $\Theta(\mbox{\texttt{{\color{var}r}}}))$ time.

We note that, immediately after a call to \mbox{\texttt{grow()}} or \mbox{\texttt{shrink()}}, the
situation is clear. The final block is completely empty and all other
blocks are completely full.  Another call to \mbox{\texttt{grow()}} or \mbox{\texttt{shrink()}} will
not happen until at least $\mbox{\texttt{{\color{var}r}}}-1$ elements have been added or removed.
Therefore, even if \mbox{\texttt{grow()}} and \mbox{\texttt{shrink()}} take $O(\mbox{\texttt{{\color{var}r}}})$ time, this
cost can be amortized over at least $\mbox{\texttt{{\color{var}r}}}-1$ \mbox{\texttt{add({\color{var}i},{\color{var}x})}} or \mbox{\texttt{remove({\color{var}i})}}
operations, so that the amortized cost of \mbox{\texttt{grow()}} and \mbox{\texttt{shrink()}} is
$O(1)$ per operation.

\subsection{Space Usage}

Next, we analyze the amount of extra space used by a \mbox{\texttt{RootishArrayStack}}.
In particular, we want to count any space used by a \mbox{\texttt{RootishArrayStack}} that is not an array element currently used to hold a list element.  We call all such space \emph{wasted space}.

The \mbox{\texttt{remove({\color{var}i})}} operation ensures that a \mbox{\texttt{RootishArrayStack}} never has
more than 2 blocks that are not completely full.  The number of blocks,
\mbox{\texttt{{\color{var}r}}}, used by a \mbox{\texttt{RootishArrayStack}} that stores \mbox{\texttt{{\color{var}n}}} elements therefore
satisfies
\[
    (\mbox{\texttt{{\color{var}r}}}-2)(\mbox{\texttt{{\color{var}r}}}-1) \le \mbox{\texttt{{\color{var}n}}}
\]
Again, using the quadratic equation on this gives
\[
   \mbox{\texttt{{\color{var}r}}} \le (3+\sqrt{1+4\mbox{\texttt{{\color{var}n}}}})/2 = O(\sqrt{\mbox{\texttt{{\color{var}n}}}})
\]
The last two blocks have sizes \mbox{\texttt{{\color{var}r}}} and \mbox{\texttt{{\color{var}r}-1}}, so the space wasted by these
two blocks is at most $2\mbox{\texttt{{\color{var}r}}} = O(\sqrt{\mbox{\texttt{{\color{var}n}}}})$.  If we store the blocks
in (for example) an \mbox{\texttt{ArrayList}}, then the amount of space wasted by the
\mbox{\texttt{List}} that stores those \mbox{\texttt{{\color{var}r}}} blocks is also $O(\mbox{\texttt{{\color{var}r}}})=O(\sqrt{\mbox{\texttt{{\color{var}n}}}})$.  The
other space for storing \mbox{\texttt{{\color{var}n}}} and other accounting information is $O(1)$.
Therefore, the total amount of wasted space in a \mbox{\texttt{RootishArrayStack}}
is $O(\sqrt{\mbox{\texttt{{\color{var}n}}}})$.

Next, we argue that this is optimal for any data structure that starts out
empty and can support the addition of one item at a time. More precisely,
we will show that, at some point during the addition of \mbox{\texttt{{\color{var}n}}} items, the
data structure has at least $\sqrt{\mbox{\texttt{{\color{var}n}}}}$ wasted space (though it may be
only for a moment).

Suppose we start with an empty data structure and we add \mbox{\texttt{{\color{var}n}}} items
one at a time.  At the end of this process, all \mbox{\texttt{{\color{var}n}}} items are stored
in the structure and they are distributed among a collection of \mbox{\texttt{{\color{var}r}}}
memory blocks.  If $\mbox{\texttt{{\color{var}r}}}\ge \sqrt{\mbox{\texttt{{\color{var}n}}}}$, then the data structure must be
using \mbox{\texttt{{\color{var}r}}} pointers (or references) to keep track of these \mbox{\texttt{{\color{var}r}}} blocks,
and this is wasted space.  On the other hand, if $\mbox{\texttt{{\color{var}r}}} < \sqrt{\mbox{\texttt{{\color{var}n}}}}$
then, by the pigeonhole principle, some block must have size at least
$\mbox{\texttt{{\color{var}n}}}/\mbox{\texttt{{\color{var}r}}} > \sqrt{\mbox{\texttt{{\color{var}n}}}}$.  Consider the moment at which this block was
first allocated.  Immediately after it was allocated, this block was
empty, and was therefore wasting $\sqrt{\mbox{\texttt{{\color{var}n}}}}$ space.  Therefore, at some
point in time during the insertion of \mbox{\texttt{{\color{var}n}}} elements, the data structure was
wasting $O(\sqrt{\mbox{\texttt{{\color{var}n}}}})$ space.

\subsection{Summary}

The following theorem summarizes the performance of the \mbox{\texttt{RootishArrayStack}}
data structure:

\begin{thm}\thmlabel{rootisharraystack}
  A \mbox{\texttt{RootishArrayStack}} implements the \mbox{\texttt{List}} interface.  Ignoring the cost of
  calls to \mbox{\texttt{grow()}} and \mbox{\texttt{shrink()}}, a \mbox{\texttt{RootishArrayStack}} supports the operations
  \begin{itemize}
    \item \mbox{\texttt{get({\color{var}i})}} and \mbox{\texttt{set({\color{var}i},{\color{var}x})}} in $O(1)$ time per operation; and
    \item \mbox{\texttt{add({\color{var}i},{\color{var}x})}} and \mbox{\texttt{remove({\color{var}i})}} in $O(1+\mbox{\texttt{{\color{var}n}}}-\mbox{\texttt{{\color{var}i}}})$ time per operation.
  \end{itemize}
  Furthermore, beginning with an empty \mbox{\texttt{RootishArrayStack}}, any sequence of $m$
  \mbox{\texttt{add({\color{var}i},{\color{var}x})}} and \mbox{\texttt{remove({\color{var}i})}} operations results in a total of $O(m)$
  work during all calls to \mbox{\texttt{grow()}} and \mbox{\texttt{shrink()}}.

  The space (measured in words)\footnote{Recall \secref{model} for a
  discussion of how memory is measured.} used by a \mbox{\texttt{RootishArrayStack}}
  that stores \mbox{\texttt{{\color{var}n}}} elements is $\mbox{\texttt{{\color{var}n}}} +O(\sqrt{\mbox{\texttt{{\color{var}n}}}})$.
\end{thm}

\section{Exercises}


\begin{enumerate}
\item The \mbox{\texttt{List}} method \mbox{\texttt{addAll({\color{var}i},{\color{var}c})}} inserts all elements of the
\mbox{\texttt{Collection}} \mbox{\texttt{{\color{var}c}}} into the list at position \mbox{\texttt{{\color{var}i}}}.  (The \mbox{\texttt{add({\color{var}i},{\color{var}x})}} method is a special case where $\mbox{\texttt{{\color{var}c}}}=\{\mbox{\texttt{{\color{var}x}}}\}$.)  Explain why, for the data structures in this chapter, it is not efficient to implement \mbox{\texttt{addAll({\color{var}i},{\color{var}c})}} by repeated calls to \mbox{\texttt{add({\color{var}i},{\color{var}x})}}.  Describe a more efficient implementation.

\item Design and implement a \mbox{\texttt{Treque}} (triple-ended queue). This is a \mbox{\texttt{List}} implementation in which \mbox{\texttt{get({\color{var}i})}} and \mbox{\texttt{set({\color{var}i},{\color{var}x})}} run in constant time and \mbox{\texttt{add({\color{var}i},{\color{var}x})}} and \mbox{\texttt{remove({\color{var}i})}} run in time
\[
   O(1+\min\{\mbox{\texttt{{\color{var}i}}}, \mbox{\texttt{{\color{var}n}}}-\mbox{\texttt{{\color{var}i}}}, |\mbox{\texttt{{\color{var}n}}}/2|-\mbox{\texttt{{\color{var}i}}}\}) \enspace .
\]

\item Implement a method \mbox{\texttt{rotate({\color{var}r})}} that ``rotates'' a \mbox{\texttt{List}} so
that list item \mbox{\texttt{{\color{var}i}}} becomes list item $(\mbox{\texttt{{\color{var}i}}}+\mbox{\texttt{{\color{var}r}}})\bmod n$.  When run
on an \mbox{\texttt{ArrayDeque}}, or a \mbox{\texttt{DualArrayDeque}}, \mbox{\texttt{rotate({\color{var}r})}} should run in
$O(1+\min\{r,n-r\})$.

The \mbox{\texttt{List}} method \mbox{\texttt{addAll({\color{var}i},{\color{var}c})}} inserts all elements of the
\mbox{\texttt{Collection}} \mbox{\texttt{{\color{var}c}}} into the list at position \mbox{\texttt{{\color{var}i}}}.  (The \mbox{\texttt{add({\color{var}i},{\color{var}x})}} method is a special case where $\mbox{\texttt{{\color{var}c}}}=\{\mbox{\texttt{{\color{var}x}}}\}$.)  Explain why, for the data structures in this chapter, it is not efficient to implement \mbox{\texttt{addAll({\color{var}i},{\color{var}c})}} by repeated calls to \mbox{\texttt{add({\color{var}i},{\color{var}x})}}.  Describe a more efficient implementation.


\item Design and implement a \mbox{\texttt{RandomQueue}}.  This is an implementation of \mbox{\texttt{Queue}} in which the \mbox{\texttt{remove()}} operation removes an element chosen uniformly at random from all of the elements currently stored in the queue.  The \mbox{\texttt{add({\color{var}x})}} and \mbox{\texttt{remove()}} operations should run in $O(1)$ time.

\item Modify the \mbox{\texttt{ArrayDeque}} implementation so that the shifting done by \mbox{\texttt{add({\color{var}i},{\color{var}x})}}, \mbox{\texttt{remove({\color{var}i})}}, and \mbox{\texttt{resize()}} is done using \mbox{\texttt{System.arraycopy({\color{var}s},{\color{var}i},{\color{var}d},{\color{var}j},{\color{var}n})}}.

\item Modify the \mbox{\texttt{ArrayDeque}} implementation so that it does not use the
\mbox{\texttt{\%}} operator (which is expensive on some systems).  Instead, it should make
use of the fact that, if \mbox{\texttt{{\color{var}a}.{\color{var}length}}} is a power of 2, then
$\mbox{\texttt{{\color{var}k}\%{\color{var}a}.{\color{var}length}}}=\mbox{\texttt{{\color{var}k}\&({\color{var}a}.{\color{var}length}-1)}}$. (Here, \mbox{\texttt{\&}} is the bitwise-and operator
that takes the bitwise and of its two arguments.)

\item Design and implement a version of a \mbox{\texttt{RootishArrayStack}} that has only $O(\sqrt{\mbox{\texttt{{\color{var}n}}}})$ wasted space, but that can perform \mbox{\texttt{add({\color{var}i},{\color{var}x})}} and \mbox{\texttt{remove({\color{var}i},{\color{var}x})}} operations in $O(1+\min\{\mbox{\texttt{{\color{var}i}}},\mbox{\texttt{{\color{var}n}}}-\mbox{\texttt{{\color{var}i}}}\})$ time.

\item Design and implement a version of a \mbox{\texttt{RootishArrayStack}} that has only
$O(\sqrt{\mbox{\texttt{{\color{var}n}}}})$ wasted space, but that can perform \mbox{\texttt{add({\color{var}i},{\color{var}x})}} and
\mbox{\texttt{remove({\color{var}i},{\color{var}x})}} operations in $O(1+\min\{\sqrt{\mbox{\texttt{{\color{var}n}}}},\mbox{\texttt{{\color{var}n}}}-\mbox{\texttt{{\color{var}i}}}\})$ time. (For an
idea on how to do this, see \secref{selist}.)

\item Design and implement a version of a \mbox{\texttt{RootishArrayStack}} that has only $O(\sqrt{\mbox{\texttt{{\color{var}n}}}})$ wasted space, but that can perform \mbox{\texttt{add({\color{var}i},{\color{var}x})}} and \mbox{\texttt{remove({\color{var}i},{\color{var}x})}} operations in $O(1+\min\{\mbox{\texttt{{\color{var}i}}},\sqrt {\mbox{\texttt{{\color{var}n}}}},\mbox{\texttt{{\color{var}n}}}-\mbox{\texttt{{\color{var}i}}}\})$ time.  (See \secref{selist} for ideas on how to achieve this.)
\end{enumerate}



\chapter{Linked Lists}

In this chapter, we continue to study implementations of the \mbox{\texttt{List}}
interface, this time using pointer-based data structures rather than
arrays.  The structures in this chapter are made up of nodes that
contain the list items.  The nodes are linked together into a sequence
using references (pointers).  We first study singly-linked lists, which
can implement \mbox{\texttt{Stack}} and (FIFO) \mbox{\texttt{Queue}} operations in constant time
per operation.

Compared to array-based list implementations, linked lists have their
advantages and disadvantages.  The primary disadvantage is that we
lose the ability to access any element using \mbox{\texttt{get({\color{var}i})}} or \mbox{\texttt{set({\color{var}i},{\color{var}x})}} in
constant time.  Instead, we have to walk through the list, one element
at a time, until we reach the \mbox{\texttt{{\color{var}i}}}th element.  The primary advantage is
that they are more dynamic:  Given a reference to any list node \mbox{\texttt{{\color{var}u}}}, we
can delete \mbox{\texttt{{\color{var}u}}} or insert a node adjacent to to \mbox{\texttt{{\color{var}u}}} in constant time. This
is true no matter where \mbox{\texttt{{\color{var}u}}} is in the list.


\section{\mbox{\texttt{SLList}}: A Singly-Linked List}
\seclabel{sllist}

An \mbox{\texttt{SLList}} (singly-linked list) is a sequence of \mbox{\texttt{Node}}s.  Each node
\mbox{\texttt{{\color{var}u}}} stores a data value \mbox{\texttt{{\color{var}u}.{\color{var}x}}} and a reference \mbox{\texttt{{\color{var}u}.{\color{var}next}}} to the next node in
the sequence.  For the last node \mbox{\texttt{{\color{var}w}}} in the sequence, $\mbox{\texttt{{\color{var}w}.{\color{var}next}}} = \mbox{\texttt{{\color{var}null}}}$

% TODO: Remove constructors from SLList.Node
\begin{Verbatim}[tabsize=2,frame=single,commandchars=\\@\$,label=\texttt{SLList},labelposition=topline]
	@\color@keyword$class$ Node {
		@\color@keyword$T$ @\color@var$x$;
		Node @\color@var$next$;
	}
\end{Verbatim}

For efficiency, an \mbox{\texttt{SLList}} uses variables \mbox{\texttt{{\color{var}head}}} and \mbox{\texttt{{\color{var}tail}}} to keep
track of the first and last node in the sequence, as well as an integer
\mbox{\texttt{{\color{var}n}}} to keep track of the length of the sequence:
\begin{Verbatim}[tabsize=2,frame=single,commandchars=\\@\$,label=\texttt{SLList},labelposition=topline]
	Node @\color@var$head$;
	Node @\color@var$tail$;
	@\color@keyword$int$ @\color@var$n$;
\end{Verbatim}
A sequence of \mbox{\texttt{Stack}} and \mbox{\texttt{Queue}} operations is illustrated on an
\figref{sllist} in \figref{sllist}.

\begin{figure}
  \begin{center}
    \includegraphics{figs/sllist}
  \end{center}
  \caption{A sequence of \mbox{\texttt{Queue}} (\mbox{\texttt{add({\color{var}x})}} and \mbox{\texttt{remove()}}) and \mbox{\texttt{Stack}} (\mbox{\texttt{push({\color{var}x})}} and \mbox{\texttt{pop()}}) operations in an \mbox{\texttt{SLList}}.}
  \figlabel{sllist}
\end{figure}


An \mbox{\texttt{SLList}} can efficiently implement the \mbox{\texttt{Stack}} operations \mbox{\texttt{push()}}
and \mbox{\texttt{pop()}} by adding and removing elements at the head of the sequence.
The \mbox{\texttt{push()}} operation simply creates a new node \mbox{\texttt{{\color{var}u}}} with data value \mbox{\texttt{{\color{var}x}}},
sets \mbox{\texttt{{\color{var}u}.{\color{var}next}}} to the old head of the list and makes \mbox{\texttt{{\color{var}u}}} the new head
of the list. Finally, it increments \mbox{\texttt{{\color{var}n}}} since the size of the \mbox{\texttt{SLList}}
has increased by one:

\begin{Verbatim}[tabsize=2,frame=single,commandchars=\\@\$,label=\texttt{SLList},labelposition=topline]
	@\color@keyword$T$ push(@\color@keyword$T$ @\color@var$x$) {
		Node @\color@var$u$ = @\color@keyword$new$ Node();
		@\color@var$u$.@\color@var$x$ = @\color@var$x$;
		@\color@var$u$.@\color@var$next$ = @\color@var$head$;;
		@\color@var$head$ = @\color@var$u$;
		@\color@keyword$if$ (@\color@var$n$ == 0)
			@\color@var$tail$ = @\color@var$u$;
		@\color@var$n$++;
		@\color@var$return$ @\color@var$x$;
	}
\end{Verbatim}

The \mbox{\texttt{pop()}} operation, after checking that the \mbox{\texttt{SLList}} is not empty,
removes the head by setting $\mbox{\texttt{{\color{var}head}}}=\mbox{\texttt{{\color{var}head}.{\color{var}next}}}$ and decrementing \mbox{\texttt{{\color{var}n}}}.
A special case occurs when the last element is being removed, in which case \mbox{\texttt{{\color{var}tail}}} is set to \mbox{\texttt{{\color{var}null}}}:

\begin{Verbatim}[tabsize=2,frame=single,commandchars=\\@\$,label=\texttt{SLList},labelposition=topline]
	@\color@keyword$T$ pop() {
		@\color@keyword$if$ (@\color@var$n$ == 0)	@\color@var$return$ @\color@var$null$;
		@\color@keyword$T$ @\color@var$x$ = @\color@var$head$.@\color@var$x$;
		@\color@var$head$ = @\color@var$head$.@\color@var$next$;
		@\color@keyword$if$ (--@\color@var$n$ == 0) @\color@var$tail$ = @\color@var$null$;
		@\color@var$return$ @\color@var$x$;
	}	
\end{Verbatim}

Clearly both the \mbox{\texttt{push({\color{var}x})}} and \mbox{\texttt{pop()}} operations run in $O(1)$ time.

\subsection{Queue Operations}

An \mbox{\texttt{SLList}} can also efficiently implement the FIFO queue operations \mbox{\texttt{add({\color{var}x})}} and \mbox{\texttt{remove()}}.  Removals are done from the head of the list, and are identical to the \mbox{\texttt{pop()}} operation:

\begin{Verbatim}[tabsize=2,frame=single,commandchars=\\@\$,label=\texttt{SLList},labelposition=topline]
	@\color@keyword$T$ remove() {
		@\color@keyword$if$ (@\color@var$n$ == 0)	@\color@var$return$ @\color@var$null$;
		@\color@keyword$T$ @\color@var$x$ = @\color@var$head$.@\color@var$x$;
		@\color@var$head$ = @\color@var$head$.@\color@var$next$;
		@\color@keyword$if$ (--@\color@var$n$ == 0) @\color@var$tail$ = @\color@var$null$;
		@\color@var$return$ @\color@var$x$;
	}	
\end{Verbatim}

Additions, on the other hand, are done at the tail of the list.  In most
cases, this is done by setting $\mbox{\texttt{{\color{var}tail}.{\color{var}next}}}=\mbox{\texttt{{\color{var}u}}}$, where \mbox{\texttt{{\color{var}u}}} is the newly
created node that contains \mbox{\texttt{{\color{var}x}}}.  However, a special case occurs when
$\mbox{\texttt{{\color{var}n}}}=0$, in which case $\mbox{\texttt{{\color{var}tail}}}=\mbox{\texttt{{\color{var}head}}}=\mbox{\texttt{{\color{var}null}}}$.  In this case, both \mbox{\texttt{{\color{var}tail}}}
and \mbox{\texttt{{\color{var}head}}} are set to \mbox{\texttt{{\color{var}u}}}.

\begin{Verbatim}[tabsize=2,frame=single,commandchars=\\@\$,label=\texttt{SLList},labelposition=topline]
	@\color@var$boolean$ add(@\color@keyword$T$ @\color@var$x$) {
		Node @\color@var$u$ = @\color@keyword$new$ Node();
		@\color@var$u$.@\color@var$x$ = @\color@var$x$;
		@\color@keyword$if$ (@\color@var$n$ == 0) {
			@\color@var$head$ = @\color@var$u$;
		} @\color@keyword$else$ {
			@\color@var$tail$.@\color@var$next$ = @\color@var$u$;
		}
		@\color@var$tail$ = @\color@var$u$;
		@\color@var$n$++;
		@\color@var$return$ @\color@var$true$;
	}
\end{Verbatim}

Clearly, both \mbox{\texttt{add({\color{var}x})}} and \mbox{\texttt{remove()}} take constant time.

\subsection{Summary}

The following theorem summarizes the performance of an \mbox{\texttt{SLList}}:

\begin{thm}\thmlabel{sllist}
  An \mbox{\texttt{SLList}} implements the \mbox{\texttt{Stack}} and (FIFO) \mbox{\texttt{Queue}} interfaces.  
  The \mbox{\texttt{push({\color{var}x})}}, \mbox{\texttt{pop()}}, \mbox{\texttt{add({\color{var}x})}} and \mbox{\texttt{remove()}} operations run
  in $O(1)$ time per operation.
\end{thm}

An \mbox{\texttt{SLList}} comes very close to implementing the full set of \mbox{\texttt{Deque}}
operations.  The only missing operation is removal from the tail
of an \mbox{\texttt{SLList}}.  Removing from the tail of an \mbox{\texttt{SLList}} is difficult
because it requires updating the value of \mbox{\texttt{{\color{var}tail}}} so that it points to
the node \mbox{\texttt{{\color{var}w}}} that precedes \mbox{\texttt{{\color{var}tail}}} in the \mbox{\texttt{SLList}}; this is the node \mbox{\texttt{{\color{var}w}}}
such that $\mbox{\texttt{{\color{var}w}.{\color{var}next}}}=\mbox{\texttt{{\color{var}tail}}}$.  Unfortunately, the only way to get to \mbox{\texttt{{\color{var}w}}}
is by traversing the \mbox{\texttt{SLList}} starting at \mbox{\texttt{{\color{var}head}}} and taking $\mbox{\texttt{{\color{var}n}}}-2$ steps.

\section{\mbox{\texttt{DLList}}: A Doubly-Linked List}
\seclabel{dllist}

A \mbox{\texttt{DLList}} (doubly-linked list) is very similar to an \mbox{\texttt{SLList}} except
that each node \mbox{\texttt{{\color{var}u}}} in a \mbox{\texttt{DLList}} has references to both the node \mbox{\texttt{{\color{var}u}.{\color{var}next}}}
that follows it and the node \mbox{\texttt{{\color{var}u}.{\color{var}prev}}} that precedes it.

\begin{Verbatim}[tabsize=2,frame=single,commandchars=\\@\$,label=\texttt{DLList},labelposition=topline]
	@\color@keyword$class$ Node {
		@\color@keyword$T$ @\color@var$x$;
		Node @\color@var$prev$, @\color@var$next$;
	}
\end{Verbatim}

When implementing an \mbox{\texttt{SLList}}, we saw that there were always some special
cases to worry about. For example, removing the last element from a
\mbox{\texttt{SLList}} or adding an element to an empty \mbox{\texttt{SLList}} requires special
care so that \mbox{\texttt{{\color{var}head}}} and \mbox{\texttt{{\color{var}tail}}} are correctly updated.  In a \mbox{\texttt{DLList}},
the number of these special cases increases considerably.  Perhaps the
cleanest way to take care of all these special cases in a \mbox{\texttt{DLList}} is to
introduce a \mbox{\texttt{{\color{var}dummy}}} node. This is a node that does not contain any data,
but acts as a placeholder so that there are no special nodes; every node
has both a \mbox{\texttt{{\color{var}next}}} and a \mbox{\texttt{{\color{var}prev}}}, with \mbox{\texttt{{\color{var}dummy}}} acting as the node that
follows the last node in the list and that precedes the first node in
the list.  In this way, the nodes of the list are (doubly-)linked into
a cycle, as illustrated in \figref{dllist}.

\begin{figure}
  \begin{center}
    \includegraphics{figs/dllist2}
  \end{center}
  \caption{A \mbox{\texttt{DLList}} containing a,b,c,d,e.}
  \figlabel{dllist}
\end{figure}


%TODO: Remove constructors from class Node

\begin{Verbatim}[tabsize=2,frame=single,commandchars=\\@\$,label=\texttt{DLList},labelposition=topline]
	@\color@keyword$int$ @\color@var$n$;
	Node @\color@var$dummy$;
	DLList() {
		@\color@var$dummy$ = @\color@keyword$new$ Node();
		@\color@var$dummy$.@\color@var$next$ = @\color@var$dummy$;
		@\color@var$dummy$.@\color@var$prev$ = @\color@var$dummy$;
		@\color@var$n$ = 0;
	}
\end{Verbatim}

Finding the node with a particular index in a \mbox{\texttt{DLList}} is easy;  we can
either start at the head of the list (\mbox{\texttt{{\color{var}dummy}.{\color{var}next}}}) and work forward,
or start at the tail of the list (\mbox{\texttt{{\color{var}dummy}.{\color{var}prev}}}) and work backward.
This allows us to reach the \mbox{\texttt{{\color{var}i}}}th node in $O(1+\min\{\mbox{\texttt{{\color{var}i}}},\mbox{\texttt{{\color{var}n}}}-\mbox{\texttt{{\color{var}i}}}\})$ time:

\begin{Verbatim}[tabsize=2,frame=single,commandchars=\\@\$,label=\texttt{DLList},labelposition=topline]
	Node getNode(@\color@keyword$int$ @\color@var$i$) {
		Node @\color@var$p$ = @\color@var$null$;
		@\color@keyword$if$ (@\color@var$i$ < @\color@var$n$ / 2) {
			@\color@var$p$ = @\color@var$dummy$.@\color@var$next$;
			@\color@keyword$for$ (@\color@keyword$int$ @\color@var$j$ = 0; @\color@var$j$ < @\color@var$i$; @\color@var$j$++)
				@\color@var$p$ = @\color@var$p$.@\color@var$next$;
		} @\color@keyword$else$ {
			@\color@var$p$ = @\color@var$dummy$;
			@\color@keyword$for$ (@\color@keyword$int$ @\color@var$j$ = @\color@var$n$; @\color@var$j$ > @\color@var$i$; @\color@var$j$--)
				@\color@var$p$ = @\color@var$p$.@\color@var$prev$;
		}
		return (@\color@var$p$);
	}
\end{Verbatim}

The \mbox{\texttt{get({\color{var}i})}} and \mbox{\texttt{set({\color{var}i},{\color{var}x})}} operations are now easy.  We first find the \mbox{\texttt{{\color{var}i}}}th node and then get or set it's \mbox{\texttt{{\color{var}x}}} value:

\begin{Verbatim}[tabsize=2,frame=single,commandchars=\\@\$,label=\texttt{DLList},labelposition=topline]
	@\color@keyword$T$ get(@\color@keyword$int$ @\color@var$i$) {
		@\color@keyword$if$ (@\color@var$i$ < 0 || @\color@var$i$ > @\color@var$n$ - 1) @\color@keyword$throw$ @\color@keyword$new$ IndexOutOfBoundsException();
		@\color@var$return$ getNode(@\color@var$i$).@\color@var$x$;
	}
	@\color@keyword$T$ set(@\color@keyword$int$ @\color@var$i$, @\color@keyword$T$ @\color@var$x$) {
		@\color@keyword$if$ (@\color@var$i$ < 0 || @\color@var$i$ > @\color@var$n$ - 1) @\color@keyword$throw$ @\color@keyword$new$ IndexOutOfBoundsException();
		Node @\color@var$u$ = getNode(@\color@var$i$);
		@\color@keyword$T$ @\color@var$y$ = @\color@var$u$.@\color@var$x$;
		@\color@var$u$.@\color@var$x$ = @\color@var$x$;
		@\color@var$return$ @\color@var$y$;
	}
\end{Verbatim}

The running time of these operations is dominated by the time it takes
to find the \mbox{\texttt{{\color{var}i}}}th node, and is therefore $O(1+\min\{\mbox{\texttt{{\color{var}i}}},\mbox{\texttt{{\color{var}n}}}-\mbox{\texttt{{\color{var}i}}}\})$.

\subsection{Adding and Removing}

If we have a reference to a node \mbox{\texttt{{\color{var}w}}} in a \mbox{\texttt{DLList}} and we want to insert a
node \mbox{\texttt{{\color{var}u}}} before \mbox{\texttt{{\color{var}w}}}, then this is just a matter of setting $\mbox{\texttt{{\color{var}u}.{\color{var}next}}}=\mbox{\texttt{{\color{var}w}}}$,
$\mbox{\texttt{{\color{var}u}.{\color{var}prev}}}=\mbox{\texttt{{\color{var}w}.{\color{var}prev}}}$, and then adjusting \mbox{\texttt{{\color{var}u}.{\color{var}prev}.{\color{var}next}}} and \mbox{\texttt{{\color{var}u}.{\color{var}next}.{\color{var}prev}}}.
Thanks to the dummy node, there is no need to worry about \mbox{\texttt{{\color{var}w}.{\color{var}prev}}}
or \mbox{\texttt{{\color{var}w}.{\color{var}next}}} not existing.

\begin{Verbatim}[tabsize=2,frame=single,commandchars=\\@\$,label=\texttt{DLList},labelposition=topline]
	Node addBefore(Node @\color@var$w$, @\color@keyword$T$ @\color@var$x$) {
		Node @\color@var$u$ = @\color@keyword$new$ Node();
		@\color@var$u$.@\color@var$x$ = @\color@var$x$;
		@\color@var$u$.@\color@var$prev$ = @\color@var$w$.@\color@var$prev$;
		@\color@var$u$.@\color@var$next$ = @\color@var$w$;
		@\color@var$u$.@\color@var$next$.@\color@var$prev$ = @\color@var$u$;
		@\color@var$u$.@\color@var$prev$.@\color@var$next$ = @\color@var$u$;
		@\color@var$n$++;
		@\color@var$return$ @\color@var$u$;
	}
\end{Verbatim}

Now, the list operation \mbox{\texttt{add({\color{var}i},{\color{var}x})}} is trivial to implement.  We find the
\mbox{\texttt{{\color{var}i}}}th node in the \mbox{\texttt{DLList}} and insert a new node \mbox{\texttt{{\color{var}u}}} that contains \mbox{\texttt{{\color{var}x}}}
just before it.

\begin{Verbatim}[tabsize=2,frame=single,commandchars=\\@\$,label=\texttt{DLList},labelposition=topline]
	@\color@var$void$ add(@\color@keyword$int$ @\color@var$i$, @\color@keyword$T$ @\color@var$x$) {
		@\color@keyword$if$ (@\color@var$i$ < 0 || @\color@var$i$ > @\color@var$n$) @\color@keyword$throw$ @\color@keyword$new$ IndexOutOfBoundsException();
		addBefore(getNode(@\color@var$i$), @\color@var$x$);
	}
\end{Verbatim}

The only non-constant part of the running time of \mbox{\texttt{add({\color{var}i},{\color{var}x})}} is the time
it takes to find the \mbox{\texttt{{\color{var}i}}}th node (using \mbox{\texttt{getNode({\color{var}i})}}).  Thus, \mbox{\texttt{add({\color{var}i},{\color{var}x})}}
runs in $O(1+\min\{\mbox{\texttt{{\color{var}i}}}, \mbox{\texttt{{\color{var}n}}}-\mbox{\texttt{{\color{var}i}}}\})$ time.

Removing a node \mbox{\texttt{{\color{var}w}}} from a \mbox{\texttt{DLList}} is easy.  We need only adjust pointers
at \mbox{\texttt{{\color{var}w}.{\color{var}next}}} and \mbox{\texttt{{\color{var}w}.{\color{var}prev}}} so that they skip over \mbox{\texttt{{\color{var}w}}}.  Again, the use of the dummy node eliminates the need to consider any special cases:

\begin{Verbatim}[tabsize=2,frame=single,commandchars=\\@\$,label=\texttt{DLList},labelposition=topline]
	@\color@var$void$ remove(Node @\color@var$w$) {
		@\color@var$w$.@\color@var$prev$.@\color@var$next$ = @\color@var$w$.@\color@var$next$;
		@\color@var$w$.@\color@var$next$.@\color@var$prev$ = @\color@var$w$.@\color@var$prev$;
		@\color@var$n$--;
	}
\end{Verbatim}

Now the \mbox{\texttt{remove({\color{var}i})}} operation is trivial. We find the node with index \mbox{\texttt{{\color{var}i}}} and remove it:

\begin{Verbatim}[tabsize=2,frame=single,commandchars=\\@\$,label=\texttt{DLList},labelposition=topline]
	@\color@keyword$T$ remove(@\color@keyword$int$ @\color@var$i$) {
		@\color@keyword$if$ (@\color@var$i$ < 0 || @\color@var$i$ > @\color@var$n$ - 1) @\color@keyword$throw$ @\color@keyword$new$ IndexOutOfBoundsException();
		Node @\color@var$w$ = getNode(@\color@var$i$);
		remove(@\color@var$w$);
		@\color@var$return$ @\color@var$w$.@\color@var$x$;
	}
\end{Verbatim}

Again, the only expensive part of this operation is finding the \mbox{\texttt{{\color{var}i}}}th node
using \mbox{\texttt{getNode({\color{var}i})}}, so \mbox{\texttt{remove({\color{var}i})}} runs in $O(1+\min\{\mbox{\texttt{{\color{var}i}}}, \mbox{\texttt{{\color{var}n}}}-\mbox{\texttt{{\color{var}i}}}\})$
time.

\subsection{Summary}

The following theorem summarizes the performance of a \mbox{\texttt{DLList}}:

\begin{thm}\thmlabel{dllist}
  A \mbox{\texttt{DLList}} implements the \mbox{\texttt{List}} interface.  
  The \mbox{\texttt{get({\color{var}i})}}, \mbox{\texttt{set({\color{var}i},{\color{var}x})}}, \mbox{\texttt{add({\color{var}i},{\color{var}x})}} and \mbox{\texttt{remove({\color{var}i})}} operations run
  in $O(1+\min\{\mbox{\texttt{{\color{var}i}}},\mbox{\texttt{{\color{var}n}}}-\mbox{\texttt{{\color{var}i}}}\})$ time per operation.
\end{thm}

It is worth noting that, if we ignore the cost of the \mbox{\texttt{getNode({\color{var}i})}}
operation, then all operations on a \mbox{\texttt{DLList}} take constant time.
Thus, the only expensive part of operations on a \mbox{\texttt{DLList}} is finding
the relevant node.  Once we have the relevant node, adding, removing,
or accessing the data at the node take only constant time.

This is in sharp contrast to the array-based \mbox{\texttt{List}} implementations of
\chapref{arrays}. In those implementations, finding the relevant array
item can be done in constant time. However, addition or removal requires
shifting elements in the array and, in general, takes non-constant time.

For this reason, linked list structures are well-suited to applications
where references to list nodes can be obtained through external means.
An example of this is the \mbox{\texttt{LinkedHashSet}} data structure found in the
Java Collections Framework, in which a set of items is stored in a
doubly-linked list and the nodes of the doubly-linked list are stored
in a hash table (discussed in \chapref{hashing}).  When elements
are removed from a \mbox{\texttt{LinkedHashSet}}, the hash table is used to find the
relevant list node in constant time and then the list node is deleted
(also in constant time).


\section{\mbox{\texttt{SEList}}: A Space-Efficient Linked List}
\seclabel{selist}

One of the drawbacks of linked lists (besides the time it takes to
access elements that are deep within the list) is their space usage.
Each node in a \mbox{\texttt{DLList}} requires an additional 2 references to the next
and previous node in the list.  Two thirds of the fields in a \mbox{\texttt{Node}}
are dedicated to maintaining the list and only one third of the fields
are for storing data!

A \mbox{\texttt{SEList}} (space-efficient list) reduces this wasted space using
a simple idea: Rather than store individual elements in a \mbox{\texttt{DLList}},
we store a block (array) containing several items. More precisely, an
\mbox{\texttt{SEList}} is parameterized by a \emph{block size} \mbox{\texttt{{\color{var}b}}}. Each individual
node in an \mbox{\texttt{SEList}} stores a block that can hold up to \mbox{\texttt{{\color{var}b}+1}} elements.

It will turn out, for reasons that become clear later, that it will
be helpful if we can do \mbox{\texttt{Deque}} operations on each block.  The data
structure we choose for this is an \mbox{\texttt{BDeque}} (bounded deque), derived
from the \mbox{\texttt{ArrayDeque}} structure described in \secref{arraydeque}.
The \mbox{\texttt{BDeque}} differs from the \mbox{\texttt{ArrayDeque}} in one small way: When a
new \mbox{\texttt{BDeque}} is created, the size of the backing array \mbox{\texttt{{\color{var}a}}}
is fixed at \mbox{\texttt{{\color{var}b}+1}} and it never grows or shrinks.
The important property of a \mbox{\texttt{BDeque}} is that it allows for addition or
removal of elements at either the front or back in constant time. This
will be useful as elements are shifted from one block to another.


\begin{Verbatim}[tabsize=2,frame=single,commandchars=\\@\$,label=\texttt{SEList},labelposition=topline]
	@\color@keyword$class$ BDeque @\color@keyword$extends$ ArrayDeque<@\color@keyword$T$> {
		BDeque() {
			super(SEList.@\color@var$this$.type());
			@\color@var$a$ = newArray(@\color@var$b$+1);
		}
		@\color@var$void$ grow() { }
		@\color@var$void$ shrink() { }
	}
\end{Verbatim}

An \mbox{\texttt{SEList}} is then a doubly-linked list of blocks:

\begin{Verbatim}[tabsize=2,frame=single,commandchars=\\@\$,label=\texttt{SEList},labelposition=topline]
	@\color@keyword$class$ Node {
		BDeque @\color@var$d$;
		Node @\color@var$prev$, @\color@var$next$;
	}
\end{Verbatim}
\begin{Verbatim}[tabsize=2,frame=single,commandchars=\\@\$,label=\texttt{SEList},labelposition=topline]
	@\color@keyword$int$ @\color@var$n$;
	Node @\color@var$dummy$;
\end{Verbatim}

\subsection{Space Requirements}

An \mbox{\texttt{SEList}} places very tight restrictions on the number of elements
in a block: Unless a block is the last block, then that block contains
at least $\mbox{\texttt{{\color{var}b}}}-1$ and at most $\mbox{\texttt{{\color{var}b}}}+1$ elements.  This means that, if a
\mbox{\texttt{SEList}} contains \mbox{\texttt{{\color{var}n}}} elements, then it has at most
\[
    \mbox{\texttt{{\color{var}n}}}/(\mbox{\texttt{{\color{var}b}}}-1) + 1 = O(\mbox{\texttt{{\color{var}n}}}/\mbox{\texttt{{\color{var}b}}})
\]
blocks.  The \mbox{\texttt{BDeque}} for each block contains an array of length $\mbox{\texttt{{\color{var}b}}}+1$,
but otherwise, the amount of memory used by a block is constant.
This means that the wasted space in an \mbox{\texttt{SEList}} is only $O(\mbox{\texttt{{\color{var}n}}}/\mbox{\texttt{{\color{var}b}}})$.
By choosing a large value of \mbox{\texttt{{\color{var}b}}} we can make the space-overhead of an
SEList negligible compared to the amount of memory required to store
the \mbox{\texttt{{\color{var}n}}} items in the \mbox{\texttt{SEList}}.

\subsection{Finding Elements}

The first challenge we face with an \mbox{\texttt{SEList}} is finding the list item
with a given index \mbox{\texttt{{\color{var}i}}}.  Note that the location of an element consists
of two parts: The node \mbox{\texttt{{\color{var}u}}} that contains the block that contains the
element as well as the index \mbox{\texttt{{\color{var}j}}} of the element within its block:

\begin{Verbatim}[tabsize=2,frame=single,commandchars=\\@\$,label=\texttt{SEList},labelposition=topline]
	@\color@keyword$class$ Location {
		Node @\color@var$u$;
		@\color@keyword$int$ @\color@var$j$;
		Location(Node @\color@var$u$, @\color@keyword$int$ @\color@var$j$) {
			@\color@var$this$.@\color@var$u$ = @\color@var$u$;
			@\color@var$this$.@\color@var$j$ = @\color@var$j$;
		}
	}
\end{Verbatim}

To find the block that contains a particular element, we proceed in the
same way as in a \mbox{\texttt{DLList}}.  We either start at the front of the list and
traverse in the forward direction or at the back of the list and traverse
backwards until we reach the node we want.  The only difference is that,
each time we move from one node to the next, we skip over a whole block
of elements.

\begin{Verbatim}[tabsize=2,frame=single,commandchars=\\@\$,label=\texttt{SEList},labelposition=topline]
	Location getLocation(@\color@keyword$int$ @\color@var$i$) {
		@\color@keyword$if$ (@\color@var$i$ < @\color@var$n$/2) {
			Node @\color@var$u$ = @\color@var$dummy$.@\color@var$next$;
			@\color@keyword$while$ (@\color@var$i$ >= @\color@var$u$.@\color@var$d$.size()) {
				@\color@var$i$ -= @\color@var$u$.@\color@var$d$.size();
				@\color@var$u$ = @\color@var$u$.@\color@var$next$;
			}
			@\color@var$return$ @\color@keyword$new$ Location(@\color@var$u$, @\color@var$i$);
		} @\color@keyword$else$ {
			Node @\color@var$u$ = @\color@var$dummy$;
			@\color@keyword$int$ @\color@var$idx$ = @\color@var$n$;
			@\color@keyword$while$ (@\color@var$i$ < @\color@var$idx$) {
				@\color@var$u$ = @\color@var$u$.@\color@var$prev$;
				@\color@var$idx$ -= @\color@var$u$.@\color@var$d$.size();
			}
			@\color@var$return$ @\color@keyword$new$ Location(@\color@var$u$, @\color@var$i$-@\color@var$idx$);
		}
	}
\end{Verbatim}

Remember that, with the exception of at most one block, each block
contains at least $\mbox{\texttt{{\color{var}b}}}-1$ elements, so each step in our search gets
us $\mbox{\texttt{{\color{var}b}}}-1$ elements closer to the element we are looking for.  If we
are searching forward, this means we reach the node we want after
$O(1+\mbox{\texttt{{\color{var}i}}}/\mbox{\texttt{{\color{var}b}}})$ steps.  If we search backwards, we reach the node we want
after $O(1+(\mbox{\texttt{{\color{var}n}}}-\mbox{\texttt{{\color{var}i}}})/\mbox{\texttt{{\color{var}b}}})$ steps.  The algorithm takes the smaller of
these two quantities depending on the value of \mbox{\texttt{{\color{var}i}}}, so the time to locate
the item with index \mbox{\texttt{{\color{var}i}}} is $O(1+\min\{\mbox{\texttt{{\color{var}i}}},\mbox{\texttt{{\color{var}n}}}-\mbox{\texttt{{\color{var}i}}}\}/\mbox{\texttt{{\color{var}b}}})$.

Once we know how to locate the item with index \mbox{\texttt{{\color{var}i}}}, the \mbox{\texttt{get({\color{var}i})}} and
\mbox{\texttt{set({\color{var}i},{\color{var}x})}} operations translate into getting or setting a particular
index in the correct block:

\begin{Verbatim}[tabsize=2,frame=single,commandchars=\\@\$,label=\texttt{SEList},labelposition=topline]
	@\color@keyword$T$ get(@\color@keyword$int$ @\color@var$i$) {
		@\color@keyword$if$ (@\color@var$i$ < 0 || @\color@var$i$ > @\color@var$n$ - 1) @\color@keyword$throw$ @\color@keyword$new$ IndexOutOfBoundsException();
		Location @\color@var$l$ = getLocation(@\color@var$i$);
		@\color@var$return$ @\color@var$l$.@\color@var$u$.@\color@var$d$.get(@\color@var$l$.@\color@var$j$);
	}
	@\color@keyword$T$ set(@\color@keyword$int$ @\color@var$i$, @\color@keyword$T$ @\color@var$x$) {
		@\color@keyword$if$ (@\color@var$i$ < 0 || @\color@var$i$ > @\color@var$n$ - 1) @\color@keyword$throw$ @\color@keyword$new$ IndexOutOfBoundsException();
		Location @\color@var$l$ = getLocation(@\color@var$i$);
		@\color@keyword$T$ @\color@var$y$ = @\color@var$l$.@\color@var$u$.@\color@var$d$.get(@\color@var$l$.@\color@var$j$);
		@\color@var$l$.@\color@var$u$.@\color@var$d$.set(@\color@var$l$.@\color@var$j$,@\color@var$x$);
		@\color@var$return$ @\color@var$y$;
	}
\end{Verbatim}

These operations are dominated by the time it takes to locate the item,
so they also run in $O(1+\min\{\mbox{\texttt{{\color{var}i}}},\mbox{\texttt{{\color{var}n}}}-\mbox{\texttt{{\color{var}i}}}\}/\mbox{\texttt{{\color{var}b}}})$ time.

\subsection{Adding an Element}

Where things start to get complicated is when adding elements.  Before
considering the general case, we first consider the easier operation,
\mbox{\texttt{add({\color{var}x})}}, in which \mbox{\texttt{{\color{var}x}}} is added to the end of the list.  In this case,
\mbox{\texttt{{\color{var}x}}} will go at the end of the last block.  If the last block is full
(or does not exist because there are no blocks yet), then we allocate a
new block and append it the list of blocks.  Now that we are sure that
the last block exists and is not full, we append \mbox{\texttt{{\color{var}x}}} to the last block.

\begin{Verbatim}[tabsize=2,frame=single,commandchars=\\@\$,label=\texttt{SEList},labelposition=topline]
	@\color@var$boolean$ add(@\color@keyword$T$ @\color@var$x$) {
		Node @\color@var$last$ = @\color@var$dummy$.@\color@var$prev$;
		@\color@keyword$if$ (@\color@var$last$ == @\color@var$dummy$ || @\color@var$last$.@\color@var$d$.size() == @\color@var$b$+1) {
			@\color@var$last$ = addBefore(@\color@var$dummy$);
		}
		@\color@var$last$.@\color@var$d$.add(@\color@var$x$);
		@\color@var$n$++;
		@\color@var$return$ @\color@var$true$;
	}
\end{Verbatim}

Things get more complicated when we add in the interior of the list
using \mbox{\texttt{add({\color{var}i},{\color{var}x})}}.  We first locate \mbox{\texttt{{\color{var}i}}} to get the node \mbox{\texttt{{\color{var}u}}} whose block
contains the \mbox{\texttt{{\color{var}i}}}th list item.  The problem is that we want to insert \mbox{\texttt{{\color{var}x}}}
into \mbox{\texttt{{\color{var}u}}}'s block, but we have to be prepared for the case where \mbox{\texttt{{\color{var}u}}}'s
block already contains $\mbox{\texttt{{\color{var}b}}}+1$ elements, so that it is full and there
is no room for \mbox{\texttt{{\color{var}x}}}.

Let $\mbox{\texttt{{\color{var}u}}}_0,\mbox{\texttt{{\color{var}u}}}_1,\mbox{\texttt{{\color{var}u}}}_2,\ldots$ denote \mbox{\texttt{{\color{var}u}}}, \mbox{\texttt{{\color{var}u}.{\color{var}next}}}, \mbox{\texttt{{\color{var}u}.{\color{var}next}.{\color{var}next}}},
and so on.  We explore in $\mbox{\texttt{{\color{var}u}}}_0,\mbox{\texttt{{\color{var}u}}}_1,\mbox{\texttt{{\color{var}u}}}_2,\ldots$ looking for a node
that can provide space for \mbox{\texttt{{\color{var}x}}}.  Three cases can occur during our space
exploration (see \figref{selist-add}): 

\begin{figure}
  \noindent
  \begin{center}
    \begin{tabular}{l}
      \includegraphics{figs/selist-add-a}\\[4ex]
      \includegraphics{figs/selist-add-b}\\[4ex]
      \includegraphics{figs/selist-add-c}\\
    \end{tabular}
  \end{center}
  \caption{The three cases that occur during the addition of an item \mbox{\texttt{{\color{var}x}}} in the interior of an \mbox{\texttt{SEList}}.  (This \mbox{\texttt{SEList}} has block size $\mbox{\texttt{{\color{var}b}}}=3$.)}
  \figlabel{selist-add}
\end{figure}


\begin{enumerate}
\item We quickly (in $r < \mbox{\texttt{{\color{var}b}}}$ steps) find a node $\mbox{\texttt{{\color{var}u}}}_r$ whose block
is not full.  In this case, we perform $r$ shifts of an element from
one block into the next, so that the free space in $\mbox{\texttt{{\color{var}u}}}_r$ becomes a
free space in $\mbox{\texttt{{\color{var}u}}}_0$.  We can then insert \mbox{\texttt{{\color{var}x}}} into $\mbox{\texttt{{\color{var}u}}}_0$'s block.

\item We quickly (in $r<\mbox{\texttt{{\color{var}b}}}$ steps) run off the end of the list
of blocks.  In this case, we add a new empty block to the end of the
list of blocks and proceed as in the first case.

\item After \mbox{\texttt{{\color{var}b}}} steps we do not find any block that is not full.
In this case, $\mbox{\texttt{{\color{var}u}}}_0,\ldots,\mbox{\texttt{{\color{var}u}}}_{\mbox{\texttt{{\color{var}b}}}-1}$ is a sequence of \mbox{\texttt{{\color{var}b}}} blocks
that each contain $\mbox{\texttt{{\color{var}b}}}+1$ elements.  We insert a new block $\mbox{\texttt{{\color{var}u}}}_{\mbox{\texttt{{\color{var}b}}}}$
at the end of this sequence and \emph{spread} the original $\mbox{\texttt{{\color{var}b}}}(\mbox{\texttt{{\color{var}b}}}+1)$
elements so that each block of $\mbox{\texttt{{\color{var}u}}}_0,\ldots,\mbox{\texttt{{\color{var}u}}}_{\mbox{\texttt{{\color{var}b}}}}$ contains exactly
\mbox{\texttt{{\color{var}b}}} elements.  Now $\mbox{\texttt{{\color{var}u}}}_0$'s block contains only \mbox{\texttt{{\color{var}b}}} elements so it has
room for us to insert \mbox{\texttt{{\color{var}x}}}.
\end{enumerate}

\begin{Verbatim}[tabsize=2,frame=single,commandchars=\\@\$,label=\texttt{SEList},labelposition=topline]
	@\color@var$void$ add(@\color@keyword$int$ @\color@var$i$, @\color@keyword$T$ @\color@var$x$) {
		@\color@keyword$if$ (@\color@var$i$ < 0 || @\color@var$i$ > @\color@var$n$) @\color@keyword$throw$ @\color@keyword$new$ IndexOutOfBoundsException();
		@\color@keyword$if$ (@\color@var$i$ == @\color@var$n$) {
			add(@\color@var$x$);
			@\color@var$return$;
		}
		Location @\color@var$l$ = getLocation(@\color@var$i$);
		Node @\color@var$u$ = @\color@var$l$.@\color@var$u$;
		@\color@keyword$int$ @\color@var$r$ = 0;
		@\color@keyword$while$ (@\color@var$r$ < @\color@var$b$ && @\color@var$u$ != @\color@var$dummy$ && @\color@var$u$.@\color@var$d$.size() == @\color@var$b$+1) {
			@\color@var$u$ = @\color@var$u$.@\color@var$next$;
			@\color@var$r$++;
		}
		@\color@keyword$if$ (@\color@var$r$ == @\color@var$b$) {      @\color@comment$// found b blocks each with b+1 elements$
			spread(@\color@var$l$.@\color@var$u$);
			@\color@var$u$ = @\color@var$l$.@\color@var$u$;
		} 
		@\color@keyword$if$ (@\color@var$u$ == @\color@var$dummy$) {  @\color@comment$// ran off the end of the list - add new node$
			@\color@var$u$ = addBefore(@\color@var$u$);
		}
		@\color@keyword$while$ (@\color@var$u$ != @\color@var$l$.@\color@var$u$) { @\color@comment$// work backwards, shifting an element at each step $
			@\color@var$u$.@\color@var$d$.add(0, @\color@var$u$.@\color@var$prev$.@\color@var$d$.remove(@\color@var$u$.@\color@var$prev$.@\color@var$d$.size()-1));
			@\color@var$u$ = @\color@var$u$.@\color@var$prev$;
		}
		@\color@var$u$.@\color@var$d$.add(@\color@var$l$.@\color@var$j$, @\color@var$x$);
		@\color@var$n$++;
	}
\end{Verbatim}

The running time of the \mbox{\texttt{add({\color{var}i},{\color{var}x})}} operation depends on which of
the three cases above occurs.  Cases~1 and 2 involve examining and
shifting elements through at most \mbox{\texttt{{\color{var}b}}} blocks and take $O(\mbox{\texttt{{\color{var}b}}})$ time.
Case~3 involves calling the \mbox{\texttt{spread({\color{var}u})}} method, which  moves $\mbox{\texttt{{\color{var}b}}}(\mbox{\texttt{{\color{var}b}}}+1)$
elements and takes $O(\mbox{\texttt{{\color{var}b}}}^2)$ time.  If we ignore the cost of Case~3
(which we will account for later using amortization) this means that
the total running time to locate \mbox{\texttt{{\color{var}i}}} and perform the insertion of \mbox{\texttt{{\color{var}x}}}
is $O(\mbox{\texttt{{\color{var}b}}}+\min\{\mbox{\texttt{{\color{var}i}}},\mbox{\texttt{{\color{var}n}}}-\mbox{\texttt{{\color{var}i}}}\}/\mbox{\texttt{{\color{var}b}}})$.

\subsection{Removing an Element}

Removing an element, using the \mbox{\texttt{remove({\color{var}i})}} method from an \mbox{\texttt{SEList}}
is similar to adding an element.  We first locate the node \mbox{\texttt{{\color{var}u}}} that
contains the element with index \mbox{\texttt{{\color{var}i}}}. Now, we have to be prepared for
the case that we cannot remove an element from \mbox{\texttt{{\color{var}u}}} without causing \mbox{\texttt{{\color{var}u}}}'s
block to have size less than $\mbox{\texttt{{\color{var}b}}}-1$, which is not allowed.

Again, let $\mbox{\texttt{{\color{var}u}}}_0,\mbox{\texttt{{\color{var}u}}}_1,\mbox{\texttt{{\color{var}u}}}_2,\ldots$ denote \mbox{\texttt{{\color{var}u}}}, \mbox{\texttt{{\color{var}u}.{\color{var}next}}}, \mbox{\texttt{{\color{var}u}.{\color{var}next}.{\color{var}next}}},
We examine $\mbox{\texttt{{\color{var}u}}}_0,\mbox{\texttt{{\color{var}u}}}_1,\mbox{\texttt{{\color{var}u}}}_2,\ldots$ in order looking for a node from
which we can borrow an element to make the size of $\mbox{\texttt{{\color{var}u}}}_0$'s block larger
than $\mbox{\texttt{{\color{var}b}}}-1$.  There are three cases to consider
(see \figref{selist-remove}): 

\begin{figure}
  \noindent
  \begin{center}
    \begin{tabular}{l}
      \includegraphics{figs/selist-remove-a}\\[4ex]
      \includegraphics{figs/selist-remove-b}\\[4ex]
      \includegraphics{figs/selist-remove-c}\\
    \end{tabular}
  \end{center}
  \caption{The three cases that occur during the removeition of an item \mbox{\texttt{{\color{var}x}}} in the interior of an \mbox{\texttt{SEList}}.  (This \mbox{\texttt{SEList}} has block size $\mbox{\texttt{{\color{var}b}}}=3$.)}
  \figlabel{selist-remove}
\end{figure}


\begin{enumerate}
\item We quickly (in $r < \mbox{\texttt{{\color{var}b}}}$ steps) find a node whose block contains
more than $\mbox{\texttt{{\color{var}b}}}-1$ elements. In this case, we perform $r$ shifts of an
element from one block into the previous, so that the extra element in
$\mbox{\texttt{{\color{var}u}}}_r$ becomes a an extra element in $\mbox{\texttt{{\color{var}u}}}_0$.  We can then remove the
appropriate element from $\mbox{\texttt{{\color{var}u}}}_0$'s block.

\item We quickly (in $r<\mbox{\texttt{{\color{var}b}}}$ steps) run off the end of the list of blocks.
In this case, $\mbox{\texttt{{\color{var}u}}}_r$ is the last block, and there is no requirement that
$\mbox{\texttt{{\color{var}u}}}_r$'s block contain at least $\mbox{\texttt{{\color{var}b}}}-1$ elements.  Therefore, we proceed
as above, borrowing an element from $\mbox{\texttt{{\color{var}u}}}_r$ to make an extra element in
$\mbox{\texttt{{\color{var}u}}}_0$.  If this causes $\mbox{\texttt{{\color{var}u}}}_r$ block to become empty, then we remove it.

\item After \mbox{\texttt{{\color{var}b}}} steps we do not find any block containing more than
$\mbox{\texttt{{\color{var}b}}}-1$ elements.  In this case, $\mbox{\texttt{{\color{var}u}}}_0,\ldots,\mbox{\texttt{{\color{var}u}}}_{\mbox{\texttt{{\color{var}b}}}-1}$ is a sequence
of \mbox{\texttt{{\color{var}b}}} blocks that each contain $\mbox{\texttt{{\color{var}b}}}-1$ elements.  We \emph{gather}
these $\mbox{\texttt{{\color{var}b}}}(\mbox{\texttt{{\color{var}b}}}-1)$ elements into $\mbox{\texttt{{\color{var}u}}}_0,\ldots,\mbox{\texttt{{\color{var}u}}}_{\mbox{\texttt{{\color{var}b}}}-2}$ so that each
of these $\mbox{\texttt{{\color{var}b}}}-1$ blocks contains exactly \mbox{\texttt{{\color{var}b}}} elements and we remove
$\mbox{\texttt{{\color{var}u}}}_{\mbox{\texttt{{\color{var}b}}}-1}$, which is now empty.  Now $\mbox{\texttt{{\color{var}u}}}_0$'s block contains \mbox{\texttt{{\color{var}b}}}
elements so we can remove the appropriate element from it.
\end{enumerate}

\begin{Verbatim}[tabsize=2,frame=single,commandchars=\\@\$,label=\texttt{SEList},labelposition=topline]
	@\color@keyword$T$ remove(@\color@keyword$int$ @\color@var$i$) {
		@\color@keyword$if$ (@\color@var$i$ < 0 || @\color@var$i$ > @\color@var$n$ - 1) @\color@keyword$throw$ @\color@keyword$new$ IndexOutOfBoundsException();
		Location @\color@var$l$ = getLocation(@\color@var$i$);
		@\color@keyword$T$ @\color@var$y$ = @\color@var$l$.@\color@var$u$.@\color@var$d$.get(@\color@var$l$.@\color@var$j$);
		Node @\color@var$u$ = @\color@var$l$.@\color@var$u$;
		@\color@keyword$int$ @\color@var$r$ = 0;
		@\color@keyword$while$ (@\color@var$r$ < @\color@var$b$ && @\color@var$u$ != @\color@var$dummy$ && @\color@var$u$.@\color@var$d$.size() == @\color@var$b$-1) {
			@\color@var$u$ = @\color@var$u$.@\color@var$next$;
			@\color@var$r$++;
		}
		@\color@keyword$if$ (@\color@var$r$ == @\color@var$b$) {  @\color@comment$// found b blocks each with b-1 elements$
			gather(@\color@var$l$.@\color@var$u$);
		}
		@\color@var$u$ = @\color@var$l$.@\color@var$u$;
		@\color@var$u$.@\color@var$d$.remove(@\color@var$l$.@\color@var$j$);
		@\color@keyword$while$ (@\color@var$u$.@\color@var$d$.size() < @\color@var$b$-1 && @\color@var$u$.@\color@var$next$ != @\color@var$dummy$ && !@\color@var$u$.@\color@var$next$.@\color@var$d$.isEmpty()) {
			@\color@var$u$.@\color@var$d$.add(@\color@var$u$.@\color@var$next$.@\color@var$d$.remove(0));
			@\color@var$u$ = @\color@var$u$.@\color@var$next$; @\color@comment$// TODO: careful about end$
		}
		@\color@keyword$if$ (@\color@var$u$.@\color@var$next$ != @\color@var$dummy$ && @\color@var$u$.@\color@var$next$.@\color@var$d$.isEmpty()) remove(@\color@var$u$.@\color@var$next$);
		@\color@var$n$--;
		@\color@var$return$ @\color@var$y$;
	}
\end{Verbatim}

Like the \mbox{\texttt{add({\color{var}i},{\color{var}x})}} operation, the running time of the \mbox{\texttt{remove({\color{var}i})}}
operation is $O(\mbox{\texttt{{\color{var}b}}}+\min\{\mbox{\texttt{{\color{var}i}}},\mbox{\texttt{{\color{var}n}}}-\mbox{\texttt{{\color{var}i}}}\}/\mbox{\texttt{{\color{var}b}}})$ if we ignore the cost of
the \mbox{\texttt{gather({\color{var}u})}} method that occurs in Case~3.

\subsection{Amortized Analysis of Spreading and Gathering}

Next, we consider the cost of the \mbox{\texttt{gather({\color{var}u})}} and \mbox{\texttt{spread({\color{var}u})}} methods that may be executed by the \mbox{\texttt{add({\color{var}i},{\color{var}x})}} and \mbox{\texttt{remove({\color{var}i})}} methods.  For completeness, here they are:

\begin{Verbatim}[tabsize=2,frame=single,commandchars=\\@\$,label=\texttt{SEList},labelposition=topline]
	@\color@var$void$ spread(Node @\color@var$u$) {
		Node @\color@var$w$ = @\color@var$u$;
		@\color@keyword$for$ (@\color@keyword$int$ @\color@var$j$ = 0; @\color@var$j$ < @\color@var$b$; @\color@var$j$++) {
			@\color@var$w$ = @\color@var$w$.@\color@var$next$;
		}
		@\color@var$w$ = addBefore(@\color@var$w$);
		@\color@keyword$while$ (@\color@var$w$ != @\color@var$u$) {
			@\color@keyword$while$ (@\color@var$w$.@\color@var$d$.size() < @\color@var$b$)
				@\color@var$w$.@\color@var$d$.add(0,@\color@var$w$.@\color@var$prev$.@\color@var$d$.remove(@\color@var$w$.@\color@var$prev$.@\color@var$d$.size()-1));
			@\color@var$w$ = @\color@var$w$.@\color@var$prev$;
		}
	}
\end{Verbatim}
\begin{Verbatim}[tabsize=2,frame=single,commandchars=\\@\$,label=\texttt{SEList},labelposition=topline]
	@\color@var$void$ gather(Node @\color@var$u$) {
		Node @\color@var$w$ = @\color@var$u$;
		@\color@keyword$for$ (@\color@keyword$int$ @\color@var$j$ = 0; @\color@var$j$ < @\color@var$b$-1; @\color@var$j$++) {
			@\color@keyword$while$ (@\color@var$w$.@\color@var$d$.size() < @\color@var$b$)
				@\color@var$w$.@\color@var$d$.add(@\color@var$w$.@\color@var$next$.@\color@var$d$.remove(0));
			@\color@var$w$ = @\color@var$w$.@\color@var$next$;
		}
		remove(@\color@var$w$);
	}
\end{Verbatim}

The running time of each of these methods is dominated by the two
nested loops.  Both the inner loop and outer loop execute at most
$\mbox{\texttt{{\color{var}b}}}+1$ times, so the total running time of each of these methods
is $O((\mbox{\texttt{{\color{var}b}}}+1)^2)=O(\mbox{\texttt{{\color{var}b}}}^2)$. However, the following lemma shows that
these methods only execute on one out of every \mbox{\texttt{{\color{var}b}}} calls to \mbox{\texttt{add({\color{var}i},{\color{var}x})}}
or \mbox{\texttt{remove({\color{var}i})}}.

\begin{lem}\lemlabel{selist-amortized}
  If an empty \mbox{\texttt{SEList}} is created and any sequence of $m\ge 1$ calls
  to \mbox{\texttt{add({\color{var}i},{\color{var}x})}} and \mbox{\texttt{remove({\color{var}i})}} are performed, then the total cost of all work
  done during all calls to \mbox{\texttt{spread()}} and \mbox{\texttt{gather()}} is $O(\mbox{\texttt{{\color{var}b}}}m)$.
\end{lem}

\begin{proof}
  We will use the potential method of amortized analysis.  We say that
  a node \mbox{\texttt{{\color{var}u}}} is \emph{fragile} if \mbox{\texttt{{\color{var}u}}}'s block does not contain \mbox{\texttt{{\color{var}b}}}
  elements (so that \mbox{\texttt{{\color{var}u}}} is either the last node, or contains $\mbox{\texttt{{\color{var}b}}}-1$
  or $\mbox{\texttt{{\color{var}b}}}+1$ elements).  Any node whose block contains \mbox{\texttt{{\color{var}b}}} elements is
  \emph{strong}. Define the \emph{potential} of an \mbox{\texttt{SEList}} as the number
  of fragile nodes it contains.  We will consider only the \mbox{\texttt{add({\color{var}i},{\color{var}x})}}
  operation and its relation to the number of calls to \mbox{\texttt{spread({\color{var}u})}}.
  The analysis of \mbox{\texttt{remove({\color{var}i})}} and \mbox{\texttt{gather({\color{var}u})}} is identical.

  Notice that, if Case~1 occurs during the \mbox{\texttt{add({\color{var}i},{\color{var}x})}} method, then
  only one node, $\mbox{\texttt{{\color{var}u}}}_r$ has the size of its block changed. Therefore,
  at most one node, namely $\mbox{\texttt{{\color{var}u}}}_r$, goes from being strong to being
  fragile.  If Case~2 occurs, then a new node is created, and this node
  is fragile, but no other node changes sizes, so the number of fragile
  nodes increases by one.  Thus, in either Case~1 or Case~2 the potential
  of the SEList increases by at most 1.

  Finally, if Case~3 occurs, it is because $\mbox{\texttt{{\color{var}u}}}_0,\ldots,\mbox{\texttt{{\color{var}u}}}_{\mbox{\texttt{{\color{var}b}}}-1}$
  are all fragile nodes.  Then $\mbox{\texttt{spread(}}u_0\mbox{\texttt{)}}$ is called and these \mbox{\texttt{{\color{var}b}}}
  fragile nodes are replace with $\mbox{\texttt{{\color{var}b}}}+1$ strong nodes.  Finally, \mbox{\texttt{{\color{var}x}}}
  is added to $\mbox{\texttt{{\color{var}u}}}_0$'s block, making $\mbox{\texttt{{\color{var}u}}}_0$ fragile.  In total the
  potential decreases by $\mbox{\texttt{{\color{var}b}}}-1$.

  In summary, the potential starts at 0 (there are no nodes in the list).
  Each time Case~1 or Case~2 occurs, the potential increases by at
  most 1.  Each time Case~3 occurs, the potential decreases by $\mbox{\texttt{{\color{var}b}}}-1$.
  The potential, which counts the number of fragile nodes) is never
  less than 0.  We conclude that, for every occurence of Case~3, there
  are at least $\mbox{\texttt{{\color{var}b}}}-1$ occurences of Case~1 or Case~2.  Thus, for every
  call to \mbox{\texttt{spread({\color{var}u})}} there are at least \mbox{\texttt{{\color{var}b}}} calls to \mbox{\texttt{add({\color{var}i},{\color{var}x})}}.  This
  completes the proof.
\end{proof}

\subsection{Summary}

The following theorem summarizes the performance of the \mbox{\texttt{SEList}} data
structure:

\begin{thm}\thmlabel{selist}
  An \mbox{\texttt{SEList}} implements the \mbox{\texttt{List}} interface.  Ignoring the cost of
  calls to \mbox{\texttt{spread({\color{var}u})}} and \mbox{\texttt{gather({\color{var}u})}}, an \mbox{\texttt{SEList}} with block size \mbox{\texttt{{\color{var}b}}}
  supports the operations
  \begin{itemize}
    \item \mbox{\texttt{get({\color{var}i})}} and \mbox{\texttt{set({\color{var}i},{\color{var}x})}} in $O(1+\min\{\mbox{\texttt{{\color{var}i}}},\mbox{\texttt{{\color{var}n}}}-\mbox{\texttt{{\color{var}i}}}\}/\mbox{\texttt{{\color{var}b}}})$ time per operation; and
    \item \mbox{\texttt{add({\color{var}i},{\color{var}x})}} and \mbox{\texttt{remove({\color{var}i})}} in $O(b+\min\{\mbox{\texttt{{\color{var}i}}},\mbox{\texttt{{\color{var}n}}}-\mbox{\texttt{{\color{var}i}}}\}/\mbox{\texttt{{\color{var}b}}})$ time per operation.
  \end{itemize}
  Furthermore, beginning with an empty \mbox{\texttt{SEList}}, any sequence of $m$
  \mbox{\texttt{add({\color{var}i},{\color{var}x})}} and \mbox{\texttt{remove({\color{var}i})}} operations results in a total of $O(\mbox{\texttt{{\color{var}b}}}m)$
  work during all calls to \mbox{\texttt{spread({\color{var}u})}} and \mbox{\texttt{gather({\color{var}u})}}.

  The space (measured in words)\footnote{Recall \secref{model} for a
  discussion of how memory is measured.} used by an \mbox{\texttt{SEList}}
  that stores \mbox{\texttt{{\color{var}n}}} elements is $\mbox{\texttt{{\color{var}n}}} +O(\mbox{\texttt{{\color{var}n}}}/b)$.
\end{thm}

The \mbox{\texttt{SEList}} is a tradeoff between an \mbox{\texttt{ArrayList}} and a \mbox{\texttt{DLList}} where
the relative mix of these two structures depends on the block size \mbox{\texttt{{\color{var}b}}}.
At the extreme $\mbox{\texttt{{\color{var}b}}}=2$, each \mbox{\texttt{SEList}} node stores at most 3 values,
and is not really much different than a \mbox{\texttt{DLList}}. At the other extreme,
$\mbox{\texttt{{\color{var}b}}}>\mbox{\texttt{{\color{var}n}}}$, in which case all the elements are stored in a single array,
just like an \mbox{\texttt{ArrayList}}.  In between these two extremes is a tradeoff
between the time it takes to add or remove a list item and the time it
takes to locate a particular list item.

\section{Exercises}

\begin{enumerate}
\item Why is it not possible, in a \mbox{\texttt{SLList}} to use a dummy node to avoid
  all the special cases that occur in the operations \mbox{\texttt{push({\color{var}x})}}, \mbox{\texttt{pop()}},
  \mbox{\texttt{add({\color{var}x})}}, and \mbox{\texttt{remove()}}?

\item Describe and implement the \mbox{\texttt{List}} operations \mbox{\texttt{get({\color{var}i})}}, \mbox{\texttt{set({\color{var}i},{\color{var}x})}},
  \mbox{\texttt{add({\color{var}i},{\color{var}x})}} and \mbox{\texttt{remove({\color{var}i})}} on an \mbox{\texttt{SLList}}.  Each of these operations
  should run in $O(1+\mbox{\texttt{{\color{var}i}}})$ time.

\item Describe and implement the \mbox{\texttt{List}} operations \mbox{\texttt{get({\color{var}i})}}, \mbox{\texttt{set({\color{var}i},{\color{var}x})}},
  \mbox{\texttt{add({\color{var}i},{\color{var}x})}} and \mbox{\texttt{remove({\color{var}i})}} on an \mbox{\texttt{SLList}}.  Each of these operations
  should run in $O(1+\mbox{\texttt{{\color{var}i}}})$ time.

\end{enumerate}


\chapter{Skiplists}
\chaplabel{skiplists}

In this chapter we discuss a beautiful data structure, the skiplist,
that has a variety of applications.  Using a skiplist we can implement
a \mbox{\texttt{List}} that is fast for all the operations \mbox{\texttt{get({\color{var}i})}}, \mbox{\texttt{set({\color{var}i},{\color{var}x})}},
\mbox{\texttt{add({\color{var}i},{\color{var}x})}}, and \mbox{\texttt{remove({\color{var}i})}}. We can also implement an \mbox{\texttt{SSet}} in which
all operations run in $O(\log n)$ time.  Finally, a skiplist can be used
to implement a \mbox{\texttt{Rope}} in which all operations run in $O(\log n)$ time.

Skiplists rely on \emph{randomization} for their efficiency.  In
particular, a skiplist uses coin tosses when an element is inserted\ldots

\section{The Basic Structure}

Conceptually, a skiplist is a sequence of singly-linked lists
$L_0,\ldots,L_h$, where each $L_r$ contains a subset of the items
in $L_{r-1}$.  We start with the input list $L_0$ that contains \mbox{\texttt{{\color{var}n}}}
items and construct $L_1$ from $L_0$, $L_2$ from $L_1$, and so on.
The items in $L_r$ are obtained by tossing a coin for each element \mbox{\texttt{{\color{var}x}}}
in $L_{r-1}$ and including \mbox{\texttt{{\color{var}x}}} in $L_r$ if the coin comes up heads.
This process ends when we create a list $L_r$ that is empty.  An example
of a skiplist is shown in \figref{skiplist}.

\begin{figure}
  \begin{center}
    \includegraphics{figs/skiplist}
  \end{center}
  \caption{A skiplist containing 6 elements.}
  \figlabel{skiplist}
\end{figure}

For an element \mbox{\texttt{{\color{var}x}}} in a skiplist, we call the \emph{height} of \mbox{\texttt{{\color{var}x}}} the
largest value $r$ such that \mbox{\texttt{{\color{var}x}}} appears in $L_r$.  Thus, for example,
elements that only appear in $L_0$ have height $0$.  Notice that the
height of \mbox{\texttt{{\color{var}x}}} corresponds to the following experiment:  Toss a coin
repeatedly until the first time it comes up tails.  How many times did
it come up heads?  The answer, not surprisingly, is that the expected
height of a node is 1.  The \emph{height} of a skiplist is the maximum
height of any of its nodes.

At the head of every list is a special node, called the \emph{sentinel}
that acts as dummy node for each list. The key property of skiplists
is that there is a short path, called the \emph{search path}, from the
sentinel in $L_h$ to every node in $L_0$.  Remembering how to construct
a search path for a node \mbox{\texttt{{\color{var}u}}} is easy (see \figref{skiplist-searchpath})
:  Start at the top left corner of your skiplist (the sentinel in $L_h$)
and always go right unless that would overshoot \mbox{\texttt{{\color{var}u}}}, in which case you
should take a step down into the list below.

More precisely, to construct the search path for the node \mbox{\texttt{{\color{var}u}}} in $L_0$
we start at the sentinel \mbox{\texttt{{\color{var}w}}} in $L_h$.  Next we examine \mbox{\texttt{{\color{var}w}.{\color{var}next}}}.
If \mbox{\texttt{{\color{var}w}.{\color{var}next}}} contains an item that appears before \mbox{\texttt{{\color{var}u}}} in $L_0$, then
we set $\mbox{\texttt{{\color{var}w}}}=\mbox{\texttt{{\color{var}w}.{\color{var}next}}}$.  Otherwise, we move down and continue the search
at the occurence of \mbox{\texttt{{\color{var}w}}} in the list $L_{h-1}$.  We continue this way
until we reach the predecessor of \mbox{\texttt{{\color{var}u}}} in $L_0$. 
\begin{figure}
  \begin{center}
    \includegraphics{figs/skiplist-searchpath}
  \end{center}
  \caption{The search path for node $4$ in a skiplist.}
  \figlabel{skiplist-searchpath}
\end{figure}

The following result, that we will prove in \secref{skiplist-analysis},  shows that the search path is quite short:

\begin{lem}\lemlabel{skiplist-searchpath}
The expected length of the search path for any node \mbox{\texttt{{\color{var}u}}} in $L_0$ is at
most $2\log_2 \mbox{\texttt{{\color{var}n}}} + O(1) = O(\log \mbox{\texttt{{\color{var}n}}})$.
\end{lem}

A space-efficient way to implement a \mbox{\texttt{Skiplist}} is to define a \mbox{\texttt{Node}}
\mbox{\texttt{{\color{var}u}}} as consisting of a data value \mbox{\texttt{{\color{var}x}}} and an array \mbox{\texttt{{\color{var}next}}} of pointers,
where \mbox{\texttt{{\color{var}u}.{\color{var}next}[{\color{var}i}]}} points to \mbox{\texttt{{\color{var}u}}}'s successor in the list $L_{\mbox{\texttt{{\color{var}i}}}}$.
In this way, the data, \mbox{\texttt{{\color{var}x}}}, in a node is stored only once even though \mbox{\texttt{{\color{var}x}}}
many appear in several lists.

\begin{Verbatim}[tabsize=2,frame=single,commandchars=\\@\$,label=\texttt{SkiplistSet},labelposition=topline]
	@\color@keyword$class$ Node {
		@\color@keyword$T$ @\color@var$x$;
		Node[] @\color@var$next$;
		Node(@\color@keyword$T$ @\color@var$ix$, @\color@keyword$int$ @\color@var$h$) {
			@\color@var$x$ = @\color@var$ix$;
			@\color@var$next$ = (Node[])Array.newInstance(Node.@\color@keyword$class$, @\color@var$h$+1);
		}
	}
\end{Verbatim}

The next three sections of this chapter discuss three different
applications of skiplists.  In each of these applications, $L_0$ stores the
main structure (a list of elements, a sorted set of elements, or a sequence
of strings).  The primary difference between these three structures is how
the search path is navigated; In particular the three structures differ in
how they decide if the search path should go down in $L_{r-1}$ or go right
in $L_r$.

\section{SkiplistSets as SortedSets}

A \mbox{\texttt{SkiplistSet}} uses a skiplist structure to implement the \mbox{\texttt{SortedSet}}
interface.   When used this way, the list $L_0$ stores the elements of
the \mbox{\texttt{SortedSet}} in sorted order.  The \mbox{\texttt{find({\color{var}x})}} method works by following
the search path for the smallest value \mbox{\texttt{{\color{var}y}}} such that $\mbox{\texttt{{\color{var}y}}}\ge\mbox{\texttt{{\color{var}x}}}$:

\begin{Verbatim}[tabsize=2,frame=single,commandchars=\\@\$,label=\texttt{SkiplistSet},labelposition=topline]
	Node findPredNode(@\color@keyword$T$ @\color@var$x$) {
		Node @\color@var$u$ = @\color@var$sentinel$;
		@\color@keyword$int$ @\color@var$r$ = @\color@var$height$ - 1;
		@\color@keyword$while$ (@\color@var$r$ >= 0) {
			@\color@keyword$while$ (@\color@var$u$.@\color@var$next$[@\color@var$r$] != @\color@var$null$ && compare(@\color@var$u$.@\color@var$next$[@\color@var$r$].@\color@var$x$,@\color@var$x$) < 0)
				@\color@var$u$ = @\color@var$u$.@\color@var$next$[@\color@var$r$];   @\color@comment$// go right in list r$
			@\color@var$r$--;                 @\color@comment$// go down into list r-1$
		}
		@\color@var$return$ @\color@var$u$;
	}
	@\color@keyword$T$ find(@\color@keyword$T$ @\color@var$x$) {
		Node @\color@var$u$ = findPredNode(@\color@var$x$);
		@\color@var$return$ @\color@var$u$.@\color@var$next$[0] == @\color@var$null$ ? @\color@var$null$ : @\color@var$u$.@\color@var$next$[0].@\color@var$x$;
	}
\end{Verbatim}

Following the search path for \mbox{\texttt{{\color{var}y}}} is easy:  When situated at some node \mbox{\texttt{{\color{var}u}}}
in  $L_{\mbox{\texttt{{\color{var}r}}}}$, we look right to \mbox{\texttt{{\color{var}u}.{\color{var}next}[{\color{var}r}].{\color{var}x}}}.  If \mbox{\texttt{{\color{var}x}}} is smaller, we take
a step to the right in $L_{\mbox{\texttt{{\color{var}r}}}}$, otherwise we move down into $L_{\mbox{\texttt{{\color{var}r}}}-1}$.
Each step (right or down) in this search takes only constant time so,
by \lemref{skiplist-searchpath}, the expected running time of \mbox{\texttt{find({\color{var}x})}}
is $O(\mbox{\texttt{{\color{var}n}}})$.

Adding an element \mbox{\texttt{{\color{var}x}}} to a skiplist using the \mbox{\texttt{add({\color{var}x})}} operation is
just a slight variation on searching.  Before we begin, we toss coins
to determine the height, \mbox{\texttt{{\color{var}k}}}, of \mbox{\texttt{{\color{var}x}}}, so that \mbox{\texttt{{\color{var}x}}} will appear in all
lists $L_0,\ldots,L_{\mbox{\texttt{{\color{var}k}}}}$.  If necessary, we increase the height of
skiplist, and then follow the search path for \mbox{\texttt{{\color{var}x}}}.  Any time the search
path goes down from a list $L_{\mbox{\texttt{{\color{var}r}}}}$ into $L_{\mbox{\texttt{{\color{var}r}}}-1}$, with $\mbox{\texttt{{\color{var}r}}}\le \mbox{\texttt{{\color{var}k}}}$,
we splice \mbox{\texttt{{\color{var}x}}} into the list $L_{\mbox{\texttt{{\color{var}r}}}}$ the same way we splice a node into
an \mbox{\texttt{SLList}}:

\begin{Verbatim}[tabsize=2,frame=single,commandchars=\\@\$,label=\texttt{SkiplistSet},labelposition=topline]
	@\color@var$boolean$ add(Node @\color@var$w$) {
		@\color@keyword$int$ @\color@var$k$ = @\color@var$w$.@\color@var$next$.@\color@var$length$ - 1;
		Node @\color@var$u$ = @\color@var$sentinel$;
		@\color@keyword$int$ @\color@var$r$ = @\color@var$height$ - 1;
		@\color@keyword$int$ @\color@var$comp$ = 0;
		@\color@keyword$while$ (@\color@var$r$ >= 0) {
			@\color@keyword$while$ (@\color@var$u$.@\color@var$next$[@\color@var$r$] != @\color@var$null$ && (@\color@var$comp$ = compare(@\color@var$u$.@\color@var$next$[@\color@var$r$].@\color@var$x$,@\color@var$w$.@\color@var$x$)) < 0)
				@\color@var$u$ = @\color@var$u$.@\color@var$next$[@\color@var$r$];
			@\color@keyword$if$ (@\color@var$u$.@\color@var$next$[@\color@var$r$] != @\color@var$null$ && @\color@var$comp$ == 0) @\color@var$return$ @\color@var$false$; @\color@comment$// already present$
			@\color@keyword$if$ (@\color@var$r$ <= @\color@var$k$) {
				@\color@var$w$.@\color@var$next$[@\color@var$r$] = @\color@var$u$.@\color@var$next$[@\color@var$r$];
				@\color@var$u$.@\color@var$next$[@\color@var$r$] = @\color@var$w$;
			}
			@\color@var$r$--;
		}
		@\color@var$n$++;
		@\color@var$return$ @\color@var$true$;
	}
\end{Verbatim}
\begin{Verbatim}[tabsize=2,frame=single,commandchars=\\@\$,label=\texttt{SkiplistSet},labelposition=topline]
	@\color@var$boolean$ add(@\color@keyword$T$ @\color@var$x$) {
		Node @\color@var$w$ = @\color@keyword$new$ Node(@\color@var$x$, pickHeight());
		@\color@keyword$if$ (@\color@var$w$.@\color@var$next$.@\color@var$length$ > @\color@var$height$)
			@\color@var$height$ = @\color@var$w$.@\color@var$next$.@\color@var$length$;
		@\color@var$return$ add(@\color@var$w$);
	}
\end{Verbatim}

\begin{figure}
  \begin{center}
    \includegraphics{figs/skiplist-add}
  \end{center}
  \caption{Adding the node 3.5 to a skiplist.}
  \figlabel{skiplist-add}
\end{figure}

Removing an element \mbox{\texttt{{\color{var}x}}} is done in a similar way.  We search for \mbox{\texttt{{\color{var}x}}} and each time the search moves downward from a node \mbox{\texttt{{\color{var}u}}}, we check if $\mbox{\texttt{{\color{var}u}.{\color{var}next}.{\color{var}x}}}=\mbox{\texttt{{\color{var}x}}}$ and if so, we splice \mbox{\texttt{{\color{var}u}}} out of the list:

\begin{Verbatim}[tabsize=2,frame=single,commandchars=\\@\$,label=\texttt{SkiplistSet},labelposition=topline]
	@\color@var$boolean$ remove(@\color@keyword$T$ @\color@var$x$) {
		@\color@var$boolean$ @\color@var$removed$ = @\color@var$false$;
		Node @\color@var$u$ = @\color@var$sentinel$;
		@\color@keyword$int$ @\color@var$r$ = @\color@var$height$ - 1;
		@\color@keyword$int$ @\color@var$comp$ = 0;
		@\color@keyword$while$ (@\color@var$r$ >= 0) {
			@\color@keyword$while$ (@\color@var$u$.@\color@var$next$[@\color@var$r$] != @\color@var$null$ && (@\color@var$comp$ = compare(@\color@var$u$.@\color@var$next$[@\color@var$r$].@\color@var$x$, @\color@var$x$)) < 0) {
				@\color@var$u$ = @\color@var$u$.@\color@var$next$[@\color@var$r$];
			}
			@\color@keyword$if$ (@\color@var$u$.@\color@var$next$[@\color@var$r$] != @\color@var$null$ && @\color@var$comp$ == 0) {
				@\color@var$removed$ = @\color@var$true$;
				@\color@var$u$.@\color@var$next$[@\color@var$r$] = @\color@var$u$.@\color@var$next$[@\color@var$r$].@\color@var$next$[@\color@var$r$];
			}
			@\color@var$r$--;
		}
		@\color@keyword$if$ (@\color@var$removed$) @\color@var$n$--;
		@\color@var$return$ @\color@var$removed$;
	}
\end{Verbatim}

\begin{figure}
  \begin{center}
    \includegraphics{figs/skiplist-remove}
  \end{center}
  \caption{Removing the node 3 from a skiplist.}
  \figlabel{skiplist-remove}
\end{figure}

\subsection{Summary}

The following theorem summarizes the performance of skiplists when used to
implement sorted sets:

\begin{thm}\thmlabel{skiplist}
A \mbox{\texttt{SkiplistSet}} implements the \mbox{\texttt{SortedSet}} interface. A \mbox{\texttt{SkiplistSet}} supports
the operations \mbox{\texttt{add({\color{var}x})}}, \mbox{\texttt{remove({\color{var}x})}}, and \mbox{\texttt{find({\color{var}x})}} in $O(\log n)$
expected time per operation.
\end{thm}

\section{Skiplists as Lists}

A \mbox{\texttt{SkiplistList}} is a realization of the skiplist idea that implements
the \mbox{\texttt{List}} interface.  In a \mbox{\texttt{SkiplistList}}, $L_0$ contains the elements of the
list in the order they appear in the list.   Just like a \mbox{\texttt{SkiplistSet}}
elements can be added, removed, and accessed in $O(\log \mbox{\texttt{{\color{var}n}}})$ time.

For this idea to work, we need a way to follow the search path for
the \mbox{\texttt{{\color{var}i}}}th element in $L_0$.  The easiest way to do this is to define
the notion of the \emph{length} of an edge in some list $L_{\mbox{\texttt{{\color{var}r}}}}$.
The length of every edge in $L_{0}$ is 1.  The length of an edge \mbox{\texttt{{\color{var}e}}}
in $L_{\mbox{\texttt{{\color{var}r}}}}$, $\mbox{\texttt{{\color{var}r}}}>0$ is the sum of the lengths of the edges below \mbox{\texttt{{\color{var}e}}}
in $L_{\mbox{\texttt{{\color{var}r}}}-1}$.  Equivalently, the length of an edge \mbox{\texttt{{\color{var}e}}} in $L_{\mbox{\texttt{{\color{var}r}}}}$ is
the number of edges in $L_0$ below \mbox{\texttt{{\color{var}e}}}.  See \figref{skiplist-lengths} for
an example of a skiplist with the lengths of its edges shown.  Since the
edges of skiplists are stored in arrays, the lengths can be stored the same
way:

\begin{figure}
  \begin{center}
    \includegraphics{figs/skiplist-lengths}
  \end{center}
  \caption{The lengths of the edges in a skiplist.}
  \figlabel{skiplist-lengths}
\end{figure}

\begin{Verbatim}[tabsize=2,frame=single,commandchars=\\@\$,label=\texttt{SkiplistList},labelposition=topline]
	@\color@keyword$class$ Node {
		@\color@keyword$T$ @\color@var$x$;
		Node[] @\color@var$next$;
		@\color@keyword$int$[] @\color@var$length$;
		Node(@\color@keyword$T$ @\color@var$ix$, @\color@keyword$int$ @\color@var$h$) {
			@\color@var$x$ = @\color@var$ix$;
			@\color@var$next$ = (Node[])Array.newInstance(Node.@\color@keyword$class$, @\color@var$h$+1);
			@\color@var$length$ = @\color@keyword$new$ @\color@keyword$int$[@\color@var$h$+1];
		}
	}
\end{Verbatim}

The useful property of this definition of length is that, if we are
currently at a node that is at position \mbox{\texttt{{\color{var}j}}} in $L_0$ and we follow an
edge of length $\ell$, then we move to a node whose position, in $L_0$,
is $\mbox{\texttt{{\color{var}j}}}+\ell$.  In this way, while following a search path, we can keep
track of the position, \mbox{\texttt{{\color{var}j}}}, of the current node in $L_0$.  When at a
node \mbox{\texttt{{\color{var}u}}} in $L_{\mbox{\texttt{{\color{var}r}}}}$ we go right if \mbox{\texttt{{\color{var}j}}}
plus the length of the edge \mbox{\texttt{{\color{var}u}.{\color{var}next}}} is less than \mbox{\texttt{{\color{var}i}}}, otherwise we go
down into $L_{\mbox{\texttt{{\color{var}r}}}-1}$.

\begin{Verbatim}[tabsize=2,frame=single,commandchars=\\@\$,label=\texttt{SkiplistList},labelposition=topline]
	Node findPred(@\color@keyword$int$ @\color@var$i$) {
		Node @\color@var$u$ = @\color@var$sentinel$;
		@\color@keyword$int$ @\color@var$r$ = @\color@var$height$ - 1;
		@\color@keyword$int$ @\color@var$j$ = -1;   @\color@comment$// the index of the current node in list 0$
		@\color@keyword$while$ (@\color@var$r$ >= 0) {
			@\color@keyword$while$ (@\color@var$u$.@\color@var$next$[@\color@var$r$] != @\color@var$null$ && @\color@var$j$ + @\color@var$u$.@\color@var$length$[@\color@var$r$] < @\color@var$i$) {
				@\color@var$j$ += @\color@var$u$.@\color@var$length$[@\color@var$r$];
				@\color@var$u$ = @\color@var$u$.@\color@var$next$[@\color@var$r$];
			}
			@\color@var$r$--;
		}
		@\color@var$return$ @\color@var$u$;
	}
\end{Verbatim}
\begin{Verbatim}[tabsize=2,frame=single,commandchars=\\@\$,label=\texttt{SkiplistList},labelposition=topline]
	@\color@keyword$T$ get(@\color@keyword$int$ @\color@var$i$) {
		@\color@var$return$ findPred(@\color@var$i$).@\color@var$next$[0].@\color@var$x$;
	}
\end{Verbatim}
\begin{Verbatim}[tabsize=2,frame=single,commandchars=\\@\$,label=\texttt{SkiplistList},labelposition=topline]
	@\color@keyword$T$ set(@\color@keyword$int$ @\color@var$i$, @\color@keyword$T$ @\color@var$x$) {
		Node @\color@var$u$ = findPred(@\color@var$i$).@\color@var$next$[0];
		@\color@keyword$T$ @\color@var$y$ = @\color@var$u$.@\color@var$x$;
		@\color@var$u$.@\color@var$x$ = @\color@var$x$;
		@\color@var$return$ @\color@var$y$;
	}
\end{Verbatim}

Since the hardest part of the operations \mbox{\texttt{get({\color{var}i})}} and \mbox{\texttt{set({\color{var}i},{\color{var}x})}} is
finding the \mbox{\texttt{{\color{var}i}}}th node in $L_0$, these operations therefore run in
$O(\log \mbox{\texttt{{\color{var}n}}})$ time.

Adding an element to a \mbox{\texttt{SkiplistList}} is fairly straightforward.  We use
the same method as before. We first pick the height \mbox{\texttt{{\color{var}k}}} of the newly
inserted node and the follow the search path for \mbox{\texttt{{\color{var}i}}}.  Anytime the
search path moves down from $L_{\mbox{\texttt{{\color{var}r}}}}$ with $\mbox{\texttt{{\color{var}r}}}\le \mbox{\texttt{{\color{var}k}}}$, we splice the new
node into $L_{\mbox{\texttt{{\color{var}r}}}}$.  The only extra care needed is to ensure that the
lengths of edges are updated properly. 
See \figref{skiplist-addix}. 

\begin{figure}
  \begin{center}
    \includegraphics{figs/skiplist-addix}
  \end{center}
  \caption{Adding an element to a \mbox{\texttt{SkiplistList}}.}
  \figlabel{skiplist-addix}
\end{figure}

Note that, each time the search path goes down at a node \mbox{\texttt{{\color{var}u}}} in $L_{\mbox{\texttt{{\color{var}r}}}}$,
the length of the edge \mbox{\texttt{{\color{var}u}.{\color{var}next}[{\color{var}r}]}} increases by one, since we are adding
an element at position \mbox{\texttt{{\color{var}i}}}.  Splicing  the node \mbox{\texttt{{\color{var}w}}} between two nodes
\mbox{\texttt{{\color{var}u}}} and \mbox{\texttt{{\color{var}z}}} works as shown in \figref{skiplist-lengths-splice}.  While
following the search path we are already keeping track of the position,
\mbox{\texttt{{\color{var}j}}}, of \mbox{\texttt{{\color{var}w}}} in $L_0$.  Therefore, we know that the length of the edge from
\mbox{\texttt{{\color{var}u}}} to \mbox{\texttt{{\color{var}w}}} is $\mbox{\texttt{{\color{var}i}}}-\mbox{\texttt{{\color{var}j}}}$.  We can also deduce that the length of the edge
from \mbox{\texttt{{\color{var}w}}}  to \mbox{\texttt{{\color{var}z}}} from the length, $\ell$, of the edge from \mbox{\texttt{{\color{var}u}}} to \mbox{\texttt{{\color{var}z}}}.
Therefore, we can splice in \mbox{\texttt{{\color{var}w}}} and update the lengths of the edges in
constant time.

\begin{figure}
  \begin{center}
    \includegraphics{figs/skiplist-lengths-splice}
  \end{center}
  \caption{Updating the lengths of edges while splicing a node 
   \mbox{\texttt{{\color{var}w}}} into a skiplist.}
  \figlabel{skiplist-lengths-splice}
\end{figure}

This sounds more complicated than it actually is, and the code is actually
quite simple:

\begin{Verbatim}[tabsize=2,frame=single,commandchars=\\@\$,label=\texttt{SkiplistList},labelposition=topline]
	@\color@var$void$ add(@\color@keyword$int$ @\color@var$i$, @\color@keyword$T$ @\color@var$x$) {
		Node @\color@var$w$ = @\color@keyword$new$ Node(@\color@var$x$, pickHeight());
		@\color@keyword$if$ (@\color@var$w$.@\color@var$next$.@\color@var$length$ > @\color@var$height$) 
			@\color@var$height$ = @\color@var$w$.@\color@var$next$.@\color@var$length$;
		add(@\color@var$i$, @\color@var$w$);
	}
\end{Verbatim}
\begin{Verbatim}[tabsize=2,frame=single,commandchars=\\@\$,label=\texttt{SkiplistList},labelposition=topline]
	Node add(@\color@keyword$int$ @\color@var$i$, Node @\color@var$w$) {
		@\color@keyword$if$ (@\color@var$i$ < 0 || @\color@var$i$ > @\color@var$n$) @\color@keyword$throw$ @\color@keyword$new$ IndexOutOfBoundsException();
		Node @\color@var$u$ = @\color@var$sentinel$;
		@\color@keyword$int$ @\color@var$k$ = @\color@var$w$.@\color@var$next$.@\color@var$length$ - 1;
		@\color@keyword$int$ @\color@var$r$ = @\color@var$height$ - 1;
		@\color@keyword$int$ @\color@var$j$ = -1; @\color@comment$// index of u$
		@\color@keyword$while$ (@\color@var$r$ >= 0) {
			@\color@keyword$while$ (@\color@var$u$.@\color@var$next$[@\color@var$r$] != @\color@var$null$ && @\color@var$j$+@\color@var$u$.@\color@var$length$[@\color@var$r$] < @\color@var$i$) {
				@\color@var$j$ += @\color@var$u$.@\color@var$length$[@\color@var$r$];
				@\color@var$u$ = @\color@var$u$.@\color@var$next$[@\color@var$r$];
			}
			@\color@var$u$.@\color@var$length$[@\color@var$r$]++;    @\color@comment$// to account for new node in list 0$
			@\color@keyword$if$ (@\color@var$r$ <= @\color@var$k$) {
				@\color@var$w$.@\color@var$next$[@\color@var$r$] = @\color@var$u$.@\color@var$next$[@\color@var$r$];
				@\color@var$u$.@\color@var$next$[@\color@var$r$] = @\color@var$w$;
				@\color@var$w$.@\color@var$length$[@\color@var$r$] = @\color@var$u$.@\color@var$length$[@\color@var$r$] - (@\color@var$i$ - @\color@var$j$);
				@\color@var$u$.@\color@var$length$[@\color@var$r$] = @\color@var$i$ - @\color@var$j$;
			}
			@\color@var$r$--;
		}
		@\color@var$n$++;
		@\color@var$return$ @\color@var$u$;
	}
\end{Verbatim}


By now, the implementation of 
the \mbox{\texttt{remove({\color{var}i})}} operation in a \mbox{\texttt{SkiplistList}} should be obvious.  We follow the search path for the node at position \mbox{\texttt{{\color{var}i}}}.  Each time the search path takes a step down from a node \mbox{\texttt{{\color{var}u}}} at level \mbox{\texttt{{\color{var}r}}} we decrement the length of the edge leaving \mbox{\texttt{{\color{var}u}}} at level \mbox{\texttt{{\color{var}r}}}.  We also check if \mbox{\texttt{{\color{var}u}.{\color{var}next}}} is the element of rank \mbox{\texttt{{\color{var}i}}} and, if so, splice it out the list at level \mbox{\texttt{{\color{var}i}}}.   An example is shown in \figref{skiplist-removei}.
\begin{figure}
  \begin{center}
    \includegraphics{figs/skiplist-removei}
  \end{center}
  \caption{Removing an element from a \mbox{\texttt{SkiplistList}}.}
  \figlabel{skiplist-removei}
\end{figure}
\begin{Verbatim}[tabsize=2,frame=single,commandchars=\\@\$,label=\texttt{SkiplistList},labelposition=topline]
	@\color@keyword$T$ remove(@\color@keyword$int$ @\color@var$i$) {
		@\color@keyword$if$ (@\color@var$i$ < 0 || @\color@var$i$ > @\color@var$n$-1) @\color@keyword$throw$ @\color@keyword$new$ IndexOutOfBoundsException();
		@\color@keyword$T$ @\color@var$x$ = @\color@var$null$;
		Node @\color@var$u$ = @\color@var$sentinel$;
		@\color@keyword$int$ @\color@var$r$ = @\color@var$height$ - 1;
		@\color@keyword$int$ @\color@var$j$ = -1; @\color@comment$// index of node u$
		@\color@keyword$while$ (@\color@var$r$ >= 0) {
			@\color@keyword$while$ (@\color@var$u$.@\color@var$next$[@\color@var$r$] != @\color@var$null$ && @\color@var$j$+@\color@var$u$.@\color@var$length$[@\color@var$r$] < @\color@var$i$) {
				@\color@var$j$ += @\color@var$u$.@\color@var$length$[@\color@var$r$];
				@\color@var$u$ = @\color@var$u$.@\color@var$next$[@\color@var$r$];
			}
			@\color@var$u$.@\color@var$length$[@\color@var$r$]--;  @\color@comment$// for the node we are removing$
			@\color@keyword$if$ (@\color@var$j$ + @\color@var$u$.@\color@var$length$[@\color@var$r$] == @\color@var$i$ && @\color@var$u$.@\color@var$next$[@\color@var$r$] != @\color@var$null$) {
				@\color@var$x$ = @\color@var$u$.@\color@var$next$[@\color@var$r$].@\color@var$x$;
				@\color@var$u$.@\color@var$length$[@\color@var$r$] += @\color@var$u$.@\color@var$next$[@\color@var$r$].@\color@var$length$[@\color@var$r$];
				@\color@var$u$.@\color@var$next$[@\color@var$r$] = @\color@var$u$.@\color@var$next$[@\color@var$r$].@\color@var$next$[@\color@var$r$];
			}
			@\color@var$r$--;
		}
		@\color@var$n$--;
		@\color@var$return$ @\color@var$x$;
	}
\end{Verbatim}

\subsection{Summary}


\section{Skiplists as Ropes}

\section{Analysis of Skiplists}
\seclabel{skiplist-analysis}

In this section, we analyze the expected height, size, and length of
the search path in a skiplist.  This section requires a background in
basic probability.

\begin{lem}
  The expected height of a skiplist containing $n$ elements is at most
  $\log_2 n + 3$.
\end{lem}

\begin{proof}
  Define the indicator random variable
  \[ I_r = \left\{\begin{array}{ll}
     0 & \mbox{if $L_r$ is empty} \\
     1 & \mbox{if $L_r$ is non-empty}
     \end{array}\right.
  \]
  The height $h$ of the skiplist is then given by
  \[
       h = \sum_{i=0}^\infty I_r \enspace .
  \]
  Note that $I_r \le |L_r|$, so 
  \[
     \E[I_r] \le \E[|L_r|] = n/2^r \enspace .
  \]
  Therefore, we have
  \begin{eqnarray*}
       \E[h] &=& \E\left[\sum_{r=0}^\infty I_r\right] \\
        &=& \sum_{r=0}^{\infty} E[I_r] \\
        &=& \sum_{r=0}^{\lfloor\log n\rfloor} E[I_r]
                 + \sum_{r=\lfloor\log n\rfloor+1}^{\infty} E[I_r]  \\
        &=& \sum_{r=0}^{\lfloor\log n\rfloor} 1
                 + \sum_{r=\lfloor\log n\rfloor+1}^{\infty} n/2^r \\
        &\le& \log n + 1
                 + \sum_{r=0}^\infty 1/2^r \\
        &=& \log n + 3
  \end{eqnarray*}
\end{proof}

\begin{lem}
  The expected number of nodes in a skiplist containing $n$ elements,
  including all occurrences of the sentinel is $2n+O(\log n)$.
\end{lem}

\begin{proof}
The expected number of nodes, not including the sentinel is 
\[
  \E\left[\sum_{i=0}^{\infty} |L_i|\right] = 
  \sum_{i=0}^{\infty} n/2^i = 2n \enspace .
\]
The number of occurrences of the sentinel is equal to the height, $h$,
of the skiplist, so the expected number of occurrences of the sentinel
is at most $\log n+3 = O(\log n)$.
\end{proof}

\begin{lem}
The expected length of a search path in a skiplist is $2\log n + O(1)$.
\end{lem}

\begin{proof}
  The easiest way to see this is to consider the \emph{reverse search
  path} for a node \mbox{\texttt{{\color{var}x}}}.  This path starts at the predecessor of \mbox{\texttt{{\color{var}x}}}
  in $L_0$.  At any point in time, if the path can go up a level, then
  it does.  If it can not go up a level then it goes left.  Observe that
  the reverse search path for \mbox{\texttt{{\color{var}x}}} is identical to the search path for \mbox{\texttt{{\color{var}x}}},
  except that it is reversed.
  
  The number of nodes that the reverse search path visits at a particular
  level $r$ is related to the following experiment:  Toss a coin.
  If the coin comes up heads then go up and stop, otherwise go left and
  repeat the experiment.  The number of coin tosses before the heads then
  represents the number of steps to the left that a reverse search path
  takes at a particular level.  (Note that this overcounts the number of
  steps to the left, since this experiment ends either at the first head
  or when the search path reaches the sentinel, whichever comes first.)
  
  Let $R_r$ denote the number of steps the search path takes at level
  $r$ that go to the right.   We have just argued that $\E[R_r]\le 1$.
  Furthermore, $R_r\le |L_r|$, so
  \[
    \E[R_r] \le E[|L_r|] = n/2^r \enspace .
  \]
  We can now finish as in the proof of \lemref{skiplist-height}.
  Let $\ell$ be  the length of the search path for some node \mbox{\texttt{{\color{var}u}}} in a
  skiplist, and let $h$ be the height of the skiplist.  Then
  \begin{eqnarray*}
      \E[\ell] 
         &=& \E\left[ h + \sum_{r=0}^\infty R_r \right] \\
         &=& \E[h] + \sum_{r=0}^\infty \E[R_r]  \\
         &=& \E[h] + \sum_{r=0}^{\lfloor\log n\rfloor} \E[R_r] 
              + \sum_{r=\lfloor\log n\rfloor+1}^\infty \E[R_r] \\
         &\le& \E[h] + \sum_{r=0}^{\lfloor\log n\rfloor} 1
              + \sum_{r=\lfloor\log n\rfloor+1}^\infty n/2^r \\
         &\le& \E[h] + \sum_{r=0}^{\lfloor\log n\rfloor} 1
              + \sum_{r=0}^{\infty} 1/2^r \\
         &\le& \E[h] + \sum_{r=0}^{\lfloor\log n\rfloor} 1
              + \sum_{r=0}^{\infty} 1/2^r \\
         &\le& \E[h] + \log n + 3 \\
         &\le& 2\log n + 6
  \end{eqnarray*}
  as required.
\end{proof}


\section{Iteration and Finger Search}

%TODO: Write this section

\section{Summary and Exercises}

\begin{enumerate}
\item Suppose that, instead of promoting an element from $L_{i-1}$ into
$L_i$, we promote it with some probability $p$, $0 < p < 1$.  Show that
the expected length of the search path in this case is $(1/p)\log_{1/p}
n + O(1)$.  What is the value of $p$ that minimizes this expression? What
is the expected height of the skiplist? What is the expected number of
nodes in the skiplist?

\item Design and implement a version of a skiplist that implements the
\mbox{\texttt{SortedSet}} interface, but also allows access to elements by rank.
That is, it also supports the function \mbox{\texttt{get({\color{var}i})}}, which returns the element
whose rank is \mbox{\texttt{{\color{var}i}}}. (The rank of an element \mbox{\texttt{{\color{var}x}}} in a \mbox{\texttt{SortedSet}} is the number of elements in the \mbox{\texttt{SortedSet}} that are less than \mbox{\texttt{{\color{var}x}}}.)
\end{enumerate}


\include{hashing-tmp}
\include{binarytrees-tmp}
\include{rbs-tmp}
\include{scapegoat-tmp}
\include{redblack-tmp}
\include{heaps-tmp}

%\chapter{Data Structures for Integers}
%\chaplabel{bounded}
%\section{X-Fast Tries}
%\section{Y-Fast Tries}

\chapter{Sorting Algorithms}
\chaplabel{sorting}
\section{Merge Sort}

\begin{thm}
  MergeSort runs in $O(n\log n)$ time and performs at most $n\log n$
  comparisons.
\end{thm}

\begin{proof}
The proof is by induction on $n$.  The base case $n=1$ is trivial.

Merging two sorted lists of total length $n$ requires at most $n-1$
comparisons. If $n$ is even, then we apply the inductive hypothesis to
the two subproblems and obtain
\begin{align}
  C(n) 
  &= n-1 + 2C(n/2) \\
  &= n-1 + 2((n/2)\log(n/2) \\
  &= n-1 + n\log(n/2) \\
  &= n-1 + n\log n-n \\
  &< n\log n \enspace .
\end{align}
The case where $n$ is odd is slightly more complicated.  For this case,
we use the inequality (justified below) that shows $\log (x+1/2) <
\log x + (\log e)/2x$.  So we have
\begin{align}
  C(n) 
  &= n-1 + C(\lceil n/2 \rceil) + C(\lfloor n/2 \rfloor) \\
  &= n-1 + \lceil n/2 \rceil\log \lceil n/2 \rceil 
           + \lfloor n/2 \rfloor\log \lfloor n/2 \rfloor \\
  &= n-1 + (n/2+1/2)\log (n/2+1/2) 
           + \lfloor n/2 \rfloor\log \lfloor n/2 \rfloor \\
  &< n-1 + \lceil n/2 \rceil(\log n/2 + (\log e)/n)
           + \lfloor n/2 \rfloor\log (n/2) \\
  &= n\log n - 1 + (n/2+1/2)(\log e/n) \\
  &< n\log n 
\end{align}
since $(n/2+1/2)(\log e/n) < 1$ for all $n\ge 2$.

All that remains is to justify the inequality
$\log (x+1/2) <
\log x + (\log e)/2x$.  This inequality comes from the identity $\ln x = \int_1^x (1/x),\mathrm{d}x$.  This identity yields:
\[
   \ln(x+1/2) = \int_1^{x+1/2} (1/x)\,\mathrm{d}x
    = \int_1^{x} (1/x)\,\mathrm{d}x + \int_{x}^{x+1/2} (1/x)\,\mathrm{d}x
    < \int_1^{x} (1/x)\,\mathrm{d}x + 1/2x \enspace .\footnote{The same derivation yields the famous inequality, seen in many calculus books, $1+x < e^x$.}
\]
Multiplying both sides of this inequality by $\log e$ yields
$\log (x+1/2) < \log x + (\log e)/2x$.
\end{proof}



\section{Quicksort}
\section{Heapsort}

\bibliographystyle{plainurl}
\bibliography{ods}

\end{document}

