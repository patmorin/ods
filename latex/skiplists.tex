\chapter{Skiplists}
\chaplabel{skiplists}

\section{The Basic Structure}

Conceptually, a skiplist is a sequence of singly-linked lists
$L_0,\ldots,L_h$, where each $L_i$ contains a subset of the items in
$L_{i-1}$.  We start with the input list $L_0$ and construct $L_1$ from
$L_0$, $L_2$ from $L_1$, and so on.  The items in $L_i$ are obtained
by tossing a coin for each element #x# in $L_{i-1}$ and including #x#
in $L_i$ if the coin comes up heads.  This process ends when the first
list becomes empty.

At the head of every list is a non-input node, called the \emph{sentinel}
that acts as dummy node for each list.  The key property of skiplists
is that there is a short path, called the search path, from the sentinel
in $L_h$ to every node in $L_0$.  To construct the search path for the node
#u# in $L_0$, we search 




\section{Skiplists as SortedSets}
\section{Skiplists as Lists}
\section{Skiplists as Ropes}


