\chapter{Skiplists}

\chaplabel{skiplists}

In this chapter we discuss a beautiful data structure, the skip list,
that has a variety of applications.  Using a skiplist we can implement
a #List# that is fast for all the operations #get(i)#, #set(i,x)#,
#add(i,x)#, and #remove(i)#. We can also implement an #SSet# in which
all operations run in $O(\log #n#)$ expected time.
%Finally, a skiplist can be used to implement a #Rope# in which all
%operations run in $O(\log #n#)$ time.

Skiplists rely on \emph{randomization} for their efficiency.
In particular, a skiplist uses random coin tosses when an element is
inserted to determine the height of that element.  The performance
of skiplists is expressed in terms of \emph{expected} running times
and lengths of paths. This expectation is taken over the random coin
tosses used by the skiplist.  In the implementation, the random coin
tosses used by a skiplist are simulated using a pseudo-random number
(or bit) generator.

\section{The Basic Structure}

Conceptually, a skiplist is a sequence of singly-linked lists
$L_0,\ldots,L_h$, where each $L_r$ contains a subset of the items
in $L_{r-1}$.  We start with the input list $L_0$ that contains #n#
items and construct $L_1$ from $L_0$, $L_2$ from $L_1$, and so on.
The items in $L_r$ are obtained by tossing a coin for each element, #x#,
in $L_{r-1}$ and including #x# in $L_r$ if the coin comes up heads.
This process ends when we create a list $L_r$ that is empty.  An example
of a skiplist is shown in \figref{skiplist}.

\begin{figure}
  \begin{center}
    \includegraphics{figs/skiplist}
  \end{center}
  \caption{A skiplist containing 7 elements.}
  \figlabel{skiplist}
\end{figure}

For an element, #x#, in a skiplist, we call the \emph{height} of #x# the
largest value $r$ such that #x# appears in $L_r$.  Thus, for example,
elements that only appear in $L_0$ have height $0$.  Notice that the
height of #x# corresponds to the following experiment:  Toss a coin
repeatedly until the first time it comes up tails.  How many times did it
come up heads?  The answer, not surprisingly, is that the expected height
of a node is 1. (We expect to toss the coin twice before getting tails,
but we don't count the last toss.) The \emph{height} of a skiplist is
the height of its tallest node.

At the head of every list is a special node called the \emph{sentinel}
that acts as a dummy node for the list. The key property of skiplists
is that there is a short path, called the \emph{search path}, from the
sentinel in $L_h$ to every node in $L_0$.  Remembering how to construct
a search path for a node, #u#, is easy (see \figref{skiplist-searchpath})
:  Start at the top left corner of your skiplist (the sentinel in $L_h$)
and always go right unless that would overshoot #u#, in which case you
should take a step down into the list below.

More precisely, to construct the search path for the node #u# in $L_0$
we start at the sentinel, #w#, in $L_h$.  Next, we examine #w.next#.
If #w.next# contains an item that appears before #u# in $L_0$, then
we set $#w#=#w.next#$.  Otherwise, we move down and continue the search
at the occurence of #w# in the list $L_{h-1}$.  We continue this way
until we reach the predecessor of #u# in $L_0$. 
\begin{figure}
  \begin{center}
    \includegraphics{figs/skiplist-searchpath}
  \end{center}
  \caption{The search path for the node containing $4$ in a skiplist.}
  \figlabel{skiplist-searchpath}
\end{figure}

The following result, which we will prove in \secref{skiplist-analysis},
shows that the search path is quite short:

\begin{lem}\lemlabel{skiplist-searchpath}
The expected length of the search path for any node, #u#, in $L_0$ is at
most $2\log #n# + O(1) = O(\log #n#)$.
\end{lem}

A space-efficient way to implement a #Skiplist# is to define a #Node#,
#u#, as consisting of a data value, #x#, and an array, #next#, of pointers,
where #u.next[i]# points to #u#'s successor in the list $L_{#i#}$.
In this way, the data, #x#, in a node is referenced only once, even though #x#
may appear in several lists.

\javaimport{ods/SkiplistSet.Node}

The next two sections of this chapter discuss two different applications
of skiplists.  In each of these applications, $L_0$ stores the main
structure (a list of elements or a sorted set of elements).
The primary difference between these structures is in how
a search path is navigated; in particular, they differ in how
they decide if a search path should go down into $L_{r-1}$ or go right
within $L_r$.

\section{#SkiplistSet#: An Efficient #SSet# Implementation}

A #SkiplistSet# uses a skiplist structure to implement the #SSet#
interface.   When used this way, the list $L_0$ stores the elements of
the #SSet# in sorted order.  The #find(x)# method works by following
the search path for the smallest value #y# such that $#y#\ge#x#$:

\javaimport{ods/SkiplistSet.find(x).findPredNode(x)}

Following the search path for #y# is easy:  when situated at
some node, #u#, in  $L_{#r#}$, we look right to #u.next[r].x#.
If $#x#>#u.next[r].x#$, then we take a step to the right in $L_{#r#}$,
otherwise we move down into $L_{#r#-1}$.  Each step (right or down) in
this search takes only constant time so, by \lemref{skiplist-searchpath},
the expected running time of #find(x)# is $O(\log #n#)$.

Adding an element #x# to a skiplist using the #add(x)# operation is just
a slight variation on searching.  Before we begin, we simulate tossing
coins to determine the height, #k#, of #x#.  We do this by picking a
random integer, #z#, and counting the number of trailing $1$s in the binary
representation of #z#:\footnote{This method does not exactly replicate
the coin-tossing experiment since the value of #k# will always be less
than the number of bits in an #int#.  However, this will have negligible impact unless the number of elements in the structure is much greater than $2^{32}=4294967296$.}

\javaimport{ods/SkiplistSet.pickHeight()}

After picking the height, #k#, this way, we know that #x# should appear in
all the lists $L_0,\ldots,L_{#k#}$.  Next, we create a #Node#, #w#, for #x#, increase the height of
the skiplist if necessary and then follow the search path for #x#.  Any time the search
path goes down from a list $L_{#r#}$ into $L_{#r#-1}$, with $#r#\le #k#$,
we splice #w# into the list $L_{#r#}$ the same way we splice a node into
an #SLList#:

\javaimport{ods/SkiplistSet.add(x)}
\javaimport{ods/SkiplistSet.add(w)}

\begin{figure}
  \begin{center}
    \includegraphics{figs/skiplist-add}
  \end{center}
  \caption{Adding the node 3.5 to a skiplist.}
  \figlabel{skiplist-add}
\end{figure}

Removing an element, #x#, is done in a similar way.  We search for #x# and each time the search moves downward from a node #u#, we check if $#u.next.x#=#x#$ and if so, we splice #u# out of the list:

\javaimport{ods/SkiplistSet.remove(x)}

\begin{figure}
  \begin{center}
    \includegraphics{figs/skiplist-remove}
  \end{center}
  \caption{Removing the node 3 from a skiplist.}
  \figlabel{skiplist-remove}
\end{figure}

\subsection{Summary}

The following theorem summarizes the performance of skiplists when used to
implement sorted sets:

\begin{thm}\thmlabel{skiplist}
A #SkiplistSet# implements the #SSet# interface. A #SkiplistSet# supports
the operations #add(x)#, #remove(x)#, and #find(x)# in $O(\log #n#)$
expected time per operation.
\end{thm}

\section{#SkiplistList#: An Efficient Random-Access #List# Implementation}

A #SkiplistList# implements the #List# interface on top of a skiplist structure.
In a #SkiplistList#, $L_0$ contains the elements of the
list in the order they appear in the list.   Just like with a #SkiplistSet#,
elements can be added, removed, and accessed in $O(\log #n#)$ time.

For this to be possible, we need a way to follow the search path for
the #i#th element in $L_0$.  The easiest way to do this is to define
the notion of the \emph{length} of an edge in some list, $L_{#r#}$.
We define the length of every edge in $L_{0}$ as 1.  The length of an edge, #e#,
in $L_{#r#}$, $#r#>0$, is defined as the sum of the lengths of the edges below #e#
in $L_{#r#-1}$.  Equivalently, the length of #e# is
the number of edges in $L_0$ below #e#.  See \figref{skiplist-lengths} for
an example of a skiplist with the lengths of its edges shown.  Since the
edges of skiplists are stored in arrays, the lengths can be stored the same
way:

\begin{figure}
  \begin{center}
    \includegraphics{figs/skiplist-lengths}
  \end{center}
  \caption{The lengths of the edges in a skiplist.}
  \figlabel{skiplist-lengths}
\end{figure}

\javaimport{ods/SkiplistList.Node}

The useful property of this definition of length is that, if we are
currently at a node that is at position #j# in $L_0$ and we follow an
edge of length $\ell$, then we move to a node whose position, in $L_0$,
is $#j#+\ell$.  In this way, while following a search path, we can keep
track of the position, #j#, of the current node in $L_0$.  When at a
node, #u#, in $L_{#r#}$, we go right if #j#
plus the length of the edge #u.next[r]# is less than #i#, otherwise we go
down into $L_{#r#-1}$.

\javaimport{ods/SkiplistList.findPred(i)}
\javaimport{ods/SkiplistList.get(i).set(i,x)}

Since the hardest part of the operations #get(i)# and #set(i,x)# is
finding the #i#th node in $L_0$, these operations run in
$O(\log #n#)$ time.

Adding an element to a #SkiplistList# at a position, #i#, is fairly straightforward.  We use
the same method as with a #SkiplistSet#. We first pick the height, #k#, of the newly
inserted node, #w#, and then follow the search path for #i#.  Anytime the
search path moves down from $L_{#r#}$ with $#r#\le #k#$, we splice #w#
into $L_{#r#}$.  The only extra care needed is to ensure that the
lengths of edges are updated properly. 
See \figref{skiplist-addix}. 

\begin{figure}
  \begin{center}
    \includegraphics{figs/skiplist-addix}
  \end{center}
  \caption{Adding an element to a #SkiplistList#.}
  \figlabel{skiplist-addix}
\end{figure}

Note that, each time the search path goes down at a node, #u#, in $L_{#r#}$,
the length of the edge #u.next[r]# increases by one, since we are adding
an element below that edge at position #i#.  Splicing  the node #w# between two nodes,
#u# and #z#, works as shown in \figref{skiplist-lengths-splice}.  While
following the search path we are already keeping track of the position,
#j#, of #u# in $L_0$.  Therefore, we know that the length of the edge from
#u# to #w# is $#i#-#j#$.  We can also deduce the length of the edge
from #w#  to #z# from the length, $\ell$, of the edge from #u# to #z#.
Therefore, we can splice in #w# and update the lengths of the edges in
constant time.

\begin{figure}
  \begin{center}
    \includegraphics{figs/skiplist-lengths-splice}
  \end{center}
  \caption{Updating the lengths of edges while splicing a node 
   #w# into a skiplist.}
  \figlabel{skiplist-lengths-splice}
\end{figure}

This sounds more complicated than it actually is and the code is actually
quite simple:

\javaimport{ods/SkiplistList.add(i,x)}
\javaimport{ods/SkiplistList.add(i,w)}


By now, the implementation of 
the #remove(i)# operation in a #SkiplistList# should be obvious.  We follow the search path for the node at position #i#.  Each time the search path takes a step down from a node, #u#, at level #r# we decrement the length of the edge leaving #u# at that level.  We also check if #u.next[r]# is the element of rank #i# and, if so, splice it out of the list at that level.   An example is shown in \figref{skiplist-removei}.
\begin{figure}
  \begin{center}
    \includegraphics{figs/skiplist-removei}
  \end{center}
  \caption{Removing an element from a #SkiplistList#.}
  \figlabel{skiplist-removei}
\end{figure}
\javaimport{ods/SkiplistList.remove(i)}

\subsection{Summary}

The following theorem summarizes the performance of the #SkiplistList#
data structure:

\begin{thm}\thmlabel{skiplistlist}
  A #SkiplistList# implements the #List# interface.  A #SkiplistList#
  supports the operations #get(i)#, #set(i,x)#, #add(i,x)#, and
  #remove(i)# in $O(\log #n#)$ expected time per operation.
\end{thm}

%\section{Skiplists as Ropes}
%TODO: A section on ropes

\section{Analysis of Skiplists}
\seclabel{skiplist-analysis}

In this section, we analyze the expected height, size, and length of
the search path in a skiplist.  This section requires a background in
basic probability.  Several proofs are based on the following basic
observation about coin tosses.

\begin{lem}\lemlabel{coin-tosses}
  Let $T$ be the number of times a fair coin is tossed up to and including
  the first time the coin comes up heads.  Then $\E[T]=2$.
\end{lem}

\begin{proof}
  Suppose we stop tossing the coin the first time it comes up
  heads. Define the indicator variable
  \[ I_{i} = \left\{\begin{array}{ll}
     0 & \mbox{if the coin is tossed less than $i$ times} \\
     1 & \mbox{if the coin is tossed $i$ or more times} 
     \end{array}\right.
  \]
  Note that $I_i=1$ if and only if the first $i-1$ coin tosses are tails,
  so $\E[I_i]=\Pr\{I_i=1\}=1/2^{i-1}$.  Observe that $T$, the total
  number of coin tosses, can be written as $T=\sum_{i=1}^{\infty} I_i$.
  Therefore,
  \begin{align*}
    \E[T] & =  \E\left[\sum_{i=1}^\infty I_i\right] \\
     & =  \sum_{i=1}^\infty \E\left[I_i\right] \\
     & =  \sum_{i=1}^\infty 1/2^{i-1} \\
     & =  1 + 1/2 + 1/4 + 1/8 + \cdots \\
     & =  2 \enspace .   \qedhere
  \end{align*} 
\end{proof}

The next two lemmata tell us that skiplists have linear size:

\begin{lem}\lemlabel{skiplist-size1}
  The expected number of nodes in a skiplist containing $#n#$ elements,
  not including occurrences of the sentinel, is $2#n#$.
\end{lem}

\begin{proof}
  The probability that any particular element #x# is included in list
  $L_{#r#}$ is $1/2^{#r#}$, so the expected number of nodes in $L_{#r#}$
  is $#n#/2^{#r#}$.  Therefore, the total number of nodes in all lists is
  \[ \sum_{#r#=0}^\infty #n#/2^{#r#} = #n#(1+1/2+1/4+1/8+\cdots) = 2#n# \enspace . \qedhere \]
\end{proof}

\begin{lem}\lemlabel{skiplist-height}
  The expected height of a skiplist containing #n# elements is at most
  $\log #n# + 2$.
\end{lem}

\begin{proof}
  For each $#r#\in\{1,2,3,\ldots,\infty\}$, 
  define the indicator random variable
  \[ I_{#r#} = \left\{\begin{array}{ll}
     0 & \mbox{if $L_{#r#}$ is empty} \\
     1 & \mbox{if $L_{#r#}$ is non-empty}
     \end{array}\right.
  \]
  The height #h# of the skiplist is then given by
  \[
       #h# = \sum_{i=1}^\infty I_{#r#} \enspace .
  \]
  Note that $I_{#r#}$ is never more than the length, $|L_{#r#}|$, of $L_{#r#}$, so 
  \[
     \E[I_{#r#}] \le \E[|L_{#r#}|] = #n#/2^{#r#} \enspace .
  \]
  Therefore, we have
  \begin{align*}
       \E[#h#] &= \E\left[\sum_{r=1}^\infty I_{#r#}\right] \\
        &= \sum_{#r#=1}^{\infty} E[I_{#r#}] \\
        &= \sum_{#r#=1}^{\lfloor\log #n#\rfloor} E[I_{#r#}]
                 + \sum_{r=\lfloor\log #n#\rfloor+1}^{\infty} E[I_{#r#}]  \\
        &\le \sum_{#r#=1}^{\lfloor\log #n#\rfloor} 1
                 + \sum_{r=\lfloor\log #n#\rfloor+1}^{\infty} #n#/2^{#r#} \\
        &\le \log #n#
                 + \sum_{#r#=0}^\infty 1/2^{#r#} \\
        &= \log #n# + 2 \enspace . \qedhere
  \end{align*}
\end{proof}

\begin{lem}\lemlabel{skiplist-size2}
  The expected number of nodes in a skiplist containing $#n#$ elements,
  including all occurrences of the sentinel is $2#n#+O(\log #n#)$.
\end{lem}

\begin{proof}
  By \lemref{skiplist-size1}, the expected number of nodes, not
  including the sentinel is $2#n#$.  The number of occurrences of
  the sentinel is equal to the height, $#h#$, of the skiplist so, by
  \lemref{skiplist-height} the expected number of occurrences of the
  sentinel is at most $\log #n#+2 = O(\log #n#)$.
\end{proof}



\begin{lem}
The expected length of a search path in a skiplist is at most $2\log #n# + O(1)$.
\end{lem}

\begin{proof}
  The easiest way to see this is to consider the \emph{reverse search
  path} for a node, #x#.  This path starts at the predecessor of #x#
  in $L_0$.  At any point in time, if the path can go up a level, then
  it does.  If it can not go up a level then it goes left.  Observe that
  the reverse search path for #x# is identical to the search path for #x#,
  except that it is reversed.

  The number of nodes that the reverse search path visits at a particular
  level #r# is related to the following experiment:  Toss a coin.
  If the coin comes up heads then go up and stop, otherwise go left and
  repeat the experiment.  The number of coin tosses before the heads then
  represents the number of steps to the left that a reverse search path
  takes at a particular level.\footnote{Note that this might overcount
  the number of steps to the left, since the experiment should end either at
  the first heads or when the search path reaches the sentinel, whichever
  comes first. This is not a problem since the lemma is only stating an
  upper bound.} \lemref{coin-tosses} tells us that the expected number
  of coin tosses before the first heads is 1.

  Let $S_{#r#}$ denote the number of steps the forward search path takes at level
  $#r#$ that go to the right.   We have just argued that $\E[S_{#r#}]\le
  1$.  Furthermore, $S_{#r#}\le |L_{#r#}|$, since we can't take more steps
  in $L_{#r#}$ than the length of $L_{#r#}$, so
  \[
    \E[S_{#r#}] \le E[|L_{#r#}|] = #n#/2^{#r#} \enspace .
  \]
  We can now finish as in the proof of \lemref{skiplist-height}.
  Let $S$ be  the length of the search path for some node, #u#, in a
  skiplist, and let $#h#$ be the height of the skiplist.  Then
  \begin{align*}
      \E[S] 
         &= \E\left[ #h# + \sum_{#r#=0}^\infty S_{#r#} \right] \\
         &= \E[#h#] + \sum_{#r#=0}^\infty \E[S_{#r#}]  \\
         &= \E[#h#] + \sum_{#r#=0}^{\lfloor\log #n#\rfloor} \E[S_{#r#}] 
              + \sum_{#r#=\lfloor\log #n#\rfloor+1}^\infty \E[S_{#r#}] \\
         &\le \E[#h#] + \sum_{#r#=0}^{\lfloor\log #n#\rfloor} 1
              + \sum_{r=\lfloor\log #n#\rfloor+1}^\infty #n#/2^{#r#} \\
         &\le \E[#h#] + \sum_{#r#=0}^{\lfloor\log #n#\rfloor} 1
              + \sum_{#r#=0}^{\infty} 1/2^{#r#} \\
         &\le \E[#h#] + \sum_{#r#=0}^{\lfloor\log #n#\rfloor} 1
              + \sum_{#r#=0}^{\infty} 1/2^{#r#} \\
         &\le \E[#h#] + \log #n# + 3 \\
         &\le 2\log #n# + 5  \enspace . \qedhere
  \end{align*}
\end{proof}


The following theorem summarizes the results in this section
\begin{thm}
A skiplist containing $#n#$ elements has expected size $O(#n#)$ and the
expected length of the search path for any particular element is at most
$2\log #n# + O(1)$.
\end{thm}



%\section{Iteration and Finger Search}

%TODO: Write this section

\section{Summary and Exercises}

Skiplists were introduced in 1991 by Pugh \cite{p91} who also presented a
number of applications of skiplists \cite{p92}.  Since then they have been
studied extensively.  The exact expected length of the search path for
the #i#th element in a skiplist has been studied carefully \cite{pXX}.
Deterministic versions \cite{munroXX}, biased versions \cite{cXX}, and
self-adjusting versions \cite{bose-langerman} of skiplists have all
been developed.  Skiplist implementations have been written for
various languages and frameworks and have seen use in open-source
database systems
\cite{https://secure.wikimedia.org/wikipedia/en/wiki/Skip_list}.  A
variant of skiplists is even used in the HP-UX operating system
kernel's process management structures
\cite{http://h21007.www2.hp.com/portal/download/files/prot/files/STK/pdfs/proc_mgt.pdf}.

\begin{exc}\exclabel{skiplist-opt}
 Suppose that, instead of promoting an element from $L_{i-1}$
into $L_i$ based on a coin toss, we promote it with some probability $p$, $0 < p < 1$.
Show that the expected length of the search path in this case is at most
$(1/p)\log_{1/p} #n# + O(1)$.  What is the value of $p$ that minimizes
this expression? What is the expected height of the skiplist? What is
the expected number of nodes in the skiplist?
\end{exc}

\begin{exc}
 Design and implement a version of a skiplist that implements the
#SSet# interface, but also allows fast access to elements by rank.
That is, it also supports the function #get(i)#, which returns the
element whose rank is #i# in $O(\log #n#)$ expected time. (The rank of an element #x# in a #SSet#
is the number of elements in the #SSet# that are less than #x#.)
\end{exc}


